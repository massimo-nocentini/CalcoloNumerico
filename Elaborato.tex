\documentclass[12pt, a4paper]{report}

%\usepackage{fancyhdr}
\usepackage{hyperref}
\usepackage[T1]{fontenc}
\usepackage{ae,aecompl}
\usepackage{graphicx}
\usepackage[left=2cm, right=2cm]{geometry}
\usepackage{latexsym}
\usepackage{amsmath, amsthm, amssymb}
\usepackage{rotating}
\usepackage{boxedminipage}

% Package for including code in the document
\usepackage{listings}

% If you want to generate a toc for each chapter (use with book)
% \usepackage{minitoc}

\begin{titlepage}
\title{Calcolo Numerico} 
\author{Massimo Nocentini}
\end{titlepage}
% My Theorem
\newtheorem{oss}{Osservation}[section]
\newtheorem{exercise}{Exercise}[section]
\newtheorem{thm}{Theorem}[section]
\newtheorem{cor}[thm]{Corollary}
\newtheorem{lem}[thm]{Lemma}


% page settings
\pagestyle{headings}

\begin{document}

\lstset{
	language = octave
	, numbers = left 
	, basicstyle=\footnotesize
	, frame=single
	, tabsize=2
	, captionpos=b
	, breaklines=true
	, showspaces=false
	, showstringspaces=false
}

\maketitle

\tableofcontents

\newpage

\section*{Introduzione}
Queste note contengono tutto il mio materiale di studio per l'esame di Calcolo Numerico.

\subsection*{Sintassi esercizi}
Per gli esercizi utilizzo questa sintassi \textbf{Exercise n[(m)]}, dove $n$
rappresenta la numerazione locale all'interno delle sezioni di questo documento, 
$m$ rappresenta l'identificativo dell'esercizio fissato nel libro di testo 
\textbf{Calcolo Numerico} di \emph{Brugnano, Magherini, Sestini} pubblicato da \emph{Master}.
Il reference $(m)$ non \`e sempre presente, come evidenziato dall'uso delle parentesi quadre.

\chapter{Errori ed aritmetica finita}

\section{Discretizzazione} 

Questi errori sorgono tutte le volte che si vuole modellare un problema 
matematico (formulato nel continuo) con un sistema discreto. Per ottenere 
questo, uso dei processi di discretizzazione, dei quali mi interessa controllare
quanto ``bene'' approssimano il problema che voglio modellare.

Pi\`u specificamente, l'errore di troncamento \emph{locale} corrisponde 
all'errore introdotto nel passo di integrazione corrente assumendo esatto il 
valore di partenza, mentre l'errore di troncamento \emph{globale} rappresenta 
l'effetto di tutti gli errori precedenti.

Per i seguenti due esercizi si utilizza la formula di Taylor-Peano, con il 
resto ''infinitesimo di ordine superiore'' rispetto al massimo grado $n$ a cui 
si arresta lo sviluppo e quindi consente di ottenere una approssimazione
locale, cio\`e in un intorno di $x_{0}$.

\begin{exercise}[1.1]
	Sia $x = \pi \approx 3.14159265$. Considero come valore approssimato 
	$\tilde{x} = 3.1415$. 
	
	Calcolare il corrispondente errore relativo\footnote{D\'a l'ordine di 
	grandezza rispetto alla base $10$} $\varepsilon_{x}$. 
	
	Verificare che il numero di cifre decimali corrette nella rappresentazione 
	approssimata di $x$ mediante $\tilde{x}$ all'incirca \`e dato da $-\log_{10}
	{|\varepsilon_{x}|}$.
\end{exercise}
Ottengo:
\begin{displaymath}
	\varepsilon_{x} = \frac{\tilde{x}-x}{x} = \frac{3.1415-3.14159265358979}
	{3.14159265358979}
\end{displaymath}
\begin{lstlisting}
octave:19> (3.1415-pi)/pi
ans = -2.94925536215087e-05
\end{lstlisting}
Adesso considero:
\begin{equation*}
	-\log_{10}{|-2.9491 \times 10^{-5}|} = -\left (\log_{10}{2.9491} +
	\log_{10}{10^{-5}} \right ) = -\log_{10}{2.9491} + 5\log_{10}{10}
\end{equation*}
\begin{lstlisting}
octave:18> -log10(2.9491) + 5*log10(10)
ans =  4.53031050085890e+00
\end{lstlisting}
L'approssimazione $\tilde{x}$ ha 4 cifre decimali corrette.

\begin{exercise}
Dimostrare che $-\log_{10}{|\varepsilon_{x}|}$ d\`a all'incirca il numero 
di cifre decimali corrette di $\tilde{x}$, con cui approssimo $x$. 
\end{exercise}
\begin{proof}
Sia $r$ il numero di cifre decimali \emph{esatte}, tale che 
$r = -\log_{10}{|\varepsilon_{x}|}$. Passando agli esponenti ottengo 
$|\varepsilon_{x}| = 10^{-r}$.

Dato che cerco un'approssimazione del numero di cifre decimali a partire dall'
errore relativo, imposto due disequazioni per trovare un intervallo in cui $r$
pu\`o variare.

Scrivo in forma normalizzata il valore esatto $x$ e la sua approssimazione 
$\tilde{x}$, fissando $\beta = 10$ e $M \not = m$ perch\`e sto approssimando:
\begin{equation*}
	\begin{split}
		x \in \mathbb{R}, x \geq 0, x = m \times 10^{e} \\
		\tilde{x} \in \mathbb{R}, \tilde{x} \geq 0, \tilde{x} = M \times 10^{e}
	\end{split}
\end{equation*}
Per il \emph{teorema 1.2} vale $ 1 \leq m < 10 \wedge  1 \leq M < 10$.

Adesso voglio trovare sia una maggiorazione che una minorazione di 
$\varepsilon_{x}$:
\begin{equation*}
	\left | \varepsilon_{x} \right | = \left | \frac{(M - m) \times 10^{e}} 
		{m \times 10^{e}}
	\right | = \left | \frac{M - m}{m} \right |
\end{equation*}
Maggioro con:
\begin{equation*}
	\left | \frac{M - m}{m} \right | < 10 = max
\end{equation*}
In quanto il massimo lo ottengo quando $M = 9, m = 1$.

Minoro con:
\begin{equation*}
	min = \frac{1}{10} \le \left | \frac{M - m}{m} \right | 
\end{equation*}
In quanto il minimo lo ottengo quando $M = 8, m = 9$.

Adesso posso impostare le due disequazioni:
\begin{equation*}
	10^{-r -1} = \frac{\left | \varepsilon_{x} \right |}{10} \le 
		\left | \varepsilon_{x} \right | <
	10 \left | \varepsilon_{x} \right | = 10^{-r +1}
\end{equation*}
Le uguaglianze esterne valgono per quanto detto ad inizio prova.
%10^{-r-1} \leq |\varepsilon_{x}| < 10^{-r+1}
Passo ai logaritmi:
\begin{equation*}
	\begin{split}
		-r-1 \leq  \log_{10}{|\varepsilon_{x}|} < -r+1 \\
		r+1 \geq -\log_{10}{|\varepsilon_{x}|} > r-1
	\end{split}
\end{equation*}
Considero i rami $-r-1 \leq  \log_{10}{|\varepsilon_{x}|}$ dalla prima e 
$-\log_{10}{|\varepsilon_{x}|} > r-1$ dalla seconda per avere l'intervallo di variazione di $r$:
\begin{equation*}
	-1 - \log_{10}{|\varepsilon_{x}|} \leq r < 1 - \log_{10}{|\varepsilon_{x}|}
\end{equation*}
\end{proof}

\begin{exercise}[1.2]
	Dimostrare che, se $f(x)$ \`e sufficientemente regolare e $h>0$ \`e una 
	quantit\`a piccola, allora:
	\begin{equation*}
		\begin{split}
			\frac{f(x_{0}+h) - f(x_{0}-h)}{2h} =& f'(x_{0}) + O(h^{2}), \\
			\frac{f(x_{0}+h) - 2f(x_{0}) + f(x_{0}-h)}{h^{2}} =& f''(x_{0}) + O(h^{2}) 
		\end{split}
	\end{equation*}
\end{exercise}
Per entrambe considero gli sviluppi di Taylor di $f(x_{0}+h)$ e $f(x_{0}-h)$ in
$x_{0}$:
\begin{equation*}
	\begin{split}
		T(x_{0} + h)& = f(x_{0}) + f'(x_{0})h + \frac{f''(x_{0})h^{2}}{2} + 
		\frac{f'''(x_{0})h^{3}}{6} + O(h^{4}) \\ 
		T(x_{0} - h)& = f(x_{0}) - f'(x_{0})h + \frac{f''(x_{0})h^{2}}{2} - 
		\frac{f'''(x_{0})h^{3}}{6} + O(h^{4})
	\end{split}
\end{equation*}
Devo prestare attenzione al segno di alcuni termini dello sviluppo
\footnote{Attenzione: nello sviluppo di Taylor di una funzione $f(x)$,  le
derivate $n$-esime vengono calcolate nel punto $x_{0}$ in cui si vuole 
centrare lo sviluppo, e non nell'argomento della funzione}:
\begin{itemize}
	\item nel primo sviluppo, la combinazione lineare ammette tutti segni 
	positivi in quanto il passo di discretizzazione \`e positivo, ovvero si cerca
	di approssimare con valori $ > x_{0}$.
	\item nel secondo invece, il fattore $((x_{0} - h - x_{0} =h))^{k} < 0, 
	\forall{k=2n+1, n \in \mathbb{N}}$ fa si che i termini dello sviluppo di 
	grado dispari siano negativi, in quanto si sta discretizzando con un valore 
	minore a $x_{0}$, di conseguenza $x - x_{0} < 0$.
\end{itemize}
Sottraendo termine a termine e semplificando dove possibile ottengo la prima equazione: 
\begin{equation*}
	T(x_{0}+h) - T(x_{0}-h) = 2f'(x_{0})h + \frac{f'''(x_{0})h^{3}}{3} + O(h^{4}) 
\end{equation*}
dividendo per $2h$: 
\begin{equation*}
	\frac{T(x_{0}+h) - T(x_{0}-h)}{2h} = f'(x_{0}) + \frac{f'''(x_{0})h^{2}}{3}  +
	O(h^{3})
\end{equation*}
Osserviamo che, per $h \rightarrow 0$, la quantit\`a $O(h^{3})$ diminuisce pi\`u 
velocemente del termine $$\frac{f'''(x_{0})h^{2}}{3}$$ per cui possiamo dedurre
che si approssima la derivata prima con una quantit\`a $O(h^{2})$.

Per la seconda equazione invece che sottrarre termine a termine, sommiamo,
ottenendo:
\begin{equation*}
	\begin{split}
		T(x_{0}+h) + T(x_{0}-h) = 2f(x_{0}) + f''(x_{0})h^{2} + O(h^{4})\\
		\frac{T(x_{0}+h) - 2f(x_{0}) + T(x_{0}-h)}{h^{2}} = f''(x_{0}) + O(h^{2}) 
	\end{split}
\end{equation*}
ovvero la quantit\`a al primo membro approssima la $f''(x_{0})$ con un errore 
dell'ordine $O(h^{2})$.

\section{Errori di convergenza}

\begin{exercise}[1.3] 
\label{exercise:exerciseIterativeMethodFixedPoint}
Dimostrare che il metodo iterativo $$x_{n+1}=\phi(x_{n})$$ convergente a x*,
deve verificare la condizione di \textbf{consistenza} $$x^{*}=\phi(x^{*})$$
ovvero la soluzione cercata deve essere un punto fisso per la funzione di
iterazione che definisce il metodo.
\end{exercise}
\begin{proof}
Suppongo che il metodo $\phi$ sia monotono e convergente a $x^{*}$. Definisco
il preordine $\rightarrow$ per catturare la convergenza:
\begin{displaymath}
\rightarrow = \lbrace (x_{n}, x_{n+1}) : x_{n+1} = \phi(x_{n}) \wedge \lim_{n
\rightarrow \infty}{x_{n}} = x^{*}
\rbrace
\end{displaymath}
Suppongo \emph{per assurdo} che $$\phi(x^{*}) = x^{\triangle} \not = x^{*}$$
Per l' ipotesi $x_{n} \rightarrow x^{*}$ (per la transitivit\`a di
$\rightarrow$), posso applicare la monotonia di $\phi$ rispetto a $\rightarrow$:
$$\phi(x_{n}) \rightarrow \phi(x^{*}) = x^{\triangle}$$ ma questo contraddice 
l'ipotesi che $\phi$ converge a $x^{*}$.
\end{proof}

\begin{exercise}
Considerare il metodo iterativo:
\begin{displaymath}
	x_{n+1} = \phi(x_{n}) = \frac{1}{2} \left ( x_{n} + \frac{2}{x_{n}} \right ), 
		\quad x_{0} = 2
\end{displaymath}
dimostrare che il metodo genera una successione di approssimazioni tale che 
$x_{n} \rightarrow \sqrt{2}$.
\end{exercise}
\begin{proof}
Questo metodo \`e a passo \emph{singolo} in quanto usa un solo innesto per calcolare
$x_{n + 1}$.
Affinch\`e il metodo iterativo sia convergente, per l'esercizio 
\ref{exercise:exerciseIterativeMethodFixedPoint} posso costruire
un'equazione per trovare il punto fisso di $\phi$:
\begin{displaymath}
\phi(x) = \frac{1}{2} \left ( x + \frac{2}{x} \right ) = x
\end{displaymath}
Manipolando: $x + \frac{2}{x} = 2x \Rightarrow x^{2} + 2 = 2x^{2} \Rightarrow 
	2 = x^{2}$
\end{proof}

\begin{exercise}[1.4] Per il testo dell'esercizio consultare il libro di testo.
\end{exercise}
Eseguendo il codice riportato nella sezione \nameref{sec:iterativeMethodToSQRT2}, 
questo \`e l'output di Octave sulla mia macchina:
\begin{lstlisting}
octave:8> iterative(2, 1.5)
next =  1.42857142857143e+00
Difference with last step:0.0714285714285714
next =  1.41463414634146e+00
Difference with last step:0.0139372822299655
next =  1.41421568627451e+00
Difference with last step:0.000418460066953008
next =  1.41421356268887e+00
Difference with last step:2.12358564088966e-06
next =  1.41421356237310e+00
Difference with last step:3.15774073555986e-10
Difference with sqrt(2):0
\end{lstlisting}
In 5 passi si raggiunge la precisione richiesta, uno in pi\`u rispetto ai risultati
mostrati in \emph{Tabella 1.1} del libro.




\section{Errori di round-off}

\chapter{Sorgenti Octave}


\section{Metodo iterativo}
\label{sec:iterativeMethodToSQRT2}
\lstinputlisting{listings/chapterOne/iterative.m}


\end{document} 
