\documentclass[12pt, a4paper]{report}

\usepackage{charter}
\usepackage{makeidx}
\usepackage{fancyhdr}
\usepackage{hyperref}
\usepackage[utf8]{inputenc}
\usepackage{graphicx}
\usepackage[left=2cm, right=2cm]{geometry}
\usepackage{latexsym}
\usepackage{amsmath, amsthm, amssymb}
\usepackage{rotating}
\usepackage{boxedminipage}

% Package for including code in the document
\usepackage{listings}

% If you want to generate a toc for each chapter (use with book)
\usepackage{minitoc}

\begin{titlepage}
\title{Calcolo Numerico} 
\author{Massimo Nocentini}
\date{\today \\Firenze\\ \small{release 0.3}}
\end{titlepage}
% My Theorem
\newtheorem{exercise}{Exercise}[section]
\newtheorem{thm}{Theorem}[section]
\newtheorem{cor}[thm]{Corollary}
\newtheorem{lem}[thm]{Lemma}


% page settings
\pagestyle{fancy}

\begin{document}

\maketitle

\tableofcontents

\chapter{Errori ed aritmetica finita}

\section{Discretizzazione} 

Questi errori sorgono tutte le volte che si vuole modellare un problema 
matematico (formulato nel continuo) con un sistema discreto. Per ottenere 
questo, uso dei processi di discretizzazione, dei quali mi interessa controllare
quanto ``bene'' approssimano il problema che voglio modellare.

Pi\`u specificamente, l'errore di troncamento \emph{locale} corrisponde 
all'errore introdotto nel passo di integrazione corrente assumendo esatto il 
valore di partenza, mentre l'errore di troncamento \emph{globale} rappresenta 
l'effetto di tutti gli errori precedenti.

Per i seguenti due esercizi si utilizza la formula di Taylor-Peano, con il 
resto ''infinitesimo di ordine superiore'' rispetto al massimo grado $n$ a cui 
si arresta lo sviluppo e quindi consente di ottenere una approssimazione
locale, cio\`e in un intorno di $x_{0}$.

\begin{exercise}[1.1]
	Sia $x = \pi \approx 3.14159265$. Considero come valore approssimato 
	$\tilde{x} = 3.1415$. 
	
	Calcolare il corrispondente errore relativo\footnote{D\'a l'ordine di 
	grandezza rispetto alla base $10$} $\varepsilon_{x}$. 
	
	Verificare che il numero di cifre decimali corrette nella rappresentazione 
	approssimata di $x$ mediante $\tilde{x}$ all'incirca \`e dato da $-\log_{10}
	{|\varepsilon_{x}|}$.
\end{exercise}
Ottengo:
\begin{displaymath}
	\varepsilon_{x} = \frac{\tilde{x}-x}{x} = \frac{3.1415-3.14159265358979}
	{3.14159265358979}
\end{displaymath}
\begin{lstlisting}
octave:19> (3.1415-pi)/pi
ans = -2.94925536215087e-05
\end{lstlisting}
Adesso considero:
\begin{equation*}
	-\log_{10}{|-2.9491 \times 10^{-5}|} = -\left (\log_{10}{2.9491} +
	\log_{10}{10^{-5}} \right ) = -\log_{10}{2.9491} + 5\log_{10}{10}
\end{equation*}
\begin{lstlisting}
octave:18> -log10(2.9491) + 5*log10(10)
ans =  4.53031050085890e+00
\end{lstlisting}
L'approssimazione $\tilde{x}$ ha 4 cifre decimali corrette.

\begin{exercise}
Dimostrare che $-\log_{10}{|\varepsilon_{x}|}$ d\`a all'incirca il numero 
di cifre decimali corrette di $\tilde{x}$, con cui approssimo $x$. 
\end{exercise}
\begin{proof}
Sia $r$ il numero di cifre decimali \emph{esatte}, tale che 
$r = -\log_{10}{|\varepsilon_{x}|}$. Passando agli esponenti ottengo 
$|\varepsilon_{x}| = 10^{-r}$.

Dato che cerco un'approssimazione del numero di cifre decimali a partire dall'
errore relativo, imposto due disequazioni per trovare un intervallo in cui $r$
pu\`o variare.

Scrivo in forma normalizzata il valore esatto $x$ e la sua approssimazione 
$\tilde{x}$, fissando $\beta = 10$ e $M \not = m$ perch\`e sto approssimando:
\begin{equation*}
	\begin{split}
		x \in \mathbb{R}, x \geq 0, x = m \times 10^{e} \\
		\tilde{x} \in \mathbb{R}, \tilde{x} \geq 0, \tilde{x} = M \times 10^{e}
	\end{split}
\end{equation*}
Per il \emph{teorema 1.2} vale $ 1 \leq m < 10 \wedge  1 \leq M < 10$.

Adesso voglio trovare sia una maggiorazione che una minorazione di 
$\varepsilon_{x}$:
\begin{equation*}
	\left | \varepsilon_{x} \right | = \left | \frac{(M - m) \times 10^{e}} 
		{m \times 10^{e}}
	\right | = \left | \frac{M - m}{m} \right |
\end{equation*}
Maggioro con:
\begin{equation*}
	\left | \frac{M - m}{m} \right | < 10 = max
\end{equation*}
In quanto il massimo lo ottengo quando $M = 9, m = 1$.

Minoro con:
\begin{equation*}
	min = \frac{1}{10} \le \left | \frac{M - m}{m} \right | 
\end{equation*}
In quanto il minimo lo ottengo quando $M = 8, m = 9$.

Adesso posso impostare le due disequazioni:
\begin{equation*}
	10^{-r -1} = \frac{\left | \varepsilon_{x} \right |}{10} \le 
		\left | \varepsilon_{x} \right | <
	10 \left | \varepsilon_{x} \right | = 10^{-r +1}
\end{equation*}
Le uguaglianze esterne valgono per quanto detto ad inizio prova.
%10^{-r-1} \leq |\varepsilon_{x}| < 10^{-r+1}
Passo ai logaritmi:
\begin{equation*}
	\begin{split}
		-r-1 \leq  \log_{10}{|\varepsilon_{x}|} < -r+1 \\
		r+1 \geq -\log_{10}{|\varepsilon_{x}|} > r-1
	\end{split}
\end{equation*}
Considero i rami $-r-1 \leq  \log_{10}{|\varepsilon_{x}|}$ dalla prima e 
$-\log_{10}{|\varepsilon_{x}|} > r-1$ dalla seconda per avere l'intervallo di variazione di $r$:
\begin{equation*}
	-1 - \log_{10}{|\varepsilon_{x}|} \leq r < 1 - \log_{10}{|\varepsilon_{x}|}
\end{equation*}
\end{proof}

\begin{exercise}[1.2]
	Dimostrare che, se $f(x)$ \`e sufficientemente regolare e $h>0$ \`e una 
	quantit\`a piccola, allora:
	\begin{equation*}
		\begin{split}
			\frac{f(x_{0}+h) - f(x_{0}-h)}{2h} =& f'(x_{0}) + O(h^{2}), \\
			\frac{f(x_{0}+h) - 2f(x_{0}) + f(x_{0}-h)}{h^{2}} =& f''(x_{0}) + O(h^{2}) 
		\end{split}
	\end{equation*}
\end{exercise}
Per entrambe considero gli sviluppi di Taylor di $f(x_{0}+h)$ e $f(x_{0}-h)$ in
$x_{0}$:
\begin{equation*}
	\begin{split}
		T(x_{0} + h)& = f(x_{0}) + f'(x_{0})h + \frac{f''(x_{0})h^{2}}{2} + 
		\frac{f'''(x_{0})h^{3}}{6} + O(h^{4}) \\ 
		T(x_{0} - h)& = f(x_{0}) - f'(x_{0})h + \frac{f''(x_{0})h^{2}}{2} - 
		\frac{f'''(x_{0})h^{3}}{6} + O(h^{4})
	\end{split}
\end{equation*}
Devo prestare attenzione al segno di alcuni termini dello sviluppo
\footnote{Attenzione: nello sviluppo di Taylor di una funzione $f(x)$,  le
derivate $n$-esime vengono calcolate nel punto $x_{0}$ in cui si vuole 
centrare lo sviluppo, e non nell'argomento della funzione}:
\begin{itemize}
	\item nel primo sviluppo, la combinazione lineare ammette tutti segni 
	positivi in quanto il passo di discretizzazione \`e positivo, ovvero si cerca
	di approssimare con valori $ > x_{0}$.
	\item nel secondo invece, il fattore $((x_{0} - h - x_{0} =h))^{k} < 0, 
	\forall{k=2n+1, n \in \mathbb{N}}$ fa si che i termini dello sviluppo di 
	grado dispari siano negativi, in quanto si sta discretizzando con un valore 
	minore a $x_{0}$, di conseguenza $x - x_{0} < 0$.
\end{itemize}
Sottraendo termine a termine e semplificando dove possibile ottengo la prima equazione: 
\begin{equation*}
	T(x_{0}+h) - T(x_{0}-h) = 2f'(x_{0})h + \frac{f'''(x_{0})h^{3}}{3} + O(h^{4}) 
\end{equation*}
dividendo per $2h$: 
\begin{equation*}
	\frac{T(x_{0}+h) - T(x_{0}-h)}{2h} = f'(x_{0}) + \frac{f'''(x_{0})h^{2}}{3}  +
	O(h^{3})
\end{equation*}
Osserviamo che, per $h \rightarrow 0$, la quantit\`a $O(h^{3})$ diminuisce pi\`u 
velocemente del termine $$\frac{f'''(x_{0})h^{2}}{3}$$ per cui possiamo dedurre
che si approssima la derivata prima con una quantit\`a $O(h^{2})$.

Per la seconda equazione invece che sottrarre termine a termine, sommiamo,
ottenendo:
\begin{equation*}
	\begin{split}
		T(x_{0}+h) + T(x_{0}-h) = 2f(x_{0}) + f''(x_{0})h^{2} + O(h^{4})\\
		\frac{T(x_{0}+h) - 2f(x_{0}) + T(x_{0}-h)}{h^{2}} = f''(x_{0}) + O(h^{2}) 
	\end{split}
\end{equation*}
ovvero la quantit\`a al primo membro approssima la $f''(x_{0})$ con un errore 
dell'ordine $O(h^{2})$.

\section{Convergenza}

\begin{exercise}[1.3] 
\label{exercise:exerciseIterativeMethodFixedPoint}
Dimostrare che il metodo iterativo $$x_{n+1}=\phi(x_{n})$$ convergente a x*,
deve verificare la condizione di \textbf{consistenza} $$x^{*}=\phi(x^{*})$$
ovvero la soluzione cercata deve essere un punto fisso per la funzione di
iterazione che definisce il metodo.
\end{exercise}
\begin{proof}
Suppongo che il metodo $\phi$ sia monotono e convergente a $x^{*}$. Definisco
il preordine $\rightarrow$ per catturare la convergenza:
\begin{displaymath}
\rightarrow = \lbrace (x_{n}, x_{n+1}) : x_{n+1} = \phi(x_{n}) \wedge \lim_{n
\rightarrow \infty}{x_{n}} = x^{*}
\rbrace
\end{displaymath}
Suppongo \emph{per assurdo} che $$\phi(x^{*}) = x^{\triangle} \not = x^{*}$$
Per l' ipotesi $x_{n} \rightarrow x^{*}$ (per la transitivit\`a di
$\rightarrow$), posso applicare la monotonia di $\phi$ rispetto a $\rightarrow$:
$$\phi(x_{n}) \rightarrow \phi(x^{*}) = x^{\triangle}$$ ma questo contraddice 
l'ipotesi che $\phi$ converge a $x^{*}$.
\end{proof}

\begin{oss}
La prova precedente \`e stata scritta da me per\`o rivedendola ad un ricevimento
con la professoressa Sestini, abbiamo visto che richiede che $\phi$ sia
monotona. Questa richiesta \`e pi\`u restrittiva rispetto agli argomenti
trattati nel testo. Una prova pi\`u semplice \`e la seguente:
\end{oss}
\begin{proof}
Sia $\phi$ continua, per l'equazione $(1.2)$ del testo, vale $x_{n} \rightarrow
x^{*}$ e $x_{n + 1} \rightarrow x^{*}$. Per la definizione del metodo iterativo
si ha:
\begin{displaymath}
x_{n + 1} = \phi(x_{n})
\end{displaymath}
Dato che $\phi$ si assume continua e per l'equazione $(1.2)$ allora vale
\begin{displaymath}
\begin{split}
x_{n + 1} &= \phi(x_{n}) \\
\downarrow & \quad \downarrow \\
x^{*} &= \phi(x^{*})
\end{split}
\end{displaymath}
E questo dimostra che se il metodo \`e convergente allora soddisfa la condizione
necessaria di consistenza.
\end{proof}

\begin{oss}
L'ultima frase della precedente prova \`e importante, ovvero se un metodo
converge alla soluzione esatta $x^{*}$ allora soddisfa la condizione necessaria,
non \`e vero il contrario. Infatti un metodo potrebbe soddisfare la condizione
necessaria alla convergenza ma non convergere alla soluzione esatta.
\end{oss}

\begin{exercise}
Considerare il metodo iterativo:
\begin{displaymath}
	x_{n+1} = \phi(x_{n}) = \frac{1}{2} \left ( x_{n} + \frac{2}{x_{n}} \right ), 
		\quad x_{0} = 2
\end{displaymath}
dimostrare che il metodo genera una successione di approssimazioni tale che 
$x_{n} \rightarrow \sqrt{2}$.
\end{exercise}
\begin{proof}
Questo metodo \`e a passo \emph{singolo} in quanto usa un solo innesto per calcolare
$x_{n + 1}$.
Affinch\`e il metodo iterativo sia convergente, per l'esercizio 
\ref{exercise:exerciseIterativeMethodFixedPoint} posso costruire
un'equazione per trovare il punto fisso di $\phi$:
\begin{displaymath}
	\phi(x) =  \frac{1}{2} \left ( x + \frac{2}{x} \right ) = x 
\end{displaymath}
Manipolando: $x + \frac{2}{x} = 2x \Rightarrow x^{2} + 2 = 2x^{2} \Rightarrow 
	2 = x^{2}$.
\end{proof}

\begin{exercise}[1.4] Per il testo dell'esercizio consultare il libro di testo.
\end{exercise}
Eseguendo il codice riportato nella sezione \nameref{sec:iterativeMethodToSQRT2}, 
questo \`e l'output di Octave sulla mia macchina:
\begin{lstlisting}
octave:8> iterative(2, 1.5)
next =  1.42857142857143e+00
Difference with last step:0.0714285714285714
next =  1.41463414634146e+00
Difference with last step:0.0139372822299655
next =  1.41421568627451e+00
Difference with last step:0.000418460066953008
next =  1.41421356268887e+00
Difference with last step:2.12358564088966e-06
next =  1.41421356237310e+00
Difference with last step:3.15774073555986e-10
Difference with sqrt(2):0
\end{lstlisting}
In 5 passi si raggiunge la precisione richiesta, uno in pi\`u rispetto ai risultati
mostrati in \emph{Tabella 1.1} del libro.




\section{Round-off}

\begin{exercise}[1.5]
Per il testo dell'esercizio consultare il libro di testo.
\end{exercise}
Scrivo le rappresentazioni in formato stringa:
\begin{displaymath}
	\begin{split}
		\alpha_{0}\alpha_{1}\ldots\alpha_{15} = 0\underbrace{1\ldots1}_{13}01 & 
			\rightarrow  32765 \\
		\alpha_{0}\alpha_{1}\ldots\alpha_{15} = 0\underbrace{1\ldots1}_{13}10 & 
			\rightarrow  32766 \\
		\alpha_{0}\alpha_{1}\ldots\alpha_{15} = 0\underbrace{1\ldots1}_{13}11 & 
			\rightarrow  32767 \\
		\alpha_{0}\alpha_{1}\ldots\alpha_{15} = 1\underbrace{0\ldots0}_{15} & 
			\rightarrow  -32768 \\
	\end{split}
\end{displaymath}
Ad ogni passo incremento di uno, questo comporta che incrementando la rappresentazione
di $32767$ si "invade" il bit del segno $\alpha_{0}$. Per la definizione della funzione di 
valutazione ottengo $val(1\underbrace{0\ldots0}_{15}) = -2^{16} = -32768$.

\begin{exercise}[1.6]
$\mathbb{M}$ ha un numero finito di elementi.
\end{exercise}
\begin{proof}
Per far vedere che $\mathbb{M}$ ha un numero finito di elementi posso far \\
vedere che vale
$equinumerous(\mathbb{M}, \lbrace 1, \ldots, n\rbrace)$, con $n \in \mathbb{N}$.
Per la definizione della relazione $equinumerous$ posso costruire una funzione $f$
\emph{one-to-one} tale che $f: \mathbb{M} \rightarrow \mathbb{N}$.

Considerare la rappresentazione in formato stringa 
$x = \alpha_{0}\alpha_{1}\ldots\alpha_{m}\beta_{1}\ldots\beta_{s}$, $\forall x \in \mathbb{M}$.

Per ogni rappresentazione costruisco una nuova rappresentazione che astrae da $\alpha$ e $\beta$, 
ovvero $x = \alpha_{0}\alpha_{1}\ldots\alpha_{m}\beta_{1}\ldots\beta_{s} = \delta_{0} \ldots 
\delta_{m + s} = x'$
con $\alpha_{0} = \delta_{0}, \ldots, \alpha_{m} = \delta_{m}, \beta_{1} = \delta_{m + 1}, \ldots, 
\beta_{s} = \delta_{m + s}$.

Adesso posso costruire la funzione:
\begin{displaymath}
f(\delta_{0} \ldots \delta_{m + s}) = \sum_{i = 0}^{m + s}{\delta{i} * b^{m + s - i}}
\end{displaymath}
In questo modo ho costruito una biezione tra le rappresentazioni ed un sottoinsieme 
dei numeri naturali. \\\\
Adesso devo far vedere che $|\mathbb{M}| = n, n \in \mathbb{N}$.

Ragiono per assurdo. Suppongo che $\not \exists n \in \mathbb{N}: |\mathbb{M}| = n$.
Considero la rappresentazione con massimo valore della funzione $f$ costruita in precedenza,
ovvero sia $x = \delta_{0} \ldots \delta_{m + s} = \underbrace{1 \ldots 1}_{m+s+1}$.
Per le ipotesi di assurdo, allora posso trovare una rappresentazione $x':f(x') \geq f(x)$.
Quindi la rappresentazione di $x'$ dovr\`a essere della forma $x' = \delta_{0}' \delta_{0} 
\ldots \delta_{m + s}$,
ovvero devo aggiungere un bit $\alpha_{0}'$ affinch\`e possa rappresentare 
$x' = 1 \underbrace{0 \ldots 0}_{m+s+1}$. Ma \`e impossibile costruire una rappresentazione
con $m+s+2$ simboli in quanto la dimensione delle rappresentazioni \`e fissata, uguale a $m+s+1$ 
e questo termina la prova.

\end{proof}

\begin{exercise}[1.6]
$r_{1} = b^{-\nu} \wedge r_{2} = (1 - b^{-m})b^{\varphi}$, con $\varphi = b^{s} - \nu$.
\end{exercise}
Suppongo che $x$ sia normalizzato.
\begin{itemize}
\item La configurazione che rappresenta il minimo
numero in valore assoluto (non considero il simbolo $\alpha_{0}$) \`e:
\begin{displaymath}
	r_{1} = \alpha_{1}.\alpha_{2} \ldots \alpha_{m} \beta_{1} \ldots \beta_{s} = 
		1.\underbrace{0 \ldots 0}_{m - 1} \underbrace{0 \ldots 0}_{s}
\end{displaymath}
A cui corrisponde la valutazione $val(r_{1}) = b^{-\nu}$.

\item La configurazione che rappresenta il massimo numero in valore assoluto
(non considero il simbolo $\alpha_{0}$) \`e:
\begin{displaymath}
	r_{2} = \alpha_{1}.\alpha_{2} \ldots \alpha_{m} \beta_{1} \ldots \beta_{s} =
		(b - 1).\underbrace{(b - 1)}_{m - 1}\underbrace{(b - 1)}_{s}
\end{displaymath}
Calcolo adesso la valutazione di $r_{2} = \rho b^{e - \nu}$. Ricavo $e$:
\begin{displaymath}
		e = (b-1)\sum_{i = 1}^{s}{b^{s-i}} = (b-1)\sum_{i = 0}^{s-1}{b^{i}} =
			 (b-1)\frac{b^{s}-1}{b-1} = b^{s}-1
\end{displaymath}
Ricavo $r_{2}$:
\begin{displaymath}
	\begin{split}
		r_{2} 	&= \left ( (b-1)\sum_{i = 1}^{m}{b^{1 - i}} \right)b^{b^{s}-1-\nu} = 
 			\left ( (b-1)\sum_{i = 1}^{m}{b^{-(i - 1)}} \right)b^{b^{s}-1-\nu} = \\
	 			&= \left ( (b-1)\sum_{i = 1}^{m}{\left ( \frac{1}{b} \right ) ^{i - 1}} 
					\right)b^{b^{s}-1-\nu} = 
					\left ( (b-1)\frac{\left ( \frac{1}{b} \right ) ^{m} - 1}{
						\left ( \frac{1}{b} \right ) - 1}
					\right)b^{b^{s}-1-\nu} = \\
				&= \left ( (b-1) \frac{b \left( b^{-m} - 1 \right )}{
						b \left ( \left ( \frac{1}{b} \right ) - 1 \right ) }
					\right)b^{b^{s}-1-\nu} = 
					\left ( (b-1) \frac{b \left( b^{-m} - 1 \right )}{1 - b} 
						\right)b^{b^{s}-1-\nu} = \\
				&= \left ( (b-1) \frac{b \left( 1 - b^{-m} \right )}{b - 1} 
					\right)b^{b^{s}-1-\nu} = 
					\left ( b \left( 1 - b^{-m} \right ) \right)b^{b^{s}-1-\nu} = 
					\left( 1 - b^{-m} \right ) b^{b^{s} - \nu}
	\end{split}
\end{displaymath}

\begin{exercise}[1.6]
Per il testo dell'esercizio consultare il libro di testo.
\end{exercise}
Dato che rappresento mediante arrotondamento:
\begin{displaymath}
\begin{split}
u &= \frac{1}{2}b^{1-m} \quad , b = 10 \\
\log_{10}{u} &= \log_{10}{\left ( \frac{1}{2}10^{1-m} \right )} = -\log_{10}{2} + (1 - m) \\
m &= 1 - \log_{10}{2} - \log_{10}{u} \quad , u = 4.66 * 10^{-10}
\end{split}
\end{displaymath}
\begin{lstlisting}
octave:10> 1 - log10(2) - log10(4.66e-10)
ans =  1.00305840876460e+01
\end{lstlisting}


\end{itemize}


\lstset{language = java, numbers = left} 
\begin{lstlisting}
	private int a = 0; 
\end{lstlisting}

\chapter{Ciao}

\begin{exercise}[cio]
	Riferendosi all'esercizio precedente, dimostrare le stesse richieste, usando come passi di integrazione due quantit\`a $w, k > 0, w \not = k$. 
\end{exercise}

\begin{proof}
	Riferendosi all'esercizio precedente, dimostrare le stesse richieste, usando come passi di integrazione due quantit\`a $w, k > 0, w \not = k$. 
\end{proof}

\begin{thm}theo \end{thm}
\begin{cor} cor \end{cor}
\begin{lem}lem \end{lem}

\subsection{sub}
\begin{thm}theo \end{thm}
\end{document} 
