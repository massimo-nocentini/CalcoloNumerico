\section{Esercizi preliminari}

\begin{exercise}[3.2]
\label{exercise:32}
Se $A, B \in \mathbb{R}^{n \times n}$ triangolari inferiori valgono
rispettivamente:
\begin{displaymath}
A + B = C \quad \wedge \quad AB = D
\end{displaymath}
con $C, D \in \mathbb{R}^{n \times n}$ triangolari inferiori.
\end{exercise}
\begin{proof}[Proof of A+B=C]
Per definizione di triangolare inferiore devo dimostrare che $c_{ij} = 0$ se $i
< j$.

Per la definizione dell'operatore $+(,)$ tra matrici ottengo
\begin{displaymath}
c_{ij} = a_{ij} + b_{ij} 
\end{displaymath}
Per ipotesi $A,B$ sono triangolari inferiori, quindi $a_{ij} = b_{ij} = 0$ se
$i < j$, quindi $c_{ij} = 0$ se $i < j$ e questo termina la prova.
\end{proof}

\begin{proof}[Proof of AB=D]
Per definizione di triangolare inferiore devo dimostrare che $c_{ij} = 0$ se $i
< j$.

Per definizione dell'operatore $\cdot(,)$ tra matrici, $d_{ij}$ \`e dato dal
prodotto dell'$i$-esima riga con la $j$-esima colonna, formalmente se $i < j$
\begin{displaymath}
d_{ij} = \left ( \vect{e}_{i}^{T}A \right ) \left ( B \vect{e}_{j} \right) = 
\begin{bmatrix}
a_{i1} \cdots a_{ii} \underbrace{0 \cdots 0}_{n-i}
\end{bmatrix}
\begin{bmatrix}
0 \\
\vdots \\
0 \\
b_{i+1,j}
\\
\vdots \\
b_{nj}
\end{bmatrix} = 0
\end{displaymath}
Svolgendo il prodotto riga-colonna si osserva che le prime $i$ componenti della
colonna annullano le prime $i$ componenti della riga e le ultime $n-i$
componenti della riga annullano le ultime $n-i$ componenti della colonna, quindi
si sommano tutti elementi nulli, per cui
\begin{displaymath}
d_{ij} = \sum_{k = 1}^{n}{a_{ik}b_{kj}} = 0 \quad \text{ se } i < j
\end{displaymath} e questo termina la prova.
\end{proof}

Con argomento simmetrico si dimostra che vale lo stesso asserto se $A,B,C,D \in
\mathbb{R}^{n \times n}$ triangolari superiori.

\begin{exercise}[3.3]
Se $A,B \in \mathbb{R}^{n \times n}$ triangolari inferiori a diagonale unitaria
allora $AB=C$, con $C \in \mathbb{R}^{n \times n}$ triangolare inferiore a
diagonale unitaria.
\end{exercise}
\begin{proof}
$A,B$ sono triangolari inferiori, quindi per l'esercizio \ref{exercise:32},
anche $C$ \`e triangolare inferiore. Rimane da dimostrare che $c_{ii} = 1$ per
$i \in \lbrace 1, \ldots, n\rbrace$.

Posso scrivere $c_{ii}$ come prodotto dell'$i$-esima riga con la $i$-esima
colonna, formalmente se $i \geq j$
\begin{displaymath}
\begin{split}
c_{ii} = \left ( \vect{e}_{i}^{T}A \right ) \left ( B \vect{e}_{i} \right) & =
\begin{bmatrix}
a_{11} \\
\vdots & \ddots \\
a_{i1} & \cdots & a_{ii} \\
\vdots & 		&		& \ddots\\
a_{n1} & \cdots & \cdots &\cdots & a_{nn}
\end{bmatrix} %times 
\begin{bmatrix}
b_{11} \\
\vdots & \ddots \\
b_{i1} & \cdots & b_{ii} \\
\vdots & 		&		& \ddots\\
b_{n1} & \cdots & \cdots &\cdots & b_{nn}
\end{bmatrix} = \\ %split  
&= \begin{bmatrix}
a_{i1} & \cdots & a_{i,i-1} & a_{ii} & 0 & \cdots & 0
\end{bmatrix}
\begin{bmatrix}
0 \\
\vdots \\
0 \\
b_{i,i} \\
b_{i+1,i} \\
\vdots \\
b_{ni}
\end{bmatrix} = \sum_{k = 1}^{n}{a_{in}b_{ni}} = a_{ii}b_{ii} = 1
\end{split}
\end{displaymath}
Svolgendo il prodotto riga-colonna si osserva che le prime $i-1$ componenti
della colonna annullano le prime $i-1$ componenti della riga e le ultime $n-i$
componenti della riga annullano le ultime $n-i$ componenti della colonna, rimane
solo il termine $a_{ii}b_{ii}$, ma per ipotesi $a_{ii}=b_{ii}=1$ perch\`e $A, B$
sono a diagonale unitaria, quindi $c_{ii} = 1$ e questo termina la prova.
\end{proof}

\begin{exercise}[3.4]
\label{exercise:341}
Se $A$ \`e triangolare inferiore allora $A^{-1}$ \`e triangolare inferiore.
\end{exercise}
\begin{proof}
Prova per assurdo.

Le ipotesi di assurdo sono: $A$ triangolare inferiore ed $A^{-1}$ non
triangolare inferiore. 

Per definizione di matrice inversa, chiamo $B = A^{-1}$, $B$ non triangolare
inferiore, vale:
\begin{displaymath}
\begin{bmatrix}
a_{11} \\
\vdots & \ddots \\
a_{i1} & \cdots & a_{ii} \\
\vdots & 		&		& \ddots\\
a_{n1} & \cdots & \cdots &\cdots & a_{nn}
\end{bmatrix} %times 
\begin{bmatrix}
b_{11} & \cdots & \cdots &\cdots & b_{1n} \\
\vdots & \ddots &		&		& \vdots\\
b_{i1} & \cdots & b_{ii} &		& \vdots\\
\vdots & 		&		& \ddots & \vdots\\
b_{n1} & \cdots & \cdots &\cdots & b_{nn}
\end{bmatrix} =
\begin{bmatrix}
1 \\
 & \ddots \\
 & 		& \ddots \\
 & 		&		& \ddots\\
 &  	&  		&		& 1
\end{bmatrix}
\end{displaymath}
Studio come si ottiene la prima riga della matrice identit\`a:
\begin{displaymath}
\vect{e}_{i}^{T} I = 
\begin{bmatrix}
1 & 0 & \ldots & 0
\end{bmatrix} = 
\begin{array}{c}
a_{11}
	\begin{bmatrix}
	b_{11} & \ldots & b_{1n}
	\end{bmatrix} \\
+ 0
	\begin{bmatrix}
	b_{21} & \ldots & b_{2n}
	\end{bmatrix}  \\
\vdots \\
+ 0
	\begin{bmatrix}
	b_{n1} & \ldots & b_{nn}
	\end{bmatrix}
\end{array} =
a_{11}
	\begin{bmatrix}
	b_{11} & \ldots & b_{1n}
	\end{bmatrix} 
\end{displaymath}
Per ipotesi di assurdo $B = A^{-1}$, quindi $b_{11} = \frac{1}{a_{11}}$, e dato
che $A$ \`e triangolare inferiore, allore $a_{11}$ rendendo sensata la
divisione.

Per ipotesi di assurdo $B$ non \`e triangolare inferiore, quindi $\exists k:
b_{1k} \not = 0$ e quindi anche $a_{11}b_{1k} \not = 0$. Ma questo \`e assurdo
perch\`e deve valere 
\begin{displaymath}
\begin{split}
\begin{bmatrix}
1 & 0 & \ldots & 0
\end{bmatrix} \vect{e}_{k} &= 
\begin{bmatrix}
a_{11}b_{11} & \ldots & a_{11}b_{1n}
\end{bmatrix} \vect{e}_{k} \\
0 &= a_{11}b_{1k}
\end{split}
\end{displaymath}
Ma $a_{11}b_{1k} \not = 0$, cado in assurdo, quindi $B=A^{-1}$ deve essere
triangolare inferiore.
\end{proof}

\begin{exercise}[3.4]
Se $A$ \`e triangolare inferiore a diagonale unitaria allora $A^{-1}$ \`e
triangolare inferiore a diagonale unitaria.
\end{exercise}
\begin{proof}
Per ipotesi $A$ \`e triangolare inferiore, per l'esercizio \ref{exercise:341},
anche $A^{-1}$ \`e triangolare inferiore. Rimane da dimostrare che $b_{ii} = 1$
con $b_{ij} \in B = A^{-1}$.

Per definizione di matrice inversa, chiamo $B = A^{-1}$, vale:
\begin{displaymath}
\begin{bmatrix}
1 \\
\vdots & \ddots \\
a_{i1} & \cdots & 1 \\
\vdots & 		&		& \ddots\\
a_{n1} & \cdots & \cdots &\cdots & 1
\end{bmatrix} %times 
\begin{bmatrix}
b_{11} \\
\vdots & \ddots \\
b_{i1} & \cdots & b_{ii} \\
\vdots & 		&		& \ddots \\
b_{n1} & \cdots & \cdots &\cdots & b_{nn}
\end{bmatrix} =
\begin{bmatrix}
1 \\
 & \ddots \\
 & 		& \ddots \\
 & 		&		& \ddots\\
 &  	&  		&		& 1
\end{bmatrix}
\end{displaymath}
Prova per assurdo. Suppongo che $B$ non sia a diagonale unitaria, quindi
$b_{ii} \not = 1$.
Studio come si ottiene la prima riga della matrice identit\`a:
\begin{displaymath}
\vect{e}_{i}^{T} I = 
\begin{bmatrix}
1 & 0 & \ldots & 0
\end{bmatrix} = 
\begin{array}{c}
1
	\begin{bmatrix}
	b_{11} & 0 & \ldots & 0
	\end{bmatrix} \\
+ 0
	\begin{bmatrix}
	b_{21} & b_{22} & 0 & \ldots & 0
	\end{bmatrix}  \\
\vdots \\
+ 0
	\begin{bmatrix}
	b_{n1} & \ldots & b_{nn}
	\end{bmatrix}
\end{array} =
	\begin{bmatrix}
	b_{11} & 0 & \ldots & 0
	\end{bmatrix} 
\end{displaymath}
Affinch\`e sia vera l'uguaglianza deve valere $1 = b_{11}$. Ma questo \`e
assurdo perch\`e per ipotesi di assurdo $b_{11} \not = 1$, quindi $B=A^{-1}$ \`e
triangolare inferiore a diagonale unitaria.
\end{proof}

\begin{exercise}
Se $A$ \`e triangolare allora $\det(A) = a_{11} \cdots a_{nn}$
\end{exercise}
\begin{proof}
Suppongo che $A$ sia triangolare inferiore, lo stesso argomento si dimostra in
modo simmetrico per le matrici triangolari superiori $B$: considerando $B^{T}$
si torna ad una triangolare inferiore e quindi si pu\`o usare questa prova.
Quindi $A$ \`e della forma:
\begin{displaymath}
\begin{bmatrix}
a_{11} \\
\vdots & \ddots \\
a_{i1} & \cdots & a_{ii} \\
\vdots & 		&		& \ddots\\
a_{n1} & \cdots & \cdots &\cdots & a_{nn}
\end{bmatrix}
\end{displaymath}
Calcolo il determinante rispetto alla prima riga:
\begin{displaymath}
\det(A) = a_{11}A_{11} + a_{12}A_{12}\cdots + a_{1n}A_{1n}
\end{displaymath}
Per ipotesi $A$ \`e tringolare inferiore, quindi $a_{ij} = 0$ se $i < j$, quindi
\begin{displaymath}
\det(A) = a_{11}A_{11}
\end{displaymath}
Calcolare $A_{11}$ equivale a calcolare la funzione $\det$ della matrice
$B$ ottenuta da $A$ cancellando la prima riga e la prima colonna:
\begin{displaymath}
B = \begin{bmatrix}
a_{22} \\
\vdots & \ddots \\
a_{i2} & \cdots & a_{ii} \\
\vdots & 		&		& \ddots\\
a_{n2} & \cdots & \cdots &\cdots & a_{nn}
\end{bmatrix}
\end{displaymath}
Ma anche questa matrice \`e una tringolare inferiore, posso riapplicare lo
stesso ragionamento per il calcolo del determinante, ottenendo $\det(B) =
a_{22}B_{11}$, considerare una nuova matrice $C$ ottenuta da $B$ cancellando la
prima riga e la prima colonna. \footnote{si cancella sempre la prima riga e la
prima colonna, in quanto calcolando il determinante di una tringolare inferiore
rispetto alla prima riga, rimane solo il primo elemento $b_{11}, c_{11}, \ldots$
che coincidono rispettivamente con $a_{22}, a_{33}, \ldots$ perch\`e gli
elementi della matrice non vengono modificati nel calcolo del determinante}
Il procedimento ricorsivo si ferma quando si arriva alla sotto-matrice $X$ con
il solo elemento $x_{11} = a_{nn}$ e, per la definizione di $\det$ vale
$\det(a_{nn}) = a_{nn}$. Ricomponenti i pezzi si ottiene $\det(A) = a_{11}
\cdots a_{nn}$ e questo termina la prova.
\end{proof}

\begin{exercise}[3.5, Lemma 3.2]
$A \in \mathbb{R}^{n \times n}$ triangolare. \\
$A$ \`e non singolare sse tutti i suoi
minori principali sono non nulli. 
\end{exercise}
\begin{proof}[Proof of $\Rightarrow$]
Per ipotesi $A$ \`e tringolare e $\det(A) \not = 0$.
Devo dimostrare che $\det(A_{k}) \not = 0, \forall k \in \lbrace 1,\ldots, n
\rbrace$.

Prova per induzione completa su $k$.
\begin{description}
\item[base] per $k = 1$ ottengo $\det{A_{1}} = a_{11}$. Per ipotesi $A$ \`e
triangolare quindi $a_{11} \not = 0$, la base \`e dimostrata.
\item[ipotesi induttiva] suppongo vero $\det(A_{j}) \not = 0, \forall j \in
\lbrace 1,\ldots, k-1 \rbrace$.
\item[passo induttivo] devo dimostrare che $\det(A_{k}) \not = 0$.
Per la definizione della funzione $\det$:
\begin{displaymath}
\det(A) = a_{r1}A_{r1} + \ldots + a_{rn}A_{rn}, \quad \forall r \in {1,\ldots,n}
\end{displaymath}
con $A_{ij} = (-1)^{i+j}M_{ij}(A)$ e $M_{ij}(A)$ restituisce il determinante
della matrice ottenuta non considerando l'$i$-esima riga e la $j$-esima colonna
della matrice $A$ passata come parametro.
\end{description}

Applico la definizione della funzione $\det$ alla sotto-matrice $A_{k}$: per
ipotesi $\det(A) \not = 0$, quindi posso fissare l'indice $r = 1$ della riga
rispetto a cui calcolo il determinante in quanto $\det(A)$ non cambia rispetto
alla riga (e anche alle colonne) rispetto a cui viene calcolato. Ottengo:
\begin{displaymath}
\det(A_{k}) = a_{11}A_{k_{11}} + \ldots + a_{1n}A_{k_{1n}}
\end{displaymath}
Per ipotesi $A$ \`e triangolare quindi $a_{ij} = 0$ per $i < j$, quindi:
\begin{displaymath}
\det(A_{k}) = a_{11}A_{k_{11}}
\end{displaymath}
Ancora usando l'ipotesi che $A$ \`e triangolare, la struttura di triangolare
viene preservata da ogni sua sotto-matrice principale, ovvero $A_{o}$ \`e
tringolare $\forall o \in {1,\ldots,n}$, quindi anche $A_{k}$ \`e tringolare.
Per questo motivo il calcolo di $A_{k_{11}} = (-1)^{1+1}M_{11}(A_{k})$. Dato che
devo calcolare il determinante della nuova matrice $A'$ ottenuta da $A_{k}$ non
considerando la prima riga e la prima colonna, devo calcolare il determinante di
una nuova matrice triangolare. Calcolandolo sempre $\det(A')$ sulla prima riga
ottengo $det(A') = a_{22}A'_{22}$. Ripetendo il ragionamento fino ad arrivare
alla sotto-matrice degenere ad un elemento $A^{k-1} = a_{kk}$, ottengo che: 
\begin{displaymath}
\det(A_{k}) = a_{11}\cdots a_{kk}
\end{displaymath}
Dato che $\det(A) = a_{11} \cdots a_{nn} \not = 0$ per
ipotesi, allora anche $\det(A_{k}) \not = 0$ e questo completa il passo induttivo.
\end{proof}

\begin{proof}[Proof of $\Leftarrow$]
Per ipotesi vale $\det(A_{k}) \not = 0, \forall k \in \lbrace 1, \ldots,
n\rbrace$. Considero la sotto-matrice con $k = n$, quindi $A_{n} = A$.
Applicando la funzione $\det$ ad entrambi i membri, per l'ipotesi si
ottiene l'asserto.
\end{proof}



\begin{exercise}[3.5, Lemma 3.3]
Se $A = LU$ allora 
$\det(A_{k}) = \det(U_{k}), \forall k \in \lbrace 1,\ldots,n \rbrace$
\end{exercise}
\begin{proof}
$\det(A_{k}) = \det(L_{k}U_{k}) = \det(L_{k}) \det(U_{k})$ perch\`e $L_{k}$ e
$U_{k}$ hanno la stessa dimensione. Ma $L_{k}$ \`e tringolare inferiore a
diagonale unitaria per costruzione della \emph{fattorizzazione LU}, quindi
$\det(L_{k}) = 1$, quindi si ottiene l'asserto.
\end{proof}