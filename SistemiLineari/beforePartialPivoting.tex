\section{Before Partial Pivoting}

\begin{exercise}[3.6]
Per il testo dell'esercizio consultare il libro di testo.
\end{exercise}
\begin{proof}
Il numero di operazioni che compio \`e dato da:
\begin{displaymath}
\sum_{i = 1}^{n - i}{\left( (n-i) + 2(n-i) \right)} =
\sum_{k = 1}^{n - 1}{\left( k + 2k^{2} \right)} = \frac{n(n-1)}{2} +
\frac{2(n-1)n(2n-1)}{6} \approx \frac{2}{3}n^{3}
\end{displaymath}
\end{proof}

\begin{exercise}[3.9, Lemma 3.4]
Se $A$ \`e diagonale dominante per righe allora lo sono anche tutte le sue
sotto-matrici principali.
\end{exercise}
\begin{proof}
Per ipotesi $A$ \`e diagonale dominante per righe, quindi posso costruire questa 
disuguaglianza:
\begin{displaymath}
|a_{ii}| > \sum_{j = 1 \\j \not = i}^{n}{|a_{ij}|} \geq \sum_{j = 1 \\j \not =
i}^{k}{|a_{ij}|}, \quad k \leq n
\end{displaymath}
con $k$ indice della sotto-matrice di ordine $k$. La disuguaglianza a destra
dimostra l'asserto e la prova \`e terminata.
\end{proof}

\begin{exercise}[3.9, Lemma 3.5]
$A$ \`e diagonale dominante per righe sse $A^{T}$ \`e diagonale dominante per
colonne.
\end{exercise}
\begin{proof}[Proof of $\Rightarrow$]
Per ipotesi $A$ \`e diagonale dominante per righe, ovvero:
\begin{displaymath}
|a_{ii}| > \sum_{j = 1 \\j \not = i}^{n}{|a_{ij}|}
\end{displaymath}
Costruisco la trasposta: se $a_{ij} \in A$ allora $a_{ji} \in A^{T}$. Riscrivo
la definizione precedente:
\begin{displaymath}
|a_{ii}| > \sum_{j = 1 \\j \not = i}^{n}{|a_{ji}|}
\end{displaymath}
Ma questa \`e la definizione di matrice diagonale dominante per colonne e questo
termina la prova.
\end{proof}

\begin{proof}[Proof of $\Leftarrow$]
Con argomento simmetrico si dimostra anche l'implicazione inversa.
\end{proof}

\begin{exercise}[3.10]
Se $A = LDL^{T}$ allora $A$ \`e sdp.
\end{exercise}
\begin{proof}
Devo dimostrare che $A$ soddisfa la definizione sdp:
\begin{itemize}
\item $A = LDL^{T} = (L^{T})^{T}DL^{T} = LDL^{T} = A^{T}$, quindi $A$ \`e
simmetrica.
\item $\forall \vect{x} \in \mathbb{R}^{n}, \vect{x} \not = \vect{0}$ allora
deve valere:
\begin{displaymath}
\vect{x}^{T}A\vect{x} = \vect{x}^{T}LDL^{T}\vect{x} > 0
\end{displaymath}
Costruisco il vettore $\vect{y} = L^{T}\vect{x}$, di conseguenza $\vect{y}^{T}
= \vect{x}^{T}(L^{T})^{T} = \vect{x}^{T}L$. Quindi posso riscrivere
$\vect{y}^{T}D\vect{y} > 0$. Rappresento il prodotto $\vect{y}^{T}(D\vect{y})$:
\begin{displaymath}
\vect{y}^{T}(D\vect{y}) = 
\begin{bmatrix}
y_{1} & \cdots & y_{n}
\end{bmatrix}
\begin{bmatrix}
d_{11} \\
 & d_{22} \\
 & 		& \ddots \\
 & 		&		& \ddots\\
 &  	&  		&		& d_{nn}
\end{bmatrix}
\begin{bmatrix}
y_{1}\\
\vdots \\
y_{n}
\end{bmatrix} = 
\begin{bmatrix}
y_{1} & \cdots & y_{n}
\end{bmatrix}
\begin{bmatrix}
y_{1}d_{11}\\
\vdots \\
y_{n}d_{nn}
\end{bmatrix} =
\sum_{i = 1}^{n}{y_{i}d_{ii}} \geq 0
\end{displaymath} 
In quanto per ipotesi $d_{ii} > 0$. Per avere la somma uguale a 0, si dovrebbe
avere $\vect{y} = \vect{0}$. Ma questo non \`e possibile in quanto $\vect{y}  
= L^{T}\vect{x}$ con $L$ non singolare e $\vect{x} \not = \vect{0}$ per
definizione di sdp. Quindi si ha:
\begin{displaymath}
\sum_{i = 1}^{n}{y_{i}d_{ii}} > 0
\end{displaymath} 
questo \`e quanto richiesta dalla definizione di sdp, quindi $A$ \`e sdp e
questo termina la prova.
\end{itemize}
\end{proof}

\begin{exercise}[3.11]
Se $A$ \`e non singolare allora $A^{T}A$ e $AA^{T}$ sono sdp.
\end{exercise}
\begin{proof}
Devo verifica che $B = A^{T}A$ sia sdp quindi:
\begin{itemize}
  \item $B = A^{T}A = A^{T}(A^{T})^{T} = A^{T}A = B^{T}$, quindi $B$ \`e
  simmetrica
  \item $\forall \vect{x} \in \mathbb{R}^{n}, \vect{x} \not = \vect{0}$ allora
deve valere:
\begin{displaymath}
\vect{x}^{T}B\vect{x} = \vect{x}^{T}A^{T}A\vect{x} > 0
\end{displaymath}
Costruisco il vettore $\vect{y} = A\vect{x}$, di conseguenza $\vect{y}^{T}
= \vect{x}^{T}A^{T}$. Quindi posso riscrivere
$\vect{y}^{T}\vect{y} > 0$ e rappresento:
\begin{displaymath}
\vect{y}^{T}\vect{y} = 
\begin{bmatrix}
y_{1} & \cdots & y_{n}
\end{bmatrix}
\begin{bmatrix}
y_{1}\\
\vdots \\
y_{n}
\end{bmatrix} =
\sum_{i = 1}^{n}{y_{i}^{2}} \geq 0
\end{displaymath} 
Per avere la somma uguale a 0, si dovrebbe
avere $\vect{y} = \vect{0}$. Ma questo non \`e possibile in quanto $\vect{y}  
= A\vect{x}$ con $A$ non singolare, quindi $A$ non \`e la matrice nulla, e
$\vect{x} \not = \vect{0}$ per definizione di sdp. Quindi si ha:
\begin{displaymath}
\sum_{i = 1}^{n}{y_{i}^{2}} > 0
\end{displaymath} 
questo \`e quanto richiesta dalla definizione di sdp, quindi $B$ \`e sdp.

Dato che ho dimostrato che $B$ \`e simmetrica allora anche $B^{T}$ \`e sdp e
questo termina la prova.
\end{itemize}
\end{proof}
