\section{Before Partial Pivoting}

\begin{exercise}[3.6]
Per il testo dell'esercizio consultare il libro di testo.
\end{exercise}
\begin{proof}
Il numero di operazioni che compio \`e dato da:
\begin{displaymath}
\sum_{i = 1}^{n - i}{\left( (n-i) + 2(n-i) \right)} =
\sum_{k = 1}^{n - 1}{\left( k + 2k^{2} \right)} = \frac{n(n-1)}{2} +
\frac{2(n-1)n(2n-1)}{6} \approx \frac{2}{3}n^{3}
\end{displaymath}
\end{proof}

\begin{exercise}[3.9, Lemma 3.4]
Se $A$ \`e diagonale dominante per righe allora lo sono anche tutte le sue
sotto-matrici principali.
\end{exercise}
\begin{proof}
Per ipotesi $A$ \`e diagonale dominante per righe, quindi posso costruire questa 
disuguaglianza:
\begin{displaymath}
|a_{ii}| > \sum_{j = 1 \\j \not = i}^{n}{|a_{ij}|} \geq \sum_{j = 1 \\j \not =
i}^{k}{|a_{ij}|}, \quad k \leq n
\end{displaymath}
con $k$ indice della sotto-matrice di ordine $k$. La disuguaglianza a destra
dimostra l'asserto e la prova \`e terminata.
\end{proof}

\begin{exercise}[3.9, Lemma 3.5]
$A$ \`e diagonale dominante per righe sse $A^{T}$ \`e diagonale dominante per
colonne.
\end{exercise}
\begin{proof}[Proof of $\Rightarrow$]
Per ipotesi $A$ \`e diagonale dominante per righe, ovvero:
\begin{displaymath}
|a_{ii}| > \sum_{j = 1 \\j \not = i}^{n}{|a_{ij}|}
\end{displaymath}
Costruisco la trasposta: se $a_{ij} \in A$ allora $a_{ji} \in A^{T}$. Riscrivo
la definizione precedente:
\begin{displaymath}
|a_{ii}| > \sum_{j = 1 \\j \not = i}^{n}{|a_{ji}|}
\end{displaymath}
Ma questa \`e la definizione di matrice diagonale dominante per colonne e questo
termina la prova.
\end{proof}

\begin{proof}[Proof of $\Leftarrow$]
Con argomento simmetrico si dimostra anche l'implicazione inversa.
\end{proof}