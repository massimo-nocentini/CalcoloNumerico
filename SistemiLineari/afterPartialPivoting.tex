\section{After Partial Pivoting}

\begin{exercise}[3.21, 3.22]
Per i testi degli esercizi consultare il libro di testo.
\end{exercise}
Vedere il codice \nameref{subsection:PALUmethod}.

\begin{exercise}[3.23]
Per il testo dell'esercizio consultare il libro di testo.
\end{exercise}
\begin{itemize}
  \item applicazione al sistema dell'esercizio \ref{exercise:324}: si
osserva che usando il pivoting vengono trovate le soluzioni esatte.
\begin{lstlisting}
octave:25> A
A =
   2.22044604925031e-16   1.00000000000000e+00
   1.00000000000000e+00   0.00000000000000e+00
octave:26> b
b =
   1.00000000000000e+00
   2.50000000000000e-01
octave:27> [L, U, p, unknowns] = PALUmethod(A, b)
L =
   1.00000000000000e+00   0.00000000000000e+00
   2.22044604925031e-16   1.00000000000000e+00
U =
   1.00000000000000e+00   0.00000000000000e+00
   0.00000000000000e+00   1.00000000000000e+00
p =
   2.00000000000000e+00   1.00000000000000e+00
unknowns =
   2.50000000000000e-01
   1.00000000000000e+00
\end{lstlisting}

\item applicazione al primo sistema dell'esercizio \ref{exercise:324}:
\begin{lstlisting}
octave:42> A = [ 
1 0 0 0 0 0 0 0 0 0; 
100 1 0 0 0 0 0 0 0 0; 
0 100 1 0 0 0 0 0 0 0; 
0 0 100 1 0 0 0 0 0 0; 
0 0 0 100 1 0 0 0 0 0; 
0 0 0 0 100 1 0 0 0 0; 
0 0 0 0 0 100 1 0 0 0; 
0 0 0 0 0 0 100 1 0 0; 
0 0 0 0 0 0 0 100 1 0; 
0 0 0 0 0 0 0 0 100 1];
octave:43> x = [1 101*ones(1,9)]';
octave:45> [L, U, p, unknowns] = PALUmethod(A, x);
octave:46> format
octave:47> p
p =
    2    3    4    5    6    7    8    9   10    1
octave:50> format long e
octave:51> unknowns'
ans =
 Columns 1 through 4:
   1.00000000000000e+00   1.00000000000000e+00   9.99999999999925e-01   1.00000000000755e+00
 Columns 5 through 8:
   9.99999999245411e-01   1.00000007545890e+00   9.99992454110141e-01   1.00075458898588e+00
 Columns 9 and 10:
   9.24541101412386e-01   8.54588985876136e+00
\end{lstlisting}

\item applicazione al secondo sistema dell'esercizio \ref{exercise:324}:
\begin{lstlisting}
octave:42> A = [ 
1 0 0 0 0 0 0 0 0 0; 
100 1 0 0 0 0 0 0 0 0; 
0 100 1 0 0 0 0 0 0 0; 
0 0 100 1 0 0 0 0 0 0; 
0 0 0 100 1 0 0 0 0 0; 
0 0 0 0 100 1 0 0 0 0; 
0 0 0 0 0 100 1 0 0 0; 
0 0 0 0 0 0 100 1 0 0; 
0 0 0 0 0 0 0 100 1 0; 
0 0 0 0 0 0 0 0 100 1];
octave:43> x = [1 101*ones(1,9)]';
octave:54> y = .1*x;
octave:55> [L, U, p, unknowns] = PALUmethod(A, y);
octave:56> format
octave:57> p
p =
    2    3    4    5    6    7    8    9   10    1
octave:58> format long e
octave:59> unknowns'
ans =
 Columns 1 through 4:
   1.00000000000000e-01   1.00000000000001e-01   9.99999999998550e-02   1.00000000014504e-01
 Columns 5 through 8:
   9.99999985496383e-02   1.00000145036173e-01   9.99854963826791e-02   1.01450361732092e-01
 Columns 9 and 10:
  -4.50361732092139e-02   1.46036173209214e+01
\end{lstlisting} 
\item applicazione al sistema $B\vect{x} = \vect{f}$:
\begin{lstlisting}
octave:59> format 
octave:60> B = [0 1 1; 1 0 1; 2 3 4]
B =
   0   1   1
   1   0   1
   2   3   4
octave:61> f = [1;0;3];
octave:62> [L, U, p, unknowns] = PALUmethod(B, f);
octave:63> L
L =

   1.00000000000000e+00   0.00000000000000e+00   0.00000000000000e+00
   5.00000000000000e-01   1.00000000000000e+00   0.00000000000000e+00
   0.00000000000000e+00  -6.66666666666667e-01   1.00000000000000e+00

octave:64> U
U =
   2.00000000000000e+00   3.00000000000000e+00   4.00000000000000e+00
   0.00000000000000e+00  -1.50000000000000e+00  -1.00000000000000e+00
   0.00000000000000e+00   0.00000000000000e+00   3.33333333333333e-01
octave:66> p
p =
   3   2   1
octave:67> format long e
octave:68> unknowns 
unknowns =
   0.00000000000000e+00
   1.00000000000000e+00
   0.00000000000000e+00
octave:72> B*unknowns - f
ans =
   0
   0
   0
\end{lstlisting}
\end{itemize}

\begin{exercise}[3.24]
\label{exercise:324}
Per il testo dell'esercizio consultare il libro di testo.
\end{exercise}
\begin{lstlisting}
octave:8> A = [eps 1; 1 0], b = [1; 1/4]
A =
   2.22044604925031e-16   1.00000000000000e+00
   1.00000000000000e+00   0.00000000000000e+00
b =
   1.00000000000000e+00
   2.50000000000000e-01
octave:9> A\b
ans =
   2.50000000000000e-01
   1.00000000000000e+00
octave:10> L = [1 0; 1/eps 1], U = [eps 1; 0 -1/eps]
L =
   1.00000000000000e+00   0.00000000000000e+00
   4.50359962737050e+15   1.00000000000000e+00
U =
   2.22044604925031e-16   1.00000000000000e+00
   0.00000000000000e+00  -4.50359962737050e+15
octave:11> (L*U) - A
ans =
   0.00000000000000e+00   0.00000000000000e+00
   0.00000000000000e+00   0.00000000000000e+00
octave:12> U\(L\b)
ans =
   0.00000000000000e+00
   1.00000000000000e+00
\end{lstlisting}
La riga \emph{octave:9} restituisce la soluzione esatta del sistema, ovvero il
vettore $\vect{x} = \bigl [ \begin{smallmatrix}
.25 \\
1
\end{smallmatrix}\bigr ]$.

La riga \emph{octave:11} dimostra che $A = LU$ quindi $0 = LU - A$, le matrici
di fattorizzazione sono corrette.

La riga \emph{octave:12} dimostra che il metodo di fattorizzazione $LU$ senza
pivoting risulta essere mal condizionato, infatti la prima componente
$\tilde{x}_{1} = 0 \not = .25 = x_{1}$.

\begin{exercise}[3.25]
Per il testo dell'esercizio consultare il libro di testo.
\end{exercise}
Per calcolare $k_{\infty}(A)$ \`e necessario calcolare l'inversa:
\begin{displaymath}
\begin{bmatrix}
1 \\
-100 & 1 \\ 
100^{2} & -100 & 1\\ 
-100^{4} & 100^{2} & -100 & 1\\
\vdots & & & & \ddots \\
-100^{16} & 100^{14} & -100^{12} & \cdots & \cdots &  1
\end{bmatrix} = A^{-1}
\end{displaymath}
Questo l'output del codice:
\begin{lstlisting}
octave:35> cond(A,Inf)
warning: inverse: matrix singular to machine precision, rcond = 9.80198e-21
ans =  1.0202e+20
\end{lstlisting}

\begin{exercise}[3.26]
\label{exercise:326}
Per il testo dell'esercizio consultare il libro di testo.
\end{exercise}
Questo il codice che verifica che i vettori $\vect{x}, \vect{y}$ sono soluzioni
rispettivamente delle equazioni $A\vect{x} = \vect{b}, A\vect{y} = \vect{c}$:
\begin{lstlisting}
octave:41> A
A =

     1     0     0     0     0     0     0     0     0     0
   100     1     0     0     0     0     0     0     0     0
     0   100     1     0     0     0     0     0     0     0
     0     0   100     1     0     0     0     0     0     0
     0     0     0   100     1     0     0     0     0     0
     0     0     0     0   100     1     0     0     0     0
     0     0     0     0     0   100     1     0     0     0
     0     0     0     0     0     0   100     1     0     0
     0     0     0     0     0     0     0   100     1     0
     0     0     0     0     0     0     0     0   100     1

octave:42> b = [1;101;101;101;101;101;101;101;101;101];
octave:43> x = [1;1;1;1;1;1;1;1;1;1];
octave:44> A*x - b
ans =
   0
   0
   0
   0
   0
   0
   0
   0
   0
   0
octave:45> c = .1*b;
octave:46> y = [.1;.1;.1;.1;.1;.1;.1;.1;.1;.1];
octave:47> A*y - c
ans =
   0
   0
   0
   0
   0
   0
   0
   0
   0
   0
\end{lstlisting}
Implemento le istruzioni:
\begin{lstlisting}
octave:50> b = [1 101*ones(1,9)]';
octave:51> x(1) = b(1);
octave:52> for i=2:10, x(i) = b(i) - 100*x(i-1); end
octave:53> x = x(:)
x =
   1
   1
   1
   1
   1
   1
   1
   1
   1
   1
octave:54> c = .1*[1 101*ones(1,9)]';
octave:55> y(1) = c(1);
octave:56> for i=2:10, y(i) = c(i) - 100*y(i-1); end
octave:57> y = y(:)
y =
    0.100000
    0.100000
    0.100000
    0.100000
    0.100000
    0.100000
    0.099986
    0.101407
   -0.040702
   14.170153
octave:62> (y-yexact) ./ yexact  
ans =
     0.00000
     0.00000
    -0.00000
     0.00000
    -0.00000
     0.00000
    -0.00014
     0.01407
    -1.40702
   140.70153
\end{lstlisting}
Si osserva che il vettore $\vect{y}$, soluzione esatta del secondo sistema, non
\`e uguale al vettore approssimato dal secondo set di istruzioni: il comando
\emph{octave:62} calcola il vettore degli errori relativi che si commettono e si
nota che per l'ultima componente si ha un errore molto significativo. Questo \`e
dovuto dal fatto che le componenti del vettore dei termini noti $\vect{c}$ non
\`e possibile rappresentarle correttamente in macchina (0.1 ha rappresentazione
infinita periodica in binario). Questi errori di rappresentazione si sommano
nelle 9 iterazioni che vengono fatte dal ciclo for, risultando significative per
la decima componente.
