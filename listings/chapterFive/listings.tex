\section{Formule di quadratura}

\subsection{Exercise 5.4 on textbook - Trapezi composita}
\label{subsec:exercise54}
\lstinputlisting{listings/chapterFive/trapeziComposita.m} 
In questa implementazione si fissa $\xi = 1$ in quanto permette di annullare la
met\`a dei termini della derivata seconda.

\subsection{Exercise 5.5 on textbook - Simpson composita}
\label{subsec:exercise55}
\lstinputlisting{listings/chapterFive/simpsonComposita.m}
In questa implementazione si fissa $\xi = 1$ in quanto permette di annullare la
met\`a dei termini della derivata quarta.

\subsection{Funzione (5.17)}
\label{subsec:function517} 
\lstinputlisting{listings/chapterFive/function517.m}

\subsection{Derivata Seconda della funzione (5.17)}
\label{subsec:secondDer517}
\lstinputlisting{listings/chapterFive/secondDer517.m}

\subsection{Derivata Quarta della funzione (5.17)}
\label{subsec:fourthDer517}
\lstinputlisting{listings/chapterFive/fourthDer517.m}

\subsection{Plotter della funzione (5.17)}
\label{subsec:functionPlotter517}
\lstinputlisting{listings/chapterFive/analizer517.m}

\subsection{Solver exercise 5.9 on textbook}
\label{subsec:exercise59solver}
\lstinputlisting{listings/chapterFive/exercise59solver.m}

\subsection{Adaptive Trapezi}
\label{subsec:adaptiveTrapezi}
\lstinputlisting{listings/chapterFive/adaptiveTrapezi.m}

\subsection{Adaptive Simpson}
\label{subsec:adaptiveSimpson}
\lstinputlisting{listings/chapterFive/adaptiveSimpson.m}

\subsection{Solver exercise 5.10 on textbook}
\label{subsec:exercise510solver}
\lstinputlisting{listings/chapterFive/exercise510solver.m}