\section{Utility functions}
\subsection{invokeDelegate}
\lstinputlisting{listings/chapterTwo/invokeDelegate.m}

\subsection{prepareForPlottingMethodSegments}
\lstinputlisting{listings/chapterTwo/prepareForPlottingMethodSegments.m}

\section{Metodo di bisezione}
\label{sec:bisectionIterativeMethod}
\lstinputlisting{listings/chapterTwo/bisectionMethod.m}

\section{Metodo di Newton}
\label{sec:newtonIterativeMethod}
Questa implementazione utilizza il criterio di arresto esposto nell'
\emph{Osservazione 2.3} nel caso di radici semplici.

Una osservazione sul numero di passi effettuati: per la mia implementazione
vale $i = length(ascisse) + 2$ in quanto nel ciclo \emph{while} colleziono
$length(ascisse)$ valori, pi\`u uno per il valore di innesco iniziale, pi\`u uno
per l'ultimo valore appena usciti dal ciclo \emph{while}.
\lstinputlisting{listings/chapterTwo/newtonMethod.m}

\section{Varianti del metodo di Newton}

\subsection{Molteplicit\`a dello zero nota}
\label{subsec:newtonMethodMultKnown}
Questa implementazione utilizza il criterio di arresto esposto nell'
\emph{Osservazione 2.3} nel caso di radici multiple.
\lstinputlisting{listings/chapterTwo/newtonMethodMoltKnown.m}

\subsection{Molteplicit\`a dello zero nota con criterio di arresto lineare}
\label{subsec:newtonMethodMultKnownLinearStopCriteria}
Questa implementazione utilizza il criterio di arresto esposto nell'
\emph{Osservazione 2.3} nel caso di radici multiple, ovvero il criterio di
arresto sar\`a dato da:
\begin{displaymath}
	|x_{i+1} - x_{i}| \leq \frac{1-c}{c}tol_{X}
\end{displaymath}
Ricavo in modo dinamico la costante $c$, usando le tre iterate pi\`u recenti:
\begin{displaymath}
	c \approx \frac{|x_{i} - x_{i-1}|}{|x_{i-1} - x_{i-2}|}
\end{displaymath}
Posso usare questa approssimazione in quanto $m > 1$ per ipotesi del problema,
quindi per il \emph{Teorema 2.2}, il metodo di Newton converge linearmente verso zeri
multipli, 
condizione necessaria per applicare questo criterio di arresto.
\lstinputlisting{listings/chapterTwo/newtonMethodMoltKnownLinearCriteria.m}
