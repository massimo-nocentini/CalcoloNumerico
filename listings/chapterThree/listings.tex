\section{Fattorizzazioni}
\subsection{triangularSystemSolver}
\lstinputlisting{listings/chapterThree/triangularSystemSolver.m}

\subsection{normalizationEngine}
\lstinputlisting{listings/chapterThree/normalizationEngine.m}

\subsection{LUmethod}
\label{subsection:LUmethod}
\lstinputlisting{listings/chapterThree/LUmethod.m}

\subsection{LDLmethod}
\label{subsection:LDLmethod}
\begin{oss}[Struttura matriciale degli aggiornamenti del metodo al $j$-esimo
passo di iterazione]
\begin{displaymath}
\begin{bmatrix}
\Xi \\
 &  \ddots \\
 & &  \Xi \\
\Theta & \cdots & \Theta & a_{jj}^{*} &  &  &    
	 \\ \gamma & \cdots & \gamma & \Delta^{*}
	 \\ \vdots & \vdots &  & \vdots
	 \\ \gamma & \cdots & \gamma & \Delta^{*}
\end{bmatrix}
\end{displaymath}
Con $\Delta$ ho indicato elementi appartenti al vettore caratteristico,  con
$\Gamma$ gli elementi della matrice che vengono letti per calcolare $l_{ij}, i
\geq j$; con $\Theta$ gli elementi che uso per costruire il vettore di elementi 
$l_{jk}, k \in \{1, \ldots, j-1 \}$; con $\Xi$ gli elementi che leggo 
per costruire il vettore $d_{k}, k \in \{1, \ldots, j-1 \}$ (il prodotto
componente a componente $\vect{\Xi}.*\vect{\Theta} = \vect{v}$ (usando notazione
Octave), $\vect{v}$ \`e il vettore che viene costruito nel codice nell'errata
corrige; con * indico gli elementi che vengono aggiornati ad ogni passo del
metodo.
\\ \\
Risolvo in quest'ordine:
\begin{enumerate}
  \item $\vect{v} \leftarrow \vect{\Xi}.*\vect{\Theta}$
  \item $a_{jj} \leftarrow a_{jj} - \vect{\Theta}^{T} * \vect{v}$
  \item nella prossima equazione $\gamma \in \Gamma \in \mathbb{R}^{k \times
  k}$:
  \begin{displaymath}
	  \vect{\Delta} \leftarrow \frac{\vect{\Delta} - \Gamma *
	  \vect{v}}{a_{jj}^{*}}
  \end{displaymath}
\end{enumerate}
Tutti i vettori sono da intendersi vettori colonna e la relazione $\leftarrow$,
$(a, b) \in \leftarrow$, significa assegno $b$ ad $a$.
\end{oss}
\lstinputlisting{listings/chapterThree/LDLmethod.m}

\subsection{PALUmethod}
\label{subsection:PALUmethod}
\begin{oss}[Struttura matriciale degli aggiornamenti del metodo all'$i$-esimo
passo di iterazione]
\begin{displaymath}
\left [
\begin{array}{cccccc}
\\
\\
& & a_{ii}^{*} & \Theta & \cdots & \Theta   
	 \\ & & \Delta^{*} & \Gamma^{*} & \cdots & \Gamma^{*}
	 \\ & & \vdots & \vdots &  & \vdots
	 \\ & & \Delta^{*} & \Gamma^{*} & \cdots & \Gamma^{*}
\end{array}
\right ]
\end{displaymath}
Con $\Delta$ ho indicato elementi appartenti al vettore caratteristico di
Gauss, con $\Gamma$ elementi della matrice che vengono modificati dal
passo di aggiornamento del metodo, con $\Theta$ gli elementi che uso per
costruire la matrice da sottrarre a $I\vect{v}_{i}$ ($\vect{v}_{i}$ il vettore 
colonna che voglio annullare secondo il metodo) e con * indico gli elementi che
vengono aggiornati ad ogni passo del metodo.
\\ \\
Risolvo in quest'ordine:
\begin{enumerate}
  \item trovo l'elemento pivot e eventualmente scambio le righe, questo produce
  	un eventuale aggiornamento di $a_{ii}$
  \item normalizzo il vettore di Gauss:
  	\begin{displaymath}\vect{\Delta} \leftarrow
  		\frac{\vect{\Delta}}{a_{ii}^{*}}
	\end{displaymath}
  \item $a_{jj} \leftarrow a_{jj} - \vect{\Theta}^{T} * \vect{v}$
  \item nella prossima equazione $\gamma \in \Gamma \in \mathbb{R}^{k \times
  k}$:
  \begin{displaymath}
	  \Gamma \leftarrow \Gamma - \vect{\Delta}^{*} * \vect{\Theta}^{T}
  \end{displaymath}
\end{enumerate}
Tutti i vettori sono da intendersi vettori colonna e la relazione $\leftarrow$,
$(a, b) \in \leftarrow$, significa assegno $b$ ad $a$.
\end{oss}
\lstinputlisting{listings/chapterThree/PALUmethod.m}

\subsection{QRmethod}
\label{subsection:QRmethod}
\begin{oss}[Sull'uso e costruzione delle matrici di eliminazione $H$]
Durante il metodo non vengono create e sviluppate in modo esplicito le matrici
di eliminazione $H^{i}$ per effettuare i prodotti $H^{i}\vect{v}$: vengono
invece costruite in modo implicito, ovvero la computazione che viene effettuata
dal metodo \`e direttamente la trasformazione ortogonale 
\begin{displaymath}
H\vect{v} \rightarrow \vect{v} -
\frac{2}{\vect{z}^{T}\vect{z}}\vect{z}(\vect{z}^{T}\vect{v})
\end{displaymath}
\end{oss}

\begin{oss}[Sul costo del prodotto scalare]
Dati due vettori $\vect{a}, \vect{b} \in \mathbb{R}^{k}$, il prodotto scalare
costa $2k - 1$ operazioni, di cui $k$ per i prodotti $\forall i \in
\{1, \ldots , k \} [a_{i}b_{i}]$, e $k-1$ somme dei $k$ prodotti.
\end{oss}

\begin{oss}[Idee alla base del metodo]
Queste sono le idee alla base del metodo:
\begin{itemize}
  \item $H(\beta \vect{z}) = H(\vect{z}), \forall \beta \in \mathbb{R}$, ovvero
  la matrice di eliminazione non varia per multipli del vettore caratteristico
  $\vect{z}$ di Householder (permette quindi di scegliere $\beta$ in modo
  da avere il vettore caratteristico con una struttura particolare, $z_{1} =
  1$).
  
  \item la seguente ugualianza permette di evitare il calcolo del prodotto
  scalare, risparmiando $2m$ operazioni:
  	\begin{displaymath}
  		-\frac{z_{1}}{\alpha} =
  		\frac{2}{\vect{z}^{T}\vect{z}}
  	\end{displaymath}
  vedi esercizio \ref{exercise:exercise328}.
  
  \item $v_{1} = A(i, i) - \alpha$ con $\alpha A(i, i) < 0$, permette di
  evitare il fenomeno della cancellazione numerica.
\end{itemize}
\end{oss}

\begin{oss}[Sulla struttura delle matrici di eliminazione $H_{k}$]
Ragionando a blocchi sulla struttura delle matrici di eliminazione vale:
\begin{displaymath}
H_{i+1} = \left [ \begin{array}{c|c}
I_{i} 		& 	\vect{0}^{T} \\
\hline
\vect{0}	&	H^{(i+1)}
\end{array} \right ]
\end{displaymath}
con $H^{(i+1)}$ elementare di Householder relativa alla colonna $i+1$ di
$A^{(i)}$:
\begin{displaymath}
A^{(i)}\vect{e}_{i+1} = \begin{bmatrix}
a_{i+1, i+1}^{(i)} \\
\vdots \\
a_{m, i+1}^{(i)}
\end{bmatrix}
\end{displaymath}
Il precedente vettore \`e il vettore che si vuole rendere uguale a
$\alpha\vect{e}_{i+1}$.
\\\\
Al passo $i$-esimo ottengo:
\begin{displaymath}
\begin{split}
A^{(i)} &= H_{i}A^{(i-1)} = H_{i} \cdots H_{1}A  =
\begin{bmatrix}
a_{11}^{(1)} & a_{12}^{(1)} & \cdots & a_{1i}^{(1)}	& \cdots & \cdots &
a_{1n}^{(1)} \\ 
	& a_{22}^{(2)} & \cdots	& a_{2i}^{(2)} & \cdots & \cdots & a_{2n}^{(2)} \\
	&	&	\ddots	& \vdots &	&	& \vdots \\
	&	&	& a_{i-1,i-1}^{(i-1)}	& a_{i-1,i}^{(i-1)} & \cdots & a_{i-1,n}^{(i-1)}  \\
	&	&	& 	&  \\
 	& 	&	&	&  & H^{(i)}A_{ii}^{(i-1)} &  \\
	& 	&	&	& 	&		&  
\end{bmatrix} = \\ 
& = \begin{bmatrix}
a_{11}^{(1)} & a_{12}^{(1)} & \cdots & a_{1i}^{(1)}	& \cdots & \cdots &
a_{1n}^{(1)} \\ 
	& a_{22}^{(2)} & \cdots	& a_{2i}^{(2)} & \cdots & \cdots & a_{2n}^{(2)} \\
	&	&	\ddots	& \vdots &	&	& \vdots \\
	&	&	& a_{ii}^{(i)}	& a_{i,i+1}^{(i)} & \cdots & a_{in}^{(i)}  \\
	&	&	& 	& a_{i+1,i+1}^{(i)} & \cdots & a_{i+1,n}^{(i)}  \\
 	& 	&	&	& \vdots & \ddots & \vdots \\
	& 	&	&	& a_{m, i+1}^{(i)}	&	\cdots	& a_{m, n}^{(i)} 
\end{bmatrix}
\end{split}
\end{displaymath}
Attenzione: al primo passo $a_{11}^{(1)}$ viene aggiornato differentemente per
quanto avviene per la fattorizzazione LU senza pivoting.
\end{oss}

\begin{oss}[Forma matriciale del passo di aggiornamento]
\begin{displaymath}
\begin{bmatrix}
a_{i+1, j}^{(i+1)} \\
\vdots \\
a_{m, j}^{(i+1)}
\end{bmatrix} = 
\begin{bmatrix}
a_{i+1, j}^{(i)} \\
\vdots \\
a_{m, j}^{(i)}
\end{bmatrix} - \frac{2}{(\vect{z}^{(i+1)})^{T} \vect{z}^{(i+1)}} %times
\vect{z}^{(i+1)} %times 
\left ( (\vect{z}^{(i+1)})^{T} \begin{bmatrix}
a_{i+1, j}^{(i+1)} \\
\vdots \\
a_{m, j}^{(i+1)} 
\end{bmatrix} \right )
\end{displaymath}
La precedente uguaglianza rappresenta in forma matriciale l'istruzione
\emph{Octave} per l'aggiornamento della matrice (e implicitamente la
costruzione della matrice triangolare superiore $R$ (no $\hat{R}$)).
\end{oss}

\begin{oss}[Struttura matriciale degli aggiornamenti del metodo all'$i$-esimo
passo di iterazione]
\begin{displaymath}
\begin{bmatrix}
\\
\\
& & a_{ii}^{*} & \Gamma^{*} & \cdots & \Gamma^{*}   
	 \\ & & \Delta^{*} & \Gamma^{*} & \cdots & \Gamma^{*}
	 \\ & & \vdots & \vdots &  & \vdots
	 \\ & & \Delta^{*} & \Gamma^{*} & \cdots & \Gamma^{*}
\end{bmatrix}
\end{displaymath}
Con $\Delta$ ho indicato elementi appartenti al vettore caratteristico di
Householder, con $\Gamma$ elementi della matrice che vengono modificati dal
passo di aggiornamento del metodo e con * indico gli elementi che vengono
aggiornati ad ogni passo del metodo.
\\ \\
Risolvo in quest'ordine:
\begin{enumerate}
  \item calcolo la norma $\alpha$ ed eventualmente cambio segno per evitare
  	il fenomeno della cancellazione numerica nel calcolo di $v_{1}$, in modo da 
  	avere $A(i, i)\alpha < 0$
  \item $v_{1} \leftarrow A(i, i) - \alpha$
  \item $A(i, i) \leftarrow \alpha$
  \item normalizzo il vettore di Householder:
  	\begin{displaymath}\vect{\Delta} \leftarrow
  		\frac{\vect{\Delta}}{v_{1}}
	\end{displaymath}
  \item $\beta \leftarrow -\frac{v_{1}}{\alpha}$
  \item nella prossima equazione $\gamma \in \Gamma \in \mathbb{R}^{k \times
  k}$:
  \begin{displaymath}
	  \Gamma \leftarrow \Gamma - \beta %times
	  \begin{bmatrix}
	  	1 \\
	  	\vect{\Delta}^{*}
	  \end{bmatrix} %times
	   \begin{bmatrix}
	  	1 &
	  	(\vect{\Delta}^{*})^{T}
	  \end{bmatrix} \Gamma
  \end{displaymath}
\end{enumerate}
Tutti i vettori sono da intendersi vettori colonna e la relazione $\leftarrow$,
$(a, b) \in \leftarrow$, significa assegno $b$ ad $a$.
\end{oss}


\lstinputlisting{listings/chapterThree/QRmethod.m}

\subsection{functionExercise332}
\label{subsection:functionExercise332}
\lstinputlisting{listings/chapterThree/functionExercise332.m}