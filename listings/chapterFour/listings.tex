\section{Approssimazione di funzioni}

\subsection{Exercise 4.1 on textbook}
\label{subsec:exercise41}
\lstinputlisting{listings/chapterFour/exercise41.m} 

\subsection{Script for Exercise 4.1 on textbook}
\label{subsec:scriptForExercise41}
\lstinputlisting{listings/chapterFour/exercise41script.m}

\subsection{Exercise 4.6 on textbook - Differenze divise}
\label{subsec:differenzeDiviseEngineCode}
\lstinputlisting{listings/chapterFour/differenzeDiviseEngine.m}
Il costo di questo algoritmo \`e dato dalla seguente relazione:
\begin{displaymath}
\sum_{i = 1}^{n}{3(n-i+1)} = 3\sum_{k = 1}^{n}{k} = 3\frac{n(n+1)}{2} \in
O(n^{2})
\end{displaymath} 

\subsection{Exercise 4.7 on textbook - Horner generalizzato}
\label{subsec:HornerGeneralizzato}
\lstinputlisting{listings/chapterFour/HornerGeneralizzato.m}  
Il costo di questo algoritmo \`e $3n \in O(n)$.

\subsection{Exercise 4.7 on textbook - Function for testing}
\label{subsec:exercise47testing}
\lstinputlisting{listings/chapterFour/exercise47.m} 

\subsection{Exercise 4.8 on textbook - Differenze divise di Hermite}
\label{subsec:hermiteDifferenzeDiviseEngineCode}
\lstinputlisting{listings/chapterFour/hermiteDifferenzeDiviseEngine.m}

\subsection{Hermite engine}
\label{subsec:hermiteEngineCode}
\lstinputlisting{listings/chapterFour/Hermite.m}

\subsection{Exercise 4.9 on textbook}
\label{subsec:exercise49}
\lstinputlisting{listings/chapterFour/exercise49.m}

\subsection{Common Code For Exercises 4.11 and 4.15 on textbook}
\label{subsec:commonFactorForExercises411415}
\lstinputlisting{listings/chapterFour/exercises411415CommonFactor.m}

\subsection{Exercise 4.11 on textbook}
\label{subsec:exercise411}
\lstinputlisting{listings/chapterFour/exercise411.m}

\subsection{BuildChebyshevAscisse Maker}
\label{subsec:buildChebyshevAscisse}
\lstinputlisting{listings/chapterFour/buildChebyshevAscisse.m}

\subsection{Exercise 4.15 on textbook}
\label{subsec:exercise415}
\lstinputlisting{listings/chapterFour/exercise415.m}

\subsection{triangularSystemSolver}
\label{subsec:triangularSystemSolver}
\lstinputlisting{listings/chapterThree/triangularSystemSolver.m}

\subsection{tridiagonaleLUFactor}
\label{subsec:tridiagonaleLUFactor}
\lstinputlisting{listings/chapterFour/tridiagonaleLUFactor.m}

\subsection{hVarphiXiVectorsBuilder}
\label{subsec:hVarphiXiVectorsBuilder}
\lstinputlisting{listings/chapterFour/hVarphiXiVectorsBuilder.m}

\subsection{triangularBidiagonalMatrixBuilder}
\label{subsec:triangularBidiagonalMatrixBuilder}
\lstinputlisting{listings/chapterFour/triangularBidiagonalMatrixBuilder.m}

\subsection{cubicSplainEngine}
\label{subsec:cubicSplainEngine}
\lstinputlisting{listings/chapterFour/cubicSplainEngine.m}

\subsection{Spline stress}
\label{subsec:splineStress}
\lstinputlisting{listings/chapterFour/splineStress.m}

\subsection{Exercise 4.19 - Runge interpolation}
\label{subsec:exercise419RungeInterpolation}
\lstinputlisting{listings/chapterFour/exercise419runge.m}

\subsection{Exercise 4.19 - Runge interpolationPlotter}
\label{subsec:exercise419RungeInterpolationPlotter}
\lstinputlisting{listings/chapterFour/exercise419rungePlotter.m}

\subsection{Exercise 4.19 - Bernstein interpolation}
\label{subsec:exercise419BernsteinInterpolation}
\lstinputlisting{listings/chapterFour/exercise419bernstein.m}

\subsection{Exercise 4.19 - Bernstein interpolation Plotter}
\label{subsec:exercise419BernsteinInterpolationPlotter}
\lstinputlisting{listings/chapterFour/exercise419bernsteinPlotter.m}

\subsection{Exercise 4.19 on textbook}
\label{subsec:exercise419}
\lstinputlisting{listings/chapterFour/exercise419.m}

\subsection{Exercise 4.21 on textbook}
\label{subsec:exercise421}
\lstinputlisting{listings/chapterFour/exercise421solver.m}

\subsection{Exercise 4.22 on textbook}
\label{subsec:exercise422}
\`E interessante in questo esercizio la potenza di riportarci ad un modello
polinomiale, in quanto questo ci permette di costruire la matrice $A$ di tipo
\emph{VanderMonde} e la sua costruzione (molto semplice come si vede nel
codice) dipende solo dalle ascisse relative ai valori sperimentali. Questo ci
permette quindi da astrarre dalla forma (e dai segni) del modello polinomiale,
in quanto di questo ci interessa solo un eventuale cambio di variabile
eventualmente sulle ascisse (non il caso di questo esercizio) sia sulle ordinate
(come nel caso di questo esercizio):
\lstinputlisting{listings/chapterFour/exercise422solver.m}