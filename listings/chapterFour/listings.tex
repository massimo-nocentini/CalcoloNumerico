\section{Approssimazione di funzioni}

\subsection{Exercise 4.1 on textbook}
\label{subsec:exercise41}
\lstinputlisting{listings/chapterFour/exercise41.m} 

\subsection{Script for Exercise 4.1 on textbook}
\label{subsec:scriptForExercise41}
\lstinputlisting{listings/chapterFour/exercise41script.m}

\subsection{Exercise 4.6 on textbook - Differenze divise}
\label{subsec:differenzeDiviseEngineCode}
\lstinputlisting{listings/chapterFour/differenzeDiviseEngine.m}
Il costo di questo algoritmo \`e dato dalla seguente relazione:
\begin{displaymath}
\sum_{i = 1}^{n}{3(n-i+1)} = 3\sum_{k = 1}^{n}{k} = 3\frac{n(n+1)}{2} \in
O(n^{2})
\end{displaymath}

\subsection{Exercise 4.7 on textbook - Horner generalizzato}
\label{subsec:HornerGeneralizzato}
\lstinputlisting{listings/chapterFour/HornerGeneralizzato.m}  
Il costo di questo algoritmo \`e $3n \in O(n)$.

\subsection{Exercise 4.7 on textbook - Function for testing}
\label{subsec:exercise47testing}
\lstinputlisting{listings/chapterFour/exercise47.m} 