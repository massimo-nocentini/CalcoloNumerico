\section{Metodo di Newton}
\label{sec:metodoDiNewton}

\begin{exercise}
Implementare il metodo di newton ed applicarlo alla funzione \emph{singleZero},
con innesco iniziale $x_{0} = 7$, una tollerenza assoluta e relativa
$tol_{X} = rTol_{X} = 10^{-14}$ ed un numero massimo di iterazioni
$i_{max} = 10^{2}$.
\end{exercise}
Per l'implementazione del codice vedere \nameref{sec:newtonIterativeMethod}.
\begin{lstlisting}
octave:112> [x, i, ascisse] = newtonMethod('singleZero', 'singleZeroDerivative',7, 1e2, 1e-14, 1e-14) 
x =  3.40512483795333e+00
i =  5.00000000000000e+00
ascisse = [too long to report here]
octave:113> xSingleZero = min(ascisse)-1:0.1:max(ascisse)+1
octave:114> ySingleZero = invokeDelegate('singleZero', xSingleZero)
octave:115> [prepX, prepY] = prepareForPlottingMethodSegments(ascisse, 'invokeDelegate', 'singleZero')
octave:116> plot(xSingleZero, ySingleZero, "c", ascisse, invokeDelegate('singleZero', ascisse), "b+", prepX, prepY, "r")
octave:117> grid
octave:118> print 'newtonPlotOutput.tex' '-dTex' '-S800, 600'
\end{lstlisting}
Si raggiunge la tolleranza richiesta in 5 passi. Questo l'output del comando
\emph{octave:118}:
\begin{center}
\input{RadiciEquazione/newton/newtonPlotOutput.tex}
\end{center}

\begin{exercise}
Implementare il metodo di newton ed applicarlo alla funzione \emph{functionWithNoRealZero},
con innesco iniziale $x_{0} = 10$, una tollerenza assoluta e relativa
$tol_{X} = rTol_{X} = 10^{-14}$ ed un numero massimo di iterazioni
$i_{max} = 50$.
\end{exercise}
Per l'implementazione del codice vedere \nameref{sec:newtonIterativeMethod}.
\begin{lstlisting}
octave:112> [x, i, ascisse] = newtonMethod('functionWithNoRealZero','functionWithNoRealZeroDerivative', 10, 5e1, 1e-14, 1e-14)
Il metodo non converge.
x = -9.91800352556690e-01
i =  5.00000000000000e+01
ascisse = [too long to report here]
octave:113> xNoZero = min(ascisse)-1:0.1:max(ascisse)+1
octave:114> yNoZero = invokeDelegate('functionWithNoRealZero', xNoZero)
octave:115> [prepX, prepY] = prepareForPlottingMethodSegments(ascisse, 'invokeDelegate', 'functionWithNoRealZero')
octave:116> plot(xNoZero, yNoZero, "c", ascisse,invokeDelegate('functionWithNoRealZero', ascisse), "b+", prepX, prepY, "r")
octave:117> grid
octave:118> print 'newtonNoZeroPlotOutput.tex' '-dTex' '-S800, 600'
\end{lstlisting}
Non si raggiunge la convergenza, infatti vengono fatti il massimo possibile
dei passi fissati dal parametro $i_{max}$. Questo l'output del comando \emph{octave:118}:
\begin{center}
\input{RadiciEquazione/newton/newtonNoZeroPlotOutput.tex}
\end{center}