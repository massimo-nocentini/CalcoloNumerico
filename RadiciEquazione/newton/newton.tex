\section{Metodo di Newton}
\label{sec:metodoDiNewton}

\begin{exercise}
Implementare il metodo di newton ed applicarlo alla funzione \emph{singleZero},
con innesco iniziale $x_{0} = 7$, una tollerenza assoluta e relativa
$tol_{X} = rTol_{X} = 10^{-14}$ ed un numero massimo di iterazioni
$i_{max} = 10^{2}$.
\end{exercise}
Per l'implementazione del codice vedere \nameref{sec:newtonIterativeMethod}.
\begin{lstlisting}
octave:112> [x, i, ascisse] = newtonMethod('singleZero', 'singleZeroDerivative', 7, e^(2), e^(-14), e^(-14))
x =  3.40512483795333e+00
i =  4.00000000000000e+00
ascisse =
 Columns 1 through 3:
   7.00000000000000e+00   4.26666666666667e+00   3.48298368298368e+00
 Columns 4 through 6:
   3.40588582522216e+00   3.40512491208525e+00   3.40512483795333e+00
octave:113> xSingleZero = min(ascisse)-1:0.1:max(ascisse)+1
octave:114> ySingleZero = invokeDelegate('singleZero', xSingleZero)
octave:115> [prepX, prepY] = prepareForPlottingMethodSegments(ascisse, 'invokeDelegate', 'singleZero')
octave:116> plot(xSingleZero, ySingleZero, "c", ascisse, invokeDelegate('singleZero', ascisse), "b+", prepX, prepY, "r")
octave:117> grid
octave:118> print 'newtonPlotOutput.tex' '-dTex' '-S800, 600'
\end{lstlisting}
Si raggiunge la tolleranza richiesta in 4 passi. Questo l'output del comando
\emph{octave:118}:
\begin{center}
\input{RadiciEquazione/newton/newtonPlotOutput.tex}
\end{center}