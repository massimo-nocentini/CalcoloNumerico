\section{Metodo di Newton}
\label{sec:metodoDiNewton}

\begin{exercise}
Implementare il metodo di newton ed applicarlo alla funzione \emph{singleZero},
con innesco iniziale $x_{0} = 7$, una tollerenza assoluta e relativa
$tol_{X} = rTol_{X} = 10^{-14}$ ed un numero massimo di iterazioni
$i_{max} = 10^{2}$.
\end{exercise}
Per l'implementazione del codice vedere \nameref{sec:newtonIterativeMethod}.
\begin{lstlisting}
octave:112> [x, i, ascisse] = newtonMethod('singleZero','singleZeroDerivative',7, 1e2, 1e-14, 1e-14, 'incrementCriterion') 
x =  3.40512483795333e+00
i =  5.00000000000000e+00
ascisse = [too long to report here]
octave:113> xSingleZero = min(ascisse)-1:0.1:max(ascisse)+1
octave:114> ySingleZero = invokeDelegate('singleZero', xSingleZero)
octave:115> [prepX, prepY] = prepareForPlottingMethodSegments(ascisse, 'invokeDelegate', 'singleZero')
octave:116> plot(xSingleZero, ySingleZero, "c", ascisse, invokeDelegate('singleZero', ascisse), "b+", prepX, prepY, "r")
octave:117> grid
octave:118> print 'newtonPlotOutput.tex' '-dTex' '-S800, 600'
\end{lstlisting}
Si raggiunge la tolleranza richiesta in 5 passi. Questo l'output del comando
\emph{octave:118}:
\begin{center}
\input{RadiciEquazione/newton/newtonPlotOutput.tex}
\end{center}

\begin{oss}
Per l'applicazione eseguita sopra ho utilizzato il criterio di arresto
\emph{incremento}, il metodo converge, ma posso studiare il condizionamento che
affetta ogni valutazione della precisione richiesta. Questo il codice che
dimostra questo condizionamento, comparando l'applicazione del metodo usando il
criterio di arresto per \emph{residuo}:
\begin{lstlisting}
[x, i, incAscisse, incChecked] = newtonMethod('singleZero','singleZeroDerivative', 7, 1e2, 10^(-14), 10^(-14), 'incrementCriterion');
octave:138> i
i =  5.00000000000000e+00
octave:139> errors = errorMonitor(incAscisse)
errors =
   4.12195121951219e+00   9.88875669244497e+00   8.93522817393261e+01   8.95110152113953e+03   9.18666331414061e+07   7.66765947567831e+15
octave:140> incChecked 
incChecked =
   2.73333333333333e+00   7.83682983682984e-01   7.70978577615202e-02   7.60913136916397e-04   7.41319183816813e-08   8.88178419700125e-16
octave:141> [x, i, resAscisse, resChecked] = newtonMethod('singleZero','singleZeroDerivative', 7, 1e2, 10^(-14), 10^(-14), 'residueCriterion');
octave:142> resChecked 
resChecked =
   2.73333333333333e+00   7.83682983682984e-01   7.70978577615201e-02   7.60913136916414e-04   7.41319183470686e-08   6.82317562092805e-16
octave:143> i
i =  5.00000000000000e+00
\end{lstlisting}
Si osserva che i valori utilizzati nel controllo per il raggiungimento della
precisione richiesta sono uguali tranne l'ultimo, il metodo che usa il criterio
\emph{residueCriterion} \`e pi\`u accurato, in quanto non \`e affetto dalla
cancellazione numerica (per l'ultima coppia di ascisse trovata il fattore di
amplificazione \`e $k = 7.66765947567831e+15$). 

I due metodi utilizzano comunque
lo stesso numero di passi per convergere alla soluzione.
\end{oss}

\begin{oss}
Posso effettuare una nuova comparazione: se chiedo una precisione di $tol_{X}
= 1e-16$, si osserva che il metodo non converge se si utilizza il criterio di 
arresto per \emph{incremento}, mentre converge usando il criterio per
\emph{residuo}. Questo il codice che dimostra quanto detto:
\begin{lstlisting}
octave:5> [x, i]=newtonMethod('singleZero','singleZeroDerivative', 7, 1e2, 10^(-16), 10^(-16),'incrementCriterion');
Il metodo non converge.
octave:6> [x, i]=newtonMethod('singleZero','singleZeroDerivative', 7, 1e2, 10^(-16), 10^(-16),'residueCriterion');
octave:7> i
i =  6
octave:8>
\end{lstlisting}
A parit\`a di numero massimo di passi, il secondo criterio permette al metodo di
convergere.
\end{oss}


\begin{exercise}
Implementare il metodo di newton ed applicarlo alla funzione \emph{functionWithNoRealZero},
con innesco iniziale $x_{0} = 2.5$, una tollerenza assoluta e relativa
$tol_{X} = rTol_{X} = 10^{-14}$ ed un numero massimo di iterazioni
$i_{max} = 7$.
\end{exercise}
Per l'implementazione del codice vedere \nameref{sec:newtonIterativeMethod}.
\begin{lstlisting}
octave:112> [x, i, ascisse] =
newtonMethod('functionWithNoRealZero','functionWithNoRealZeroDerivative', 2.5, 7, 1e-14, 1e-14, 'incrementCriterion') 
Il metodo non converge.
x = -3.18786023393994e+00
i =  7.00000000000000e+00
ascisse = [too long to report here]
octave:113> xNoZero = min(ascisse)-1:0.1:max(ascisse)+1
octave:114> yNoZero = invokeDelegate('functionWithNoRealZero', xNoZero)
octave:115> [prepX, prepY] = prepareForPlottingMethodSegments(ascisse, 'invokeDelegate', 'functionWithNoRealZero')
octave:116> plot(xNoZero, yNoZero, "c", ascisse,invokeDelegate('functionWithNoRealZero', ascisse), "b+", prepX, prepY, "r")
octave:117> grid
octave:118> print 'newtonNoZeroPlotOutput.tex' '-dTex' '-S800, 600'
\end{lstlisting}
Non si raggiunge la convergenza, infatti vengono fatti il massimo possibile
dei passi fissati dal parametro $i_{max}$. Questo l'output del comando \emph{octave:118}:
\begin{center}
\input{RadiciEquazione/newton/newtonNoZeroPlotOutput.tex}
\end{center}

\begin{exercise}[2.4]
\label{exercise:newtonLoopStartingPoint}
Per il testo dell'esercizio consultare il libro di testo.
\end{exercise}
Per l'implementazione del codice vedere \nameref{sec:newtonIterativeMethod}.

Nel primo caso studio il punto di innesco $x_{0} = 10$:
\begin{lstlisting}
octave:112> [x, i, ascisse] =
newtonMethod('functionNewtonRecursion','functionNewtonRecursionDerivative', 10, 5e1, 1e-14, 1e-14, 'incrementCriterion') 
x =  2.23606797749979e+00
i =  9.00000000000000e+00
ascisse = [too long to report here]
octave:113> xNoZero = min(ascisse)-1:0.1:max(ascisse)+1
octave:114> yNoZero = invokeDelegate('functionNewtonRecursion', xNoZero)
octave:115> [prepX, prepY] = prepareForPlottingMethodSegments(ascisse, 'invokeDelegate', 'functionNewtonRecursion')
octave:116> plot(xNoZero, yNoZero, "c", ascisse,invokeDelegate('functionNewtonRecursion', ascisse), "b+", prepX, prepY, "r")
octave:117> grid
octave:118> print 'newtonRecursivePlotOutput.tex' '-dTex' '-S800, 600'
\end{lstlisting}
Si raggiunge la convergenza con 9 passi. Questo l'output del comando
\emph{octave:118}:
\begin{center}
\input{RadiciEquazione/newton/newtonRecursivePlotOutput.tex}
\end{center}

Nel secondo caso studio il punto di innesco $x_{0} = 1$:
\begin{lstlisting}
octave:112> [x, i, ascisse] =
newtonMethod('functionNewtonRecursion','functionNewtonRecursionDerivative', 1, 5e1, 1e-14, 1e-14, 'incrementCriterion') 
Il metodo non converge.
x = -1.00000000000000e+00
i =  5.00000000000000e+01
ascisse = [too long to report here]
octave:113> xNoZero = min(ascisse)-1:0.1:max(ascisse)+1
octave:114> yNoZero = invokeDelegate('functionNewtonRecursion', xNoZero)
octave:115> [prepX, prepY] = prepareForPlottingMethodSegments(ascisse, 'invokeDelegate', 'functionNewtonRecursion')
octave:116> plot(xNoZero, yNoZero, "c", ascisse,invokeDelegate('functionNewtonRecursion', ascisse), "b+", prepX, prepY, "r")
octave:117> grid
octave:118> print 'newtonRecursiveNotConvergencePlotOutput.tex' '-dTex' '-S800,600'
\end{lstlisting}
Il metodo non converge creando una finestra come si vede nel grafico. Questo
l'output del comando \emph{octave:118}:
\begin{center}
\input{RadiciEquazione/newton/newtonRecursiveNotConvergencePlotOutput.tex}
\end{center}