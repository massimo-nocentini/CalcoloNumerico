\section*{Metodi con rappresentazione grafica}
I prossimi metodi che verranno descritti utilizzeranno lo strumento Octave
per la rappresentazione grafica dei metodi per illustrare in modo pi\`u chiaro
il comportamento del metodo iterativo in questione.

Tutti i grafici hanno uno schema comune, ovvero sono composti da tre curve:
\begin{itemize}
  \item la curva cyan rappresenta la funzione $f(x)$ che si sta studiando
  \item la curva blu, composta da pochi simboli '$+$', rappresenta punti
  calcolati dal metodo per convergere allo zero $x^{*}$
  \item la curva rossa rappresenta il comportamento del metodo, visualizzando
  la sequenza con cui si procede per convergere alla soluzione
\end{itemize}

\section{Metodo di bisezione}
\label{sec:metodoDiBisezione}
Riporto il codice di pagina 23:
\begin{lstlisting}
octave:12> p = poly([1.1*ones(1,20) pi])
p =
 Columns 1 through 6:
   1.0000e+00  -2.5142e+01   2.9902e+02  -2.2396e+03   1.1860e+04  -4.7254e+04
 Columns 7 through 12:
   1.4711e+05  -3.6678e+05   7.4461e+05  -1.2444e+06   1.7234e+06  -1.9847e+06
 Columns 13 through 18:
   1.9008e+06  -1.5096e+06   9.8794e+05  -5.2718e+05   2.2572e+05  -7.5702e+04
 Columns 19 through 22:
   1.9159e+04  -3.4410e+03   3.9100e+02  -2.1135e+01
octave:13> polyval(p,pi)
ans =  2.0207e-04
\end{lstlisting}
Lo scopo della funzione $poly(r)$ \`e quello di creare un vettore di coefficienti
di un polinomio $p$, tale che le radici di $p$ appartengono al vettore $r$ 
($poly: Root[] \rightarrow PolynomialCoefficient[]$).
Valutando quindi il polinomio in una sua radice (\emph{octave:13}) in aritmetica
esatta dovrei ottenere 0, mentre in aritmetica finita non \`e vero 
($p(\pi) = ans =  2.0207e-04 \not = 0$).

\begin{exercise}
Implementare il metodo di bisezione ed applicarlo alla funzione $\sin(x)$ 
con intervallo iniziale $[2, 5]$ ed una tollerenza $tol_{X} = 10^{-14}$.
\end{exercise}
Per l'implementazione del codice vedere \nameref{sec:bisectionIterativeMethod}.
\begin{lstlisting}
octave:45> [x, i, imax, ascisse] = bisectionMethod('sin', 2, 5, 1e-14)
x =  3.14159265358980e+00
i =  4.60000000000000e+01
imax =  4.90000000000000e+01
ascisse = [too long to report here]
octave:46> length(ascisse)
ans =  4.60000000000000e+01
octave:47> xsin = min(ascisse):0.01:max(ascisse)
octave:48> ysin = feval('sin', xsin)
octave:49> [prepX, prepY] = prepareForPlottingMethodSegments(ascisse, "sin", "")
octave:50> plot(xsin, ysin, "c", ascisse, feval('sin', ascisse), "b+", prepX,prepY, "r") 
octave:51> grid
octave:52> print 'bisectionPlotOutput.tex' '-dTex' '-S800, 600'
\end{lstlisting}
Si raggiunge la tolleranza richiesta in 46 passi, tre in meno delle iterazioni
massime possibili. Questo l'output del comando \emph{octave:52}:
\begin{center}
\input{RadiciEquazione/bisection/bisectionPlotOutput.tex}
\end{center}

\begin{exercise}
Implementare il metodo di bisezione ed applicarlo alla funzione $\sin(x)$ 
con intervallo iniziale $[-0.1, 7]$, in modo da avere due zeri nell'
intervallo di confidenza, ed una tollerenza $tol_{X} = 10^{-14}$.
\end{exercise}
Per l'implementazione del codice vedere \nameref{sec:bisectionIterativeMethod}.
\begin{lstlisting}
octave:51> [x, i, imax, ascisse] = bisectionMethod('sin', -0.1, 7, 1e-14)
x =  6.28318530717958e+00
i =  4.70000000000000e+01
imax =  5.00000000000000e+01
ascisse = [too long to report here]
octave:52> length(ascisse)
ans =  4.70000000000000e+01
octave:53> xsin = min(ascisse):0.01:max(ascisse)
octave:54> ysin = feval('sin', xsin)
octave:55> [prepX, prepY] = prepareForPlottingMethodSegments(ascisse, "sin", "")
octave:56> plot(xsin, ysin, "c", ascisse, feval('sin', ascisse), "b+", prepX,prepY, "r") 
octave:57> grid
octave:58> print 'bisectionWithTwoRootsPlotOutput.tex' '-dTex' '-S800, 600'
\end{lstlisting}
Si raggiunge la tolleranza richiesta in 47 passi, tre in meno delle iterazioni
massime possibili. Questo l'output del comando \emph{octave:58}:
\begin{center}
\input{RadiciEquazione/bisection/bisectionWithTwoRootsPlotOutput.tex}
\end{center}

Da questo esercizio si vede che il metodo di bisezione converge comunque
ad una radice (in questa applicazione a $2\pi$), anche nel caso in cui
nell'intervallo di confidenza ci sono pi\`u zeri della funzione.





\section{Metodo di Newton}
\label{sec:metodoDiNewton}

\begin{exercise}
Implementare il metodo di newton ed applicarlo alla funzione \emph{singleZero},
con innesco iniziale $x_{0} = 7$, una tollerenza assoluta e relativa
$tol_{X} = rTol_{X} = 10^{-14}$ ed un numero massimo di iterazioni
$i_{max} = 10^{2}$.
\end{exercise}
Per l'implementazione del codice vedere \nameref{sec:newtonIterativeMethod}.
\begin{lstlisting}
octave:112> [x, i, ascisse] = newtonMethod('singleZero', 'singleZeroDerivative', 7, e^(2), e^(-14), e^(-14))
x =  3.40512483795333e+00
i =  4.00000000000000e+00
ascisse =
 Columns 1 through 3:
   7.00000000000000e+00   4.26666666666667e+00   3.48298368298368e+00
 Columns 4 through 6:
   3.40588582522216e+00   3.40512491208525e+00   3.40512483795333e+00
octave:113> xSingleZero = min(ascisse)-1:0.1:max(ascisse)+1
octave:114> ySingleZero = invokeDelegate('singleZero', xSingleZero)
octave:115> [prepX, prepY] = prepareForPlottingMethodSegments(ascisse, 'invokeDelegate', 'singleZero')
octave:116> plot(xSingleZero, ySingleZero, "c", ascisse, invokeDelegate('singleZero', ascisse), "b+", prepX, prepY, "r")
octave:117> grid
octave:118> print 'newtonPlotOutput.tex' '-dTex' '-S800, 600'
\end{lstlisting}
Si raggiunge la tolleranza richiesta in 4 passi. Questo l'output del comando
\emph{octave:118}:
\begin{center}
\input{RadiciEquazione/newton/newtonPlotOutput.tex}
\end{center}
\section{Varianti del metodo di Newton}
\label{sec:variantsMetodoDiNewton}

\begin{exercise}[2.5]
Per il testo dell'esercizio consultare il libro di testo.
\end{exercise}

Per l'implementazione del codice vedere \nameref{sec:newtonIterativeMethod},
\nameref{subsec:newtonMethodMultKnown},
\\\nameref{subsec:newtonMethodMultKnownLinearStopCriteria},
\nameref{subsec:newtonMethodAitken}.

I seguenti risultati sono stati generati invocando lo script
\nameref{subsec:ScriptEser25}:
\begin{lstlisting}
octave:221> scriptExercise25
\end{lstlisting}

Nella seguente tabella riporto l'applicazione dei metodi richiesti:
\begin{center}
\begin{tabular}{|c|c|c||c|c|}
 \multicolumn{1}{c}{} & 
 \multicolumn{2}{c}{function25first} &
 \multicolumn{2}{c}{function25second}
 \\
\cline{2-5}
	\multicolumn{5}{c}{\textbf{Newton standard}} \\
\hline
	$tol_{X}$ & steps & x & steps & x \\
\hline
	$0.01$ & $3.60000000000000e+01$ & $1.18248003631401e+00$ &
	$4.50000000000000e+01$ & $1.18529447118482e+00$ \\
		
	$0.0001$ & $8.00000000000000e+01$ & $1.00176964345428e+00$ & 
	$9.00000000000000e+01$ & $1.00165056388247e+00$ \\

	$1e-06$ & $1.24000000000000e+02$ & $1.00001716153733e+00$ &
	$1.33000000000000e+02$ & $1.00001778848790e+00$ \\

	$1e-08$ & $1.68000000000000e+02$ & $1.00000016642808e+00$ & 
	$1.77000000000000e+02$ & $1.00000017250842e+00$ \\ 

	$1e-10$ & $2.11000000000000e+02$ & $1.00000000179331e+00$ & 
	$2.21000000000000e+02$ & $1.00000000167294e+00$ \\

	$1e-12$ & $2.55000000000000e+02$ & $1.00000000001739e+00$ & 
	$2.65000000000000e+02$ & $1.00000000001622e+00$ \\

	$1e-14$ & $2.99000000000000e+02$ & $1.00000000000017e+00$ &
	$3.08000000000000e+02$ & $1.00000000000017e+00$ \\

	$1e-16$ & $3.48000000000000e+02$ & $1.00000000000000e+00$ & 
	$3.56000000000000e+02$ & $1.00000000000000e+00$ \\ 
\hline
	\multicolumn{5}{c}{\textbf{Newton modificato}} \\
\hline
	$tol_{X}$ & steps & x & steps & x \\
\hline
	$0.01$ & $1.00000000000000e+00$ & $1.00000000000000e+00$ &
	$4.00000000000000e+00$ & $1.00000041794715e+00$ \\
		
	$0.0001$ & $1.00000000000000e+00$ & $1.00000000000000e+00$ & 
	$5.00000000000000e+00$ & $1.00000000000002e+00$ \\

	$1e-06$ & $1.00000000000000e+00$ & $1.00000000000000e+00$ &
	$5.00000000000000e+00$ & $1.00000000000002e+00$ \\

	$1e-08$ & $1.00000000000000e+00$ & $1.00000000000000e+00$ & 
	$6.00000000000000e+00$ & $1.00000000000000e+00$ \\ 

	$1e-10$ & $1.00000000000000e+00$ & $1.00000000000000e+00$ &
	$6.00000000000000e+00$ & $1.00000000000000e+00$ \\ 

	$1e-12$ & $1.00000000000000e+00$ & $1.00000000000000e+00$ & 
	$6.00000000000000e+00$ & $1.00000000000000e+00$ \\ 

	$1e-14$ & $1.00000000000000e+00$ & $1.00000000000000e+00$ &
	$6.00000000000000e+00$ & $1.00000000000000e+00$ \\ 

	$1e-16$ & $1.00000000000000e+00$ & $1.00000000000000e+00$ &
	$7.00000000000000e+00$ & $1.00000000000000e+00$ \\ 
\hline
	\multicolumn{5}{c}{\textbf{Newton Aitken}} \\
\hline
	$tol_{X}$ & steps & x & steps & x \\
\hline
	$0.01$ & $2.00000000000000e+00$ & $1.00000000000000e+00$ &
	$6.00000000000000e+00$ & $9.99999983328898e-01$ \\
		
	$0.0001$ & $2.00000000000000e+00$ & $1.00000000000000e+00$ &
	$7.00000000000000e+00$ & $1.00000000000000e+00$ \\

	$1e-06$ & $2.00000000000000e+00$ & $1.00000000000000e+00$ &
	$7.00000000000000e+00$ & $1.00000000000000e+00$ \\

	$1e-08$ & $2.00000000000000e+00$ & $1.00000000000000e+00$ &
	$7.00000000000000e+00$ & $1.00000000000000e+00$ \\

	$1e-10$ & $2.00000000000000e+00$ & $1.00000000000000e+00$ &
	$8.00000000000000e+00$ & $1.00000000000000e+00$ \\

	$1e-12$ & $2.00000000000000e+00$ & $1.00000000000000e+00$ &
	$8.00000000000000e+00$ & $1.00000000000000e+00$ \\

	$1e-14$ & $3.00000000000000e+00$ & $1.00000000000000e+00$ &
	$8.00000000000000e+00$ & $1.00000000000000e+00$ \\

	$1e-16$ & $3.00000000000000e+00$ & $1.00000000000000e+00$ &
	$8.00000000000000e+00$ & $1.00000000000000e+00$ \\
\hline
\end{tabular}
\end{center}



\begin{exercise}[2.9]
Per il testo dell'esercizio consultare il libro di testo.
\end{exercise}
Tutti i codici riguardanti i metodi richiesti sono implementati per aumentare
la ``robustezza'' delle iterazioni (controllando che eventualmente i
denominatori delle varie funzioni siano non nulli, \ldots).

\section{Metodo di Newton}
\label{sec:metodoDiNewton}

\begin{exercise}
Implementare il metodo delle corde ed applicarlo alla funzione
\emph{chordConvergenceFunction}, con innesco iniziale $x_{0} = 5.3$, una tollerenza assoluta e relativa
$tol_{X} = rTol_{X} = 10^{-14}$ ed un numero massimo di iterazioni
$i_{max} = 10^{5}$.
\end{exercise}
Per l'implementazione del codice vedere \nameref{subsec:chordMethodLinearCriteria}.
\begin{lstlisting}
octave:112> [x, i, ascisse] =
chordMethodLinearCriteria('chordConvergenceFunction','chordConvergenceFunctionDerivative',5.3, 1e5, 1e-14, 1e-14) 
x =  1.78658833723778e+00
i =  9.40000000000000e+01
ascisse = [too long to report here]
octave:113> xSingleZero = min(ascisse)-1:0.1:max(ascisse)+1
octave:114> ySingleZero = invokeDelegate('chordConvergenceFunction', xSingleZero)
octave:115> [prepX, prepY] = prepareForPlottingMethodSegments(ascisse, 'invokeDelegate', 'chordConvergenceFunction')
octave:116> plot(xSingleZero, ySingleZero, "c", ascisse, invokeDelegate('chordConvergenceFunction', ascisse), "b+", prepX, prepY, "r")
octave:117> axis([1.3, 5.5, -5, 20])
octave:118> grid
octave:119> print 'chordPlotOutput.tex' '-dTex' '-S800, 600'
\end{lstlisting}
Si raggiunge la tolleranza richiesta in 94 passi. Questo l'output del comando
\emph{octave:119}:
\begin{center}
\input{RadiciEquazione/quasiNewton/chordPlotOutput.tex}
\end{center}


\begin{exercise}
Applicare il metodo delle corde in riferimentoa all'esercizio
\ref{exercise:newtonLoopStartingPoint} 
Usare una tollerenza assoluta e relativa
$tol_{X} = rTol_{X} = 10^{-14}$ ed un numero massimo di iterazioni
$i_{max} = 10^{5}$ e punto di innesco $x_{0} = 10$.
\end{exercise}
Per l'implementazione del codice vedere \nameref{subsec:chordMethodLinearCriteria}.
\begin{lstlisting}
octave:112> [x, i, ascisse] = chordMethodLinearCriteria('functionNewtonRecursion','functionNewtonRecursionDerivative',10,1e5, 1e-10, 1e-10)
x =  2.23606797759911e+00
i =  6.67000000000000e+02
ascisse = [too long to report here]
octave:113> xSingleZero = min(ascisse)-1:0.1:max(ascisse)+1
octave:114> ySingleZero = invokeDelegate('functionNewtonRecursion', xSingleZero)
octave:115> [prepX, prepY] = prepareForPlottingMethodSegments(ascisse, 'invokeDelegate', 'functionNewtonRecursion')
octave:116> plot(xSingleZero, ySingleZero, "c", ascisse, invokeDelegate('functionNewtonRecursion', ascisse), "b+", prepX, prepY, "r")
octave:117> axis([2, 5, -25, 100])
octave:118> grid
octave:119> print 'chordNewtonRecursionPlotOutput.tex' '-dTex' '-S800, 600'
\end{lstlisting}
Si raggiunge la tolleranza richiesta in 667 passi. Questo l'output del comando
\emph{octave:119}:
\begin{center}
\input{RadiciEquazione/quasiNewton/chordNewtonRecursionPlotOutput.tex}
\end{center}

Se invece voglio trovare la soluzione negativa:
\begin{lstlisting}
octave:112> [x, i, ascisse] = chordMethodLinearCriteria('functionNewtonRecursion','functionNewtonRecursionDerivative',-5,1e5, 1e-5, 1e-5)
i =  7.30000000000000e+01
ascisse = [too long to report here]
octave:113> xSingleZero = min(ascisse)-1:0.1:max(ascisse)+1
octave:114> ySingleZero = invokeDelegate('functionNewtonRecursion', xSingleZero)
octave:115> [prepX, prepY] = prepareForPlottingMethodSegments(ascisse, 'invokeDelegate', 'functionNewtonRecursion')
octave:116> plot(xSingleZero, ySingleZero, "c", ascisse, invokeDelegate('functionNewtonRecursion', ascisse), "b+", prepX, prepY, "r")
octave:117> axis([-4, -2, -10, 5])
octave:118> grid
octave:119> print 'chordNewtonRecursionNegativePlotOutput.tex' '-dTex' '-S800,600'
\end{lstlisting}
Si raggiunge la tolleranza richiesta in 73 passi. Questo l'output del comando
\emph{octave:119}:
\begin{center}
\input{RadiciEquazione/quasiNewton/chordNewtonRecursionNegativePlotOutput.tex}
\end{center}

\begin{exercise}
Implementare il metodo delle corde ed applicarlo alla funzione
\emph{secantConvergenceFunction}, con innesco iniziale $x_{0} = 10$, una
tollerenza assoluta e relativa $tol_{X} = rTol_{X} = 10^{-14}$ ed un numero massimo di iterazioni
$i_{max} = 10^{5}$.
\end{exercise}
Per l'implementazione del codice vedere \nameref{subsec:secantMethod}.
\begin{lstlisting}
octave:112> [x, i, ascisse] = secantMethod('secantConvergenceFunction','secantConvergenceFunctionDerivative',10,
1e5, 1e-14, 1e-14) 
x =  5.00000000000000e+00
i =  8.00000000000000e+00
ascisse = [too long to report here]
octave:113> xSingleZero = min(ascisse)-1:0.1:max(ascisse)+1
octave:114> ySingleZero = invokeDelegate('secantConvergenceFunction', xSingleZero)
octave:115> [prepX, prepY] = prepareForPlottingMethodSegments(ascisse, 'invokeDelegate', 'secantConvergenceFunction')
octave:116> plot(xSingleZero, ySingleZero, "c", ascisse, invokeDelegate('secantConvergenceFunction', ascisse), "b+", prepX, prepY, "r")
octave:118> grid
octave:119> print 'secantPlotOutput.tex' '-dTex' '-S800, 600'
\end{lstlisting}
Si raggiunge la tolleranza richiesta in 8 passi. Questo l'output del comando
\emph{octave:119}:
\begin{center}
\input{RadiciEquazione/quasiNewton/secantPlotOutput.tex}
\end{center}
\begin{oss}
Se provo ad eseguire il metodo di Newton sulla stessa funzione ottengo:
\begin{lstlisting}
octave:22> [x, i] = newtonMethod('secantConvergenceFunction','secantConvergenceFunctionDerivative',10,
> 1e5, 1e-14, 1e-14, 'residueCriterion')
x =  5.00000000000000e+00
i =  6.00000000000000e+00
\end{lstlisting}
Pi\`u veloce di tre passi.
\end{oss}

