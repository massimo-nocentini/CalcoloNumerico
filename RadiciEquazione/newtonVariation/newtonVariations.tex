\section{Varianti del metodo di Newton}
\label{sec:variantsMetodoDiNewton}

\begin{exercise}[2.5]
Per il testo dell'esercizio consultare il libro di testo.
\end{exercise}

Per l'implementazione del codice vedere \nameref{sec:newtonIterativeMethod},
\nameref{subsec:newtonMethodMultKnown},
\\\nameref{subsec:newtonMethodMultKnownLinearStopCriteria},
\nameref{subsec:newtonMethodAitken}.

I seguenti risultati sono stati generati invocando lo script
\nameref{subsec:ScriptEser25}:
\begin{lstlisting}
octave:221> scriptExercise25
\end{lstlisting}

Nella seguente tabella riporto l'applicazione dei metodi richiesti:
% i valori della seguente tabella sono stati presi dai dettagli inseriti nel
% file 'outputScriptExercise25.tex'
\begin{center}
\begin{tabular}{|c|c|c||c|c|}
\cline{2-5}
 \multicolumn{1}{c}{} &
 \multicolumn{2}{|c||}{function25first} &
 \multicolumn{2}{c|}{function25second}
 \\
\cline{2-5}
	\multicolumn{5}{c}{\textbf{Newton standard}} \\
\hline
	$tol_{X}$ & steps & x & steps & x \\
\hline
	$0.01$ & $3.60000000000000e+01$ & $1.18248003631401e+00$ &
	$4.50000000000000e+01$ & $1.18529447118482e+00$ \\
		
	$0.0001$ & $8.00000000000000e+01$ & $1.00176964345428e+00$ & 
	$9.00000000000000e+01$ & $1.00165056388247e+00$ \\

	$1e-06$ & $1.24000000000000e+02$ & $1.00001716153733e+00$ &
	$1.33000000000000e+02$ & $1.00001778848790e+00$ \\

	$1e-08$ & $1.68000000000000e+02$ & $1.00000016642808e+00$ & 
	$1.77000000000000e+02$ & $1.00000017250842e+00$ \\ 

	$1e-10$ & $2.11000000000000e+02$ & $1.00000000179331e+00$ & 
	$2.21000000000000e+02$ & $1.00000000167294e+00$ \\

	$1e-12$ & $2.55000000000000e+02$ & $1.00000000001739e+00$ & 
	$2.65000000000000e+02$ & $1.00000000001622e+00$ \\

	$1e-14$ & $2.99000000000000e+02$ & $1.00000000000017e+00$ &
	$3.08000000000000e+02$ & $1.00000000000017e+00$ \\

	$1e-16$ & $3.48000000000000e+02$ & $1.00000000000000e+00$ & 
	$3.56000000000000e+02$ & $1.00000000000000e+00$ \\ 
\hline
	\multicolumn{5}{c}{\textbf{Newton modificato}} \\
\hline
	$tol_{X}$ & steps & x & steps & x \\
\hline
	$0.01$ & $1.00000000000000e+00$ & $1.00000000000000e+00$ &
	$4.00000000000000e+00$ & $1.00000041794715e+00$ \\
		
	$0.0001$ & $1.00000000000000e+00$ & $1.00000000000000e+00$ & 
	$5.00000000000000e+00$ & $1.00000000000002e+00$ \\

	$1e-06$ & $1.00000000000000e+00$ & $1.00000000000000e+00$ &
	$5.00000000000000e+00$ & $1.00000000000002e+00$ \\

	$1e-08$ & $1.00000000000000e+00$ & $1.00000000000000e+00$ & 
	$6.00000000000000e+00$ & $1.00000000000000e+00$ \\ 

	$1e-10$ & $1.00000000000000e+00$ & $1.00000000000000e+00$ &
	$6.00000000000000e+00$ & $1.00000000000000e+00$ \\ 

	$1e-12$ & $1.00000000000000e+00$ & $1.00000000000000e+00$ & 
	$6.00000000000000e+00$ & $1.00000000000000e+00$ \\ 

	$1e-14$ & $1.00000000000000e+00$ & $1.00000000000000e+00$ &
	$6.00000000000000e+00$ & $1.00000000000000e+00$ \\ 

	$1e-16$ & $1.00000000000000e+00$ & $1.00000000000000e+00$ &
	$7.00000000000000e+00$ & $1.00000000000000e+00$ \\ 
\hline
	\multicolumn{5}{c}{\textbf{Newton Aitken}} \\
\hline
	$tol_{X}$ & steps & x & steps & x \\
\hline
	$0.01$ & $2.00000000000000e+00$ & $1.00000000000000e+00$ &
	$6.00000000000000e+00$ & $9.99999983328898e-01$ \\
		
	$0.0001$ & $2.00000000000000e+00$ & $1.00000000000000e+00$ &
	$7.00000000000000e+00$ & $1.00000000000000e+00$ \\

	$1e-06$ & $2.00000000000000e+00$ & $1.00000000000000e+00$ &
	$7.00000000000000e+00$ & $1.00000000000000e+00$ \\

	$1e-08$ & $2.00000000000000e+00$ & $1.00000000000000e+00$ &
	$7.00000000000000e+00$ & $1.00000000000000e+00$ \\

	$1e-10$ & $2.00000000000000e+00$ & $1.00000000000000e+00$ &
	$8.00000000000000e+00$ & $1.00000000000000e+00$ \\

	$1e-12$ & $2.00000000000000e+00$ & $1.00000000000000e+00$ &
	$8.00000000000000e+00$ & $1.00000000000000e+00$ \\

	$1e-14$ & $3.00000000000000e+00$ & $1.00000000000000e+00$ &
	$8.00000000000000e+00$ & $1.00000000000000e+00$ \\

	$1e-16$ & $3.00000000000000e+00$ & $1.00000000000000e+00$ &
	$8.00000000000000e+00$ & $1.00000000000000e+00$ \\
\hline
\end{tabular}
\end{center}

\newpage
% questa osservazione \`e necessario scriverla in una nuova pagina altrimenti
% non si riesce a visualizzare le due colonne.
\begin{oss}
Ho costruito lo script \nameref{subsec:ScriptEser25NewtonStopCriteriaComparison}
per effettuare un confronto tra i due metodi, nel momento in cui ho compilato la
precedente tabella non avevo ancora implementato il metodo di Newton con la
possibilit\`a di specificare il criterio di arresto, pertanto questa
osservazione \`e da ritenersi una aggiunta alla tabella precedente.
\begin{lstlisting}
octave:13> scriptExercise25NewtonComparison
\end{lstlisting}
Ottengo:	
\begin{multicols}{2}	
\begin{lstlisting}
application of the Newton method with increment stop criteria
application with tolx =0.01
x =  1.18248003631401e+00
i =  3.60000000000000e+01
application with tolx =0.0001
x =  1.00176964345428e+00
i =  8.00000000000000e+01
application with tolx =1e-06
x =  1.00001716153733e+00
i =  1.24000000000000e+02
application with tolx =1e-08
x =  1.00000016642808e+00
i =  1.68000000000000e+02
application with tolx =1e-10
x =  1.00000000179331e+00
i =  2.11000000000000e+02
application with tolx =1e-12
x =  1.00000000001739e+00
i =  2.55000000000000e+02
application with tolx =1e-14
x =  1.00000000000017e+00
i =  2.99000000000000e+02
application with tolx =1e-16
x =  1.00000000000000e+00
i =  3.48000000000000e+02

application of the second function
application with tolx =0.01
x =  1.18529447118482e+00
i =  4.50000000000000e+01
application with tolx =0.0001
x =  1.00165056388247e+00
i =  9.00000000000000e+01
application with tolx =1e-06
x =  1.00001778848790e+00
i =  1.33000000000000e+02
application with tolx =1e-08
x =  1.00000017250842e+00
i =  1.77000000000000e+02
application with tolx =1e-10
x =  1.00000000167294e+00
i =  2.21000000000000e+02
application with tolx =1e-12
x =  1.00000000001622e+00
i =  2.65000000000000e+02
application with tolx =1e-14
x =  1.00000000000017e+00
i =  3.08000000000000e+02
application with tolx =1e-16
x =  1.00000000000000e+00
i =  3.56000000000000e+02
\end{lstlisting}
\columnbreak
\begin{lstlisting}
application of the Newton method with residue stop criteria
application with tolx =0.01
x =  1.18248003631401e+00
i =  3.60000000000000e+01
application with tolx =0.0001
x =  1.00176964345428e+00
i =  8.00000000000000e+01
application with tolx =1e-06
x =  1.00001716153733e+00
i =  1.24000000000000e+02
application with tolx =1e-08
x =  1.00000016642808e+00
i =  1.68000000000000e+02
application with tolx =1e-10
x =  1.00000000179331e+00
i =  2.11000000000000e+02
application with tolx =1e-12
x =  1.00000000001739e+00
i =  2.55000000000000e+02
application with tolx =1e-14
x =  1.00000000000017e+00
i =  2.99000000000000e+02
application with tolx =1e-16
x =  1.00000000000000e+00
i =  3.43000000000000e+02

application of the second function
application with tolx =0.01
x =  1.18529447118482e+00
i =  4.50000000000000e+01
application with tolx =0.0001
x =  1.00165056388247e+00
i =  9.00000000000000e+01
application with tolx =1e-06
x =  1.00001778848790e+00
i =  1.33000000000000e+02
application with tolx =1e-08
x =  1.00000017250842e+00
i =  1.77000000000000e+02
application with tolx =1e-10
x =  1.00000000167294e+00
i =  2.21000000000000e+02
application with tolx =1e-12
x =  1.00000000001622e+00
i =  2.65000000000000e+02
application with tolx =1e-14
x =  1.00000000000017e+00
i =  3.08000000000000e+02
application with tolx =1e-16
x =  1.00000000000000e+00
i =  3.52000000000000e+02
\end{lstlisting}
\end{multicols}
L'esecuzione dello script permette di evidenziare che il metodo con criterio di
arresto per \emph{residuo} permette di convergere pi\`u velocemente alla
soluzione, infatti si guadagnano 5 passi per prima funzione e 4 per la seconda.
\end{oss}
