\section{Metodi iterativi}

\begin{exercise}[2.2]
	Per il testo dell'esercizio consultare il libro di testo.
\end{exercise}
Il metodo da trovare deve convergere a $x^{*} = \sqrt[m]{a}$ che, per l'esercizio
\ref{exercise:exerciseIterativeMethodFixedPoint}, deve verificare la condizione 
di consistenza:
\begin{displaymath}
	x^{*}=\phi(x^{*}), \quad \quad \phi(x^{*}) = x^{*} - 
		\frac{f(x^{*})}{f'(x^{*})}
\end{displaymath}
Implementando con $\phi$ il metodo di Newton che si chiede di usare.
Unendo le due uguaglianze sopra si ottiene:
\begin{equation}
\label{equation:equalsToZeroRequirement}
\begin{split}
	x^{*} &= x^{*} - \frac{f(x^{*})}{f'(x^{*})} \\
	0 &= f(x^{*})
\end{split}
\end{equation}
La (\ref{equation:equalsToZeroRequirement}) mi da un vincolo per la ricerca
della $f$ (questo vale per qualsiasi $f$!). Posso adesso sfruttare la richiesta
di convergenza:
\begin{equation}
\begin{split}
	x &= \sqrt[m]{a} \\
	f(x) = x^{m} - a &= 0
\end{split}
\end{equation}
Ho quindi costruito una funzione che rispetta il vincolo espresso nella
(\ref{equation:equalsToZeroRequirement}). Scrivo il metodo iterativo:
\begin{displaymath}
\begin{split}
	x_{i+1} &=\phi(x_{i}) \\
	\phi(x_{i}) &= x_{i} - 
		\frac{x_{i}^{m} - a}{m x_{i}^{m - 1}} = 
		\frac{m x_{i}^{m} - x_{i}^{m} + a}{m x_{i}^{m - 1}} = \\
	&= 	\frac{(m - 1) x_{i}^{m} + a}{m x_{i}^{m - 1}} = 
			\frac{m - 1}{m} x_{i} + \frac{a}{m x_{i}^{m - 1}}
\end{split}
\end{displaymath}
Verifico adesso che il metodo rispetti la condizione necessaria per la convergenza.
\begin{proof}
Per l'esercizio \ref{exercise:exerciseIterativeMethodFixedPoint} vale:
\begin{equation}
\begin{split}
	x &= \frac{m - 1}{m} x + \frac{a}{m x^{m - 1}} \\
	m x^{m} &= (m - 1) x^{m} + a \\
	x^{m} (m + 1 - m) &= a \\
	x^{m} &= a \\
\end{split}
\end{equation}
Il metodo $\phi$ soddisfa la condizione necessaria, converge al valore
richiesto e questo termina la prova.
\end{proof}

\begin{exercise}[2.3]
Per il testo dell'esercizio consultare il libro di testo.
\end{exercise}
Per la definizione del metodo delle secanti vale:
\begin{displaymath}
	x_{i + 1} = \frac{f(x_{i})x_{i-1} - f(x_{i-1})x_{i}}{f(x_{i}) - f(x_{i-1})}
\end{displaymath}
Come per l'esercizio precedente, devo verificare che il metodo soddisfi la
condizione necessaria di convergenza \footnote{qui mi \`e difficile far vedere
che vale la condizione necessaria, in quanto si annulla il denominatore}.
La verifica fornisce un vincolo per la funzione $f$ da ricercare: $f(x) = 0$, 
quindi agisco come nell'esercizio precedente, considerando la richiesta di
convergenza:
\begin{equation}
\begin{split}
	x &= \sqrt{a} \\
	f(x) = x^{2} - a &= 0
\end{split}
\end{equation}
Posso adesso sostituire nella definizione del metodo:
\begin{displaymath}
\begin{split}
	x_{i + 1} &= \frac{(x_{i}^{2} - a)x_{i-1} - (x_{i-1}^{2} - a)x_{i}}
		{x_{i}^{2} - a - (x_{i-1}^{2} - a)}
			  = \frac{x_{i}^{2} x_{i-1} - a x_{i-1} - 
						x_{i-1}^{2} x_{i} + a x_{i}}
					{x_{i}^{2} - x_{i-1}^{2}} = \\
			  &= \frac{a(x_{i} - x_{i-1}) + x_{i}x_{i-1}(x_{i} - x_{i-1})}
					{(x_{i} - x_{i-1})(x_{i} + x_{i-1})} 
			  = \frac{(x_{i} - x_{i-1})(a + x_{i}x_{i-1})}
					{(x_{i} - x_{i-1})(x_{i} + x_{i-1})} 
			  = \frac{a + x_{i}x_{i-1}}{x_{i} + x_{i-1}}
\end{split}
\end{displaymath}
Verifico che il metodo caratterizzato soddisfi la condizione necessaria di
convergenza.
\begin{proof}
Per l'esercizio \ref{exercise:exerciseIterativeMethodFixedPoint} vale:
\begin{displaymath}
\begin{split}
\lim_{i\rightarrow\infty}{x_{i}} = x^{*} & = \sqrt{a} \quad\Rightarrow \quad
x_{i+1} = x_{i} = x_{i-1} = x, \quad \text{segue che} \\
x &= \frac{a + x^{2}}{2x} \\ 
2x^{2} &= a + x^{2} \\
x^{2} &= a
\end{split}
\end{displaymath}
E questo termina la prova.
\end{proof}
Comparando con l'\emph{Esercizio 1.4} del libro di testo si vede che il metodo
proposto in tale esercizio \`e la caratterizzazione del metodo delle secanti 
qui discusso con la caratterizzazione di $a = 2$:
\begin{displaymath}
x_{i + 1} = \frac{2 + x_{i}x_{i-1}}{x_{i} + x_{i-1}}
\end{displaymath}

\begin{oss}[Sul teorema di punto fisso]
Se parto da due punti nell'intervallo $x, y \in I = (x^{*} - \delta, x^{*} +
\delta)$, allora applicando il metodo iterativo $\phi$ ai due punti ottengo una
nuova coppia $\phi(x) = x_{1}, \phi(y) = y_{2}$, la cui distanza $|x_{1} - y_{1}|$
\`e strettamente minore della distanza dei punti di partenza in quanto $L < 1$
per ipotesi del teorema: ottengo quindi $|x_{1} - y_{1}| < |x - y|$. In questo
modo sono riuscito ad ottenere una nuova coppia di punti pi\`u vicina.

Questo \`e quello che avrei voluto in quanto una volta applicato il metodo ad
una coppia, ottengo una coppia di punti pi\`u vicina: se riesco ad applicare
nuovamente il metodo alla coppia appena costruita riuscirei ad ottenere una
nuova coppia di punti pi\`u vicini rispetto alla ultima coppia generata.
Ripetendo un numero sufficiente di volte il metodo $\phi$ riesco a diminuire la
distanza tra i punti delle coppie, fino a far degenerare una coppia di punti
distinti allo stesso punto $(x_{k}, x_{k})$, ovvero al punto fisso del metodo,
e quindi convergere alla soluzione.

Importante quindi \`e partire da una coppia di punti che appartengono
all'intervallo $I$ per poter applicare almeno una volta il metodo.

Posso commentare le implicazioni del teorema:
\begin{enumerate}
  \item $x^{*}$ \`e l'unico punto fisso di $\phi$ in $I$: questo permette al
  metodo di non oscillare tra due punti fissi (soluzioni) e inoltre di non
  creare non determinismo nella scelta della soluzione a cui converge il metodo
  \item se $x_{0} \in I \Rightarrow \phi(x_{i-1}) = x_{i} \in I, \forall i =
  \{1, 2, 3, \ldots\}$: questo \`e molto importante perch\`e assicura che ogni
  punto generato dal metodo $\phi$ appartiene all'intervallo se il punto di
  innesco $x_{0} \in I$. In questo modo, tornando a ragionare per coppie come
  fatto nei paragrafi precedenti di questa osservazione, se il punto di innesco
  appartiene all'intervallo $I$ allora tutti i nuovi punti che posso costruire
  con il metodo $\phi$ apparterranno anch'essi all'intervallo e quindi posso 
  ragionare a coppie $(x_{i-1}, x_{i})$ e applicare il teorema ad ogni coppia
  per $i \in \{ 1, 2,\ldots \}$, ottenendo quindi che i punti della coppia
  $(x_{i}, x_{i+1})$ saranno pi\`u vicini dei punti della coppia $(x_{i-1},
  x_{i})$.
  \begin{proof}
  	Per induzione su $i$: 
  	\begin{description}
  	  \item[Base] se $x_{0} \not \in I$ allora la tesi \`e vera (vacously true)
  	  \item[Hp induttiva] suppungo vero che $x_{k} = \phi(x_{k-1})\in I, \forall
  	  k \in \{1, \ldots i\}$
  	  \item[Passo induttivo] dimostro per $k = i + 1$:
  	  \begin{displaymath}
  	  	|x_{k+1} - x^{*} | = |\phi(x_{k}) - \phi(x^{*} ) | \leq l |x_{k} - x^{*} | 
  	  \end{displaymath}
  	  Per ipotesi induttiva $x_{k}, x^{*} \in I$ questo implica che $l |x_{k} -
  	  x^{*}| < \delta$. Per la propriet\`a transitiva della relazione $\leq$ vale
  	  $|x_{k+1} - x^{*} | < \delta$, ovvero $x_{k+1} \in I$ e questo termina il
  	  passo induttivo.
  	\end{description}
  \end{proof}
  
  \item $\lim_{i \rightarrow \infty}{x_{i}} = x^{*}$, ovvero il metodo converge
  alla soluzione $x^{*}$ punto fisso di $\phi$ per la condizione necessaria.
  
  Nella prova riportata nel libro la regoletta per impostare l'indice del
  termine $l$ \`e questa: 
  \begin{displaymath}
  	l^{k}|x_{j } - x^{*}|, j = i - a \quad \Rightarrow \quad k = i - j
  \end{displaymath}
\end{enumerate}
\end{oss}

\begin{exercise}
 Dire se il seguente metodo iterativo \`e convergente e se si a quale punto
 converge.
 \begin{displaymath}
 	x_{k+1} = \sqrt{x_{k} + 1}
 \end{displaymath}
\end{exercise}
Formalizzo il metodo con la funzione $\phi$:
\begin{displaymath}
 	\phi(x) = \sqrt{x + 1}
 \end{displaymath}
Se il metodo converge, per la condizione necessaria, converge al suo punto
fisso, quindi lo determino:
\begin{displaymath}
 \begin{split}
 	x &= \phi(x) = \sqrt{x + 1} \\
 	x^{2} &= x + 1 	
 \end{split}
\end{displaymath}
Implica che le due soluzioni sono $x_{1,2} = \frac{1 \pm \sqrt{5}}{2}$.

Non ho dimostrato che il metodo converge, ho solo trovato il punto di
convergenza se il metodo converge. Dimostro adesso la convergenza controllando
che siano vere le ipotesi del teorema di punto fisso:
\begin{displaymath}
 \begin{split}
 	\exists \delta > 0 \exists l, 0 \leq l < 1 &: \forall x, y \in I = (x^{*} -
 	\delta, x^{*} +	\delta): \\
 	|\phi(x) - \phi(y)| &< l|x - y|
 \end{split}
\end{displaymath}
Dato che $\phi$ \`e derivabile allora sviluppo $\phi(y)$ in $x$ con resto al
primo ordine, per applicare il metodo due volte nello stesso punto:
\begin{displaymath}
 \begin{split}
 	T_{\phi(y)}(x) = \phi(x) + \phi'(\xi)(y - x), \quad \xi \in [x, y]
 	(\Rightarrow \xi \in I)
 \end{split}
\end{displaymath}
Posso sostituire nella precedente disuguaglianza:
\begin{displaymath}
 \begin{split}
 	|\phi(x) - \phi(x) - \phi'(\xi)(y - x)| &< l|x - y| \\
 	|\phi'(\xi)(x - y)| &< l|x - y| \\
 	|\phi'(\xi)||x - y| &< l|x - y| \\
 	|\phi'(\xi)| &< l < 1
 \end{split}
\end{displaymath}
Devo quindi verificare:
\begin{displaymath}
 \begin{split}
 	|\phi'(\xi)| &< 1 \\
 	\phi'(x) &= \frac{1}{2\sqrt{x + 1}} \\
 	\frac{1}{2\sqrt{x + 1}} &< 1 \\
 	1 &< 2\sqrt{x + 1}, \forall x \geq 0
 \end{split}
\end{displaymath}
Sono nelle ipotesi del teorema di punto fisso \footnote{l'unica cosa che rimane
da trovare \`e $l$} quindi per l'implicazione del teorema valgono
\begin{itemize}
  \item $x = \frac{1 \pm \sqrt{5}}{2}$ \`e l'unico punto fisso di $\phi$ in $I$
  \item tutti i punti $x_{i}$ generati dal metodo appartengono all'intervallo
  $I$
  \item il metodo converge a $x = \frac{1 \pm \sqrt{5}}{2}$.
\end{itemize}
