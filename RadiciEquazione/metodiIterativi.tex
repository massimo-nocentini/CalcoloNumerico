\section{Metodi iterativi}

\begin{exercise}[2.2]
	Per il testo dell'esercizio consultare il libro di testo.
\end{exercise}
Il metodo da trovare deve convergere a $x^{*} = \sqrt[m]{a}$ che, per l'esercizio
\ref{exercise:exerciseIterativeMethodFixedPoint}, deve verificare la condizione 
di consistenza:
\begin{displaymath}
	x^{*}=\phi(x^{*}), \quad \quad \phi(x^{*}) = x^{*} - 
		\frac{f(x^{*})}{f'(x^{*})}
\end{displaymath}
Implementando con $\phi$ il metodo di Newton che si chiede di usare.
Unendo le due uguaglianze sopra si ottiene:
\begin{equation}
\label{equation:equalsToZeroRequirement}
\begin{split}
	x^{*} &= x^{*} - \frac{f(x^{*})}{f'(x^{*})} \\
	0 &= f(x^{*})
\end{split}
\end{equation}
La (\ref{equation:equalsToZeroRequirement}) mi da un vincolo per la ricerca
della $f$. Posso adesso sfruttare la richiesta di convergenza:
\begin{equation}
\begin{split}
	x &= \sqrt[m]{a} \\
	f(x) = x^{m} - a &= 0
\end{split}
\end{equation}
Ho quindi costruito una funzione che rispetta il vincolo espresso nella
(\ref{equation:equalsToZeroRequirement}). Scrivo il metodo iterativo:
\begin{displaymath}
\begin{split}
	x_{i+1} &=\phi(x_{i}) \\
	\phi(x_{i}) &= x_{i} - 
		\frac{x_{i}^{m} - a}{m x_{i}^{m - 1}} = 
		\frac{m x_{i}^{m} - x_{i}^{m} + a}{m x_{i}^{m - 1}} = \\
	&= 	\frac{(m - 1) x_{i}^{m} + a}{m x_{i}^{m - 1}} = 
			\frac{m - 1}{m} x_{i} + \frac{a}{m x_{i}^{m - 1}}
\end{split}
\end{displaymath}
Verifico adesso che il metodo rispetti la condizione necessaria per la convergenza.
\begin{proof}
Per l'esercizio \ref{exercise:exerciseIterativeMethodFixedPoint} vale:
\begin{equation}
\begin{split}
	x &= \frac{m - 1}{m} x + \frac{a}{m x^{m - 1}} \\
	m x^{m} &= (m - 1) x^{m} + a \\
	x^{m} (m + 1 - m) &= a \\
	x^{m} &= a \\
\end{split}
\end{equation}
Il metodo $\phi$ soddisfa la condizione necessaria, converge al valore
richiesto e questo termina la prova.
\end{proof}

\begin{exercise}[2.3]
Per il testo dell'esercizio consultare il libro di testo.
\end{exercise}
Per la definizione del metodo delle secanti vale:
\begin{displaymath}
	x_{i + 1} = \frac{f(x_{i})x_{i-1} - f(x_{i-1})x_{i}}{f(x_{i}) - f(x_{i-1})}
\end{displaymath}
Come per l'esercizio precedente, devo verificare che il metodo soddisfi la
condizione necessaria di convergenza \footnote{qui mi \`e difficile far vedere
che vale la condizione necessaria, in quanto si annulla il denominatore}.
La verifica fornisce un vincolo per la funzione $f$ da ricercare: $f(x) = 0$, 
quindi agisco come nell'esercizio precedente, considerando la richiesta di
convergenza:
\begin{equation}
\begin{split}
	x &= \sqrt{a} \\
	f(x) = x^{2} - a &= 0
\end{split}
\end{equation}
Posso adesso sostituire nella definizione del metodo:
\begin{displaymath}
\begin{split}
	x_{i + 1} &= \frac{(x_{i}^{2} - a)x_{i-1} - (x_{i-1}^{2} - a)x_{i}}
		{x_{i}^{2} - a - (x_{i-1}^{2} - a)}
			  = \frac{x_{i}^{2} x_{i-1} - a x_{i-1} - 
						x_{i-1}^{2} x_{i} + a x_{i}}
					{x_{i}^{2} - x_{i-1}^{2}} = \\
			  &= \frac{a(x_{i} - x_{i-1}) + x_{i}x_{i-1}(x_{i} - x_{i-1})}
					{(x_{i} - x_{i-1})(x_{i} + x_{i-1})} 
			  = \frac{(x_{i} - x_{i-1})(a + x_{i}x_{i-1})}
					{(x_{i} - x_{i-1})(x_{i} + x_{i-1})} 
			  = \frac{a + x_{i}x_{i-1}}{x_{i} + x_{i-1}}
\end{split}
\end{displaymath}
Verifico che il metodo caratterizzato soddisfi la condizione necessaria di
convergenza.
\begin{proof}
Per l'esercizio \ref{exercise:exerciseIterativeMethodFixedPoint} vale:
\begin{displaymath}
\begin{split}
\lim_{i\rightarrow\infty}{x_{i}} = x^{*} & = \sqrt{a} \quad\Rightarrow \quad
x_{i+1} = x_{i} = x_{i-1} = x, \quad \text{segue che} \\
x &= \frac{a + x^{2}}{2x} \\ 
2x^{2} &= a + x^{2} \\
x^{2} &= a
\end{split}
\end{displaymath}
E questo termina la prova.
\end{proof}
Comparando con l'\emph{Esercizio 1.4} del libro di testo si vede che il metodo
proposto in tale esercizio \`e la caratterizzazione del metodo delle secanti 
qui discusso con la caratterizzazione di $a = 2$:
\begin{displaymath}
x_{i + 1} = \frac{2 + x_{i}x_{i-1}}{x_{i} + x_{i-1}}
\end{displaymath}
