\section{Metodo di bisezione}
\label{sec:metodoDiBisezione}
Riporto il codice di pagina 23:
\begin{lstlisting}
octave:12> p = poly([1.1*ones(1,20) pi])
p =
 Columns 1 through 6:
   1.0000e+00  -2.5142e+01   2.9902e+02  -2.2396e+03   1.1860e+04  -4.7254e+04
 Columns 7 through 12:
   1.4711e+05  -3.6678e+05   7.4461e+05  -1.2444e+06   1.7234e+06  -1.9847e+06
 Columns 13 through 18:
   1.9008e+06  -1.5096e+06   9.8794e+05  -5.2718e+05   2.2572e+05  -7.5702e+04
 Columns 19 through 22:
   1.9159e+04  -3.4410e+03   3.9100e+02  -2.1135e+01
octave:13> polyval(p,pi)
ans =  2.0207e-04
\end{lstlisting}
Lo scopo della funzione $poly(r)$ \`e quello di creare un vettore di coefficienti
di un polinomio $p$, tale che le radici di $p$ appartengono al vettore $r$ 
($poly: Root[] \rightarrow PolynomialCoefficient[]$).
Valutando quindi il polinomio in una sua radice (\emph{octave:13}) in aritmetica
esatta dovrei ottenere 0, mentre in aritmetica finita non \`e vero 
($p(\pi) = ans =  2.0207e-04 \not = 0$).

\begin{exercise}
Implementare il metodo di bisezione ed applicarlo alla funzione $\sin(x)$ 
con intervallo iniziale $[2, 5]$ ed una tollerenza $tol_{X} = 10^{-14}$.
\end{exercise}
Per l'implementazione del codice vedere \nameref{sec:bisectionIterativeMethod}.
\begin{lstlisting}
octave:45> [x, i, imax, ascisse] = bisectionMethod('sin', 2, 5, 1e-14)
x =  3.14159265358980e+00
i =  4.60000000000000e+01
imax =  4.90000000000000e+01
ascisse = [too long to report here]
octave:46> length(ascisse)
ans =  4.60000000000000e+01
octave:47> xsin = min(ascisse):0.01:max(ascisse)
octave:48> ysin = feval('sin', xsin)
octave:49> [prepX, prepY] = prepareForPlottingMethodSegments(ascisse, "sin", "")
octave:50> plot(xsin, ysin, "c", ascisse, feval('sin', ascisse), "b+", prepX,prepY, "r") 
octave:51> grid
octave:52> print 'bisectionPlotOutput.tex' '-dTex' '-S800, 600'
\end{lstlisting}
Si raggiunge la tolleranza richiesta in 46 passi, tre in meno delle iterazioni
massime possibili. Questo l'output del comando \emph{octave:52}:
\begin{center}
\input{RadiciEquazione/bisection/bisectionPlotOutput.tex}
\end{center}

\begin{exercise}
Implementare il metodo di bisezione ed applicarlo alla funzione $\sin(x)$ 
con intervallo iniziale $[-0.1, 7]$, in modo da avere due zeri nell'
intervallo di confidenza, ed una tollerenza $tol_{X} = 10^{-14}$.
\end{exercise}
Per l'implementazione del codice vedere \nameref{sec:bisectionIterativeMethod}.
\begin{lstlisting}
octave:51> [x, i, imax, ascisse] = bisectionMethod('sin', -0.1, 7, 1e-14)
x =  6.28318530717958e+00
i =  4.70000000000000e+01
imax =  5.00000000000000e+01
ascisse = [too long to report here]
octave:52> length(ascisse)
ans =  4.70000000000000e+01
octave:53> xsin = min(ascisse):0.01:max(ascisse)
octave:54> ysin = feval('sin', xsin)
octave:55> [prepX, prepY] = prepareForPlottingMethodSegments(ascisse, "sin", "")
octave:56> plot(xsin, ysin, "c", ascisse, feval('sin', ascisse), "b+", prepX,prepY, "r") 
octave:57> grid
octave:58> print 'bisectionWithTwoRootsPlotOutput.tex' '-dTex' '-S800, 600'
\end{lstlisting}
Si raggiunge la tolleranza richiesta in 47 passi, tre in meno delle iterazioni
massime possibili. Questo l'output del comando \emph{octave:58}:
\begin{center}
\input{RadiciEquazione/bisection/bisectionWithTwoRootsPlotOutput.tex}
\end{center}

Da questo esercizio si vede che il metodo di bisezione converge comunque
ad una radice (in questa applicazione a $2\pi$), anche nel caso in cui
nell'intervallo di confidenza ci sono pi\`u zeri della funzione.




