\section{Metodo quasi Newton}
\label{sec:metodiQuasiNewton}

\begin{exercise}
Implementare il metodo delle corde ed applicarlo alla funzione
\emph{chordConvergenceFunction}, con innesco iniziale $x_{0} = 5.3$, una tollerenza assoluta e relativa
$tol_{X} = rTol_{X} = 10^{-14}$ ed un numero massimo di iterazioni
$i_{max} = 10^{5}$.
\end{exercise}
Per l'implementazione del codice vedere \nameref{subsec:chordMethodLinearCriteria}.
\begin{lstlisting}
octave:112> [x, i, ascisse] =
chordMethodLinearCriteria('chordConvergenceFunction','chordConvergenceFunctionDerivative',5.3, 1e5, 1e-14, 1e-14) 
x =  1.78658833723778e+00
i =  9.40000000000000e+01
ascisse = [too long to report here]
octave:113> xSingleZero = min(ascisse)-1:0.1:max(ascisse)+1
octave:114> ySingleZero = invokeDelegate('chordConvergenceFunction', xSingleZero)
octave:115> [prepX, prepY] = prepareForPlottingMethodSegments(ascisse, 'invokeDelegate', 'chordConvergenceFunction')
octave:116> plot(xSingleZero, ySingleZero, "c", ascisse, invokeDelegate('chordConvergenceFunction', ascisse), "b+", prepX, prepY, "r")
octave:117> axis([1.3, 5.5, -5, 20])
octave:118> grid
octave:119> print 'chordPlotOutput.tex' '-dTex' '-S800, 600'
\end{lstlisting}
Si raggiunge la tolleranza richiesta in 94 passi. Questo l'output del comando
\emph{octave:119}:
\begin{center}
\input{RadiciEquazione/quasiNewton/chordPlotOutput.tex}
\end{center}


\begin{exercise}
Applicare il metodo delle corde in riferimentoa all'esercizio
\ref{exercise:newtonLoopStartingPoint} 
Usare una tollerenza assoluta e relativa
$tol_{X} = rTol_{X} = 10^{-14}$ ed un numero massimo di iterazioni
$i_{max} = 10^{5}$ e punto di innesco $x_{0} = 10$.
\end{exercise}
Per l'implementazione del codice vedere \nameref{subsec:chordMethodLinearCriteria}.
\begin{lstlisting}
octave:112> [x, i, ascisse] = chordMethodLinearCriteria('functionNewtonRecursion','functionNewtonRecursionDerivative',10,1e5, 1e-10, 1e-10)
x =  2.23606797759911e+00
i =  6.67000000000000e+02
ascisse = [too long to report here]
octave:113> xSingleZero = min(ascisse)-1:0.1:max(ascisse)+1
octave:114> ySingleZero = invokeDelegate('functionNewtonRecursion', xSingleZero)
octave:115> [prepX, prepY] = prepareForPlottingMethodSegments(ascisse, 'invokeDelegate', 'functionNewtonRecursion')
octave:116> plot(xSingleZero, ySingleZero, "c", ascisse, invokeDelegate('functionNewtonRecursion', ascisse), "b+", prepX, prepY, "r")
octave:117> axis([2, 5, -25, 100])
octave:118> grid
octave:119> print 'chordNewtonRecursionPlotOutput.tex' '-dTex' '-S800, 600'
\end{lstlisting}
Si raggiunge la tolleranza richiesta in 667 passi. Questo l'output del comando
\emph{octave:119}:
\begin{center}
\input{RadiciEquazione/quasiNewton/chordNewtonRecursionPlotOutput.tex}
\end{center}

Se invece voglio trovare la soluzione negativa:
\begin{lstlisting}
octave:112> [x, i, ascisse] = chordMethodLinearCriteria('functionNewtonRecursion','functionNewtonRecursionDerivative',-5,1e5, 1e-5, 1e-5)
i =  7.30000000000000e+01
ascisse = [too long to report here]
octave:113> xSingleZero = min(ascisse)-1:0.1:max(ascisse)+1
octave:114> ySingleZero = invokeDelegate('functionNewtonRecursion', xSingleZero)
octave:115> [prepX, prepY] = prepareForPlottingMethodSegments(ascisse, 'invokeDelegate', 'functionNewtonRecursion')
octave:116> plot(xSingleZero, ySingleZero, "c", ascisse, invokeDelegate('functionNewtonRecursion', ascisse), "b+", prepX, prepY, "r")
octave:117> axis([-4, -2, -10, 5])
octave:118> grid
octave:119> print 'chordNewtonRecursionNegativePlotOutput.tex' '-dTex' '-S800,600'
\end{lstlisting}
Si raggiunge la tolleranza richiesta in 73 passi. Questo l'output del comando
\emph{octave:119}:
\begin{center}
\input{RadiciEquazione/quasiNewton/chordNewtonRecursionNegativePlotOutput.tex}
\end{center}
\begin{oss}
Il metodo delle secanti converge pi\`u rapidamente quando la derivata non \`e
molto ripida: in questo esempio accade il contrario, ovvero la derivata \`e
molto ripida, il metodo converge lentamente ma si guadagna in precisione per
l'approssimazione fornita.
\end{oss}

\begin{exercise}
Implementare il metodo delle corde ed applicarlo alla funzione
\emph{secantConvergenceFunction}, con innesco iniziale $x_{0} = 6$, una
tollerenza assoluta e relativa $tol_{X} = rTol_{X} = 10^{-14}$ ed un numero massimo di iterazioni
$i_{max} = 10^{5}$.
\end{exercise}
Per l'implementazione del codice vedere \nameref{subsec:secantMethod}.
\begin{lstlisting}
octave:45> [x, i, ascisse] = secantMethod('secantConvergenceFunction','secantConvergenceFunctionDerivative',6,
1e5, 1e-14, 1e-14)
x =  1.00000000000000e+00
i =  1.10000000000000e+01
ascisse = [too long to report here]
octave:113> xSingleZero = min(ascisse)-1:0.1:max(ascisse)+1
octave:114> ySingleZero = invokeDelegate('secantConvergenceFunction', xSingleZero)
octave:115> [prepX, prepY] = prepareForPlottingSecantMethodSegments(ascisse, 'invokeDelegate', 'secantConvergenceFunction')
octave:116> plot(xSingleZero, ySingleZero, "c", ascisse, invokeDelegate('secantConvergenceFunction', ascisse), "b+", prepX, prepY, "r")
octave:118> grid
octave:119> axis([0.7, 6.3, -2, 30])
octave:120> print 'secantPlotOutput.tex' '-dTex' '-S800, 600'
\end{lstlisting}
Si raggiunge la tolleranza richiesta in 11 passi. Questo l'output del comando
\emph{octave:120}:
\begin{center}
\input{RadiciEquazione/quasiNewton/secantPlotOutput.tex}
\end{center}
\begin{oss}
Se provo ad eseguire il metodo di Newton sulla stessa funzione ottengo:
\begin{lstlisting}
octave:22> [x, i] =
newtonMethod('secantConvergenceFunction','secantConvergenceFunctionDerivative',6, 1e5, 1e-14, 1e-14, 'residueCriterion')
x =  5.00000000000000e+00
i =  8.00000000000000e+00
\end{lstlisting}
Pi\`u veloce di tre passi.
\end{oss}

\newpage
\begin{exercise}[2.7 e 2.8]
Per i testi degli esercizi consultare il libro di testo.
\end{exercise}
Ho costruito lo script \nameref{subsec:ScriptEser27}, lanciandolo
\begin{lstlisting}
octave:13> scriptExercise27
\end{lstlisting}
ottengo:	
\begin{multicols}{2}	
\begin{lstlisting}
bisection method
tol_x =0.01
x =  7.34375000000000e-01
i =  6.00000000000000e+00
tol_x =0.0001
x =  7.39013671875000e-01
i =  1.20000000000000e+01
tol_x =1e-06
x =  7.39084243774414e-01
i =  1.90000000000000e+01
tol_x =1e-08
x =  7.39085137844086e-01
i =  2.40000000000000e+01
tol_x =1e-10
x =  7.39085133187473e-01
i =  3.00000000000000e+01
tol_x =1e-12
x =  7.39085133214758e-01
i =  3.90000000000000e+01
tol_x =1e-14
x =  7.39085133215156e-01
i =  4.40000000000000e+01
tol_x =1e-16
x =  7.39085133215161e-01
i =  5.20000000000000e+01
\end{lstlisting}

\begin{lstlisting}
Newton method
tol_x =0.01
x =  7.39112890911362e-01
i =  2.00000000000000e+00
tol_x =0.0001
x =  7.39085133385284e-01
i =  3.00000000000000e+00
tol_x =1e-06
x =  7.39085133215161e-01
i =  4.00000000000000e+00
tol_x =1e-08
x =  7.39085133215161e-01
i =  4.00000000000000e+00
tol_x =1e-10
x =  7.39085133215161e-01
i =  4.00000000000000e+00
tol_x =1e-12
x =  7.39085133215161e-01
i =  5.00000000000000e+00
tol_x =1e-14
x =  7.39085133215161e-01
i =  5.00000000000000e+00
tol_x =1e-16
x =  7.39085133215161e-01
i =  5.00000000000000e+00
\end{lstlisting}

\columnbreak

\begin{lstlisting}
chord method
tol_x =0.01
x =  7.37506890513243e-01
i =  1.40000000000000e+01
tol_x =0.0001
x =  7.39071365298945e-01
i =  2.60000000000000e+01
tol_x =1e-06
x =  7.39085311606762e-01
i =  3.70000000000000e+01
tol_x =1e-08
x =  7.39085134772174e-01
i =  4.90000000000000e+01
tol_x =1e-10
x =  7.39085133228750e-01
i =  6.10000000000000e+01
tol_x =1e-12
x =  7.39085133214985e-01
i =  7.20000000000000e+01
tol_x =1e-14
x =  7.39085133215159e-01
i =  8.40000000000000e+01
tol_x =1e-16
x =  7.39085133215161e-01
i =  9.40000000000000e+01
\end{lstlisting}

\begin{lstlisting}
secant method
tol_x =0.01
x =  7.39119361911629e-01
i =  3.00000000000000e+00
tol_x =0.0001
x =  7.39085112127464e-01
i =  4.00000000000000e+00
tol_x =1e-06
x =  7.39085133215001e-01
i =  5.00000000000000e+00
tol_x =1e-08
x =  7.39085133215161e-01
i =  6.00000000000000e+00
tol_x =1e-10
x =  7.39085133215161e-01
i =  6.00000000000000e+00
tol_x =1e-12
x =  7.39085133215161e-01
i =  6.00000000000000e+00
tol_x =1e-14
x =  7.39085133215161e-01
i =  7.00000000000000e+00
tol_x =1e-16
x =  7.39085133215161e-01
i =  7.00000000000000e+00
\end{lstlisting}
\end{multicols}
Per l'ultima iterazione del metodo delle corde ho ottenuto uno warning di
divionse per zero.

