\section{Helps for \nameref{sec:metodoDiBisezione} section}

Riporto qui alcune righe di help per le funzioni che verranno utilizzate 
pi\`u spesso in questo capitolo:

\begin{lstlisting}
octave:1> help poly
`poly' is a function from the file /usr/share/octave/3.2.3/m/polynomial/poly.m

 -- Function File:  poly (A)
     If A is a square N-by-N matrix, `poly (A)' is the row vector of
     the coefficients of `det (z * eye (N) - a)', the characteristic
     polynomial of A.  As an example we can use this to find the
     eigenvalues of A as the roots of `poly (A)'.
          roots(poly(eye(3)))
          => 1.00000 - 0.00000i
          => 1.00000 + 0.00000i
     In real-life examples you should, however, use the `eig' function
     for computing eigenvalues.

     If X is a vector, `poly (X)' is a vector of coefficients of the
     polynomial whose roots are the elements of X.  That is, of C is a
     polynomial, then the elements of `D = roots (poly (C))' are
     contained in C.  The vectors C and D are, however, not equal due
     to sorting and numerical errors.

     See also: eig, roots
\end{lstlisting}

\begin{lstlisting}
octave:1> help ones
`ones' is a built-in function

 -- Built-in Function:  ones (X)
 -- Built-in Function:  ones (N, M)
 -- Built-in Function:  ones (N, M, K, ...)
 -- Built-in Function:  ones (..., CLASS)
     Return a matrix or N-dimensional array whose elements are all 1.
     The arguments are handled the same as the arguments for `eye'.

     If you need to create a matrix whose values are all the same, you
     should use an expression like

          val_matrix = val * ones (n, m)
\end{lstlisting}

\begin{lstlisting}
octave:1> help polyval
`polyval' is a function from the file /usr/share/octave/3.2.3/m/polynomial/polyval.m

 -- Function File: Y = polyval (P, X)
 -- Function File: Y = polyval (P, X, [], MU)
     Evaluate the polynomial at of the specified values for X.  When MU
     is present evaluate the polynomial for (X-MU(1))/MU(2).  If X is a
     vector or matrix, the polynomial is evaluated for each of the
     elements of X.

 -- Function File: [Y, DY] = polyval (P, X, S)
 -- Function File: [Y, DY] = polyval (P, X, S, MU)
     In addition to evaluating the polynomial, the second output
     represents the prediction interval, Y +/- DY, which contains at
     least 50% of the future predictions.  To calculate the prediction
     interval, the structured variable S, originating form `polyfit',
     must be present.

     See also: polyfit, polyvalm, poly, roots, conv, deconv, residue,
     filter, polyderiv, polyinteg
\end{lstlisting}

\begin{lstlisting}
octave:14> help ceil
`ceil' is a built-in function

 -- Mapping Function:  ceil (X)
     Return the smallest integer not less than X.  This is equivalent to
     rounding towards positive infinity.  If X is complex, return `ceil
     (real (X)) + ceil (imag (X)) * I'.
          ceil ([-2.7, 2.7])
            =>  -2   3

     See also: floor, round, fix
\end{lstlisting}
