\section{Round-off}

\begin{exercise}[1.5]
Per il testo dell'esercizio consultare il libro di testo.
\end{exercise}
Scrivo le rappresentazioni in formato stringa:
\begin{displaymath}
	\begin{split}
		\alpha_{0}\alpha_{1}\ldots\alpha_{15} = 0\underbrace{1\ldots1}_{13}01 & 
			\rightarrow  32765 \\
		\alpha_{0}\alpha_{1}\ldots\alpha_{15} = 0\underbrace{1\ldots1}_{13}10 & 
			\rightarrow  32766 \\
		\alpha_{0}\alpha_{1}\ldots\alpha_{15} = 0\underbrace{1\ldots1}_{13}11 & 
			\rightarrow  32767 \\
		\alpha_{0}\alpha_{1}\ldots\alpha_{15} = 1\underbrace{0\ldots0}_{15} & 
			\rightarrow  -32768 \\
	\end{split}
\end{displaymath}
Ad ogni passo incremento di uno, questo comporta che incrementando la rappresentazione
di $32767$ si riporta sul bit del segno $\alpha_{0}$. Per la definizione della funzione di 
valutazione ottengo $val(1\underbrace{0\ldots0}_{15}) = -2^{16} = -32768$.

\begin{exercise}[1.6]
$\mathcal{M}$ ha un numero finito di elementi.
\end{exercise}
\begin{proof}
Per far vedere che $\mathcal{M}$ ha un numero finito di elementi posso far \\
vedere che vale
$equinumerous(\mathcal{M}, \lbrace 1, \ldots, n\rbrace)$, con $n \in \mathbb{N}$.
Per la definizione della relazione $equinumerous$ posso costruire una funzione $f$
\emph{one-to-one} tale che $f: \mathcal{M} \rightarrow \mathbb{N}$.

Considerare la rappresentazione in formato stringa 
$x = \alpha_{0}\alpha_{1}\ldots\alpha_{m}\beta_{1}\ldots\beta_{s}$, $\forall x \in \mathcal{M}$.

Per ogni rappresentazione costruisco una nuova rappresentazione che astrae da $\alpha$ e $\beta$, 
ovvero $x = \alpha_{0}\alpha_{1}\ldots\alpha_{m}\beta_{1}\ldots\beta_{s} = \delta_{0} \ldots 
\delta_{m + s} = x'$
con $\alpha_{0} = \delta_{0}, \ldots, \alpha_{m} = \delta_{m}, \beta_{1} = \delta_{m + 1}, \ldots, 
\beta_{s} = \delta_{m + s}$.

Adesso posso costruire la funzione:
\begin{displaymath}
f(\delta_{0} \ldots \delta_{m + s}) = \sum_{i = 0}^{m + s}{\delta{i} * b^{m + s - i}}
\end{displaymath}
In questo modo ho costruito una biezione tra le rappresentazioni ed un sottoinsieme 
dei numeri naturali. \\\\
Adesso devo far vedere che $|\mathcal{M}| = n$ con $n \in \mathbb{N}$.

Ragiono per assurdo. Suppongo che $\not \exists n \in \mathbb{N}: |\mathcal{M}| = n$.
Considero la rappresentazione con massimo valore della funzione $f$ costruita in precedenza,
ovvero sia $x = \delta_{0} \ldots \delta_{m + s} = \underbrace{1 \ldots 1}_{m+s+1}$.
Per le ipotesi di assurdo, allora posso trovare una rappresentazione $x':f(x') \geq f(x)$.
Quindi la rappresentazione di $x'$ dovr\`a essere della forma $x' = \delta_{0}' \delta_{0} 
\ldots \delta_{m + s}$,
ovvero devo aggiungere un bit $\alpha_{0}'$ affinch\`e possa rappresentare 
$x' = 1 \underbrace{0 \ldots 0}_{m+s+1}$. Ma \`e impossibile costruire una rappresentazione
con $m+s+2$ simboli in quanto la dimensione delle rappresentazioni \`e fissata, uguale a $m+s+1$ 
e questo termina la prova.

\end{proof}

\begin{exercise}[1.6]
$r_{1} = b^{-\nu} \wedge r_{2} = (1 - b^{-m})b^{\varphi}$, con $\varphi = b^{s} - \nu$.
\end{exercise}
Suppongo che $x$ sia normalizzato.
\begin{itemize}
\item La configurazione che rappresenta il minimo
numero in valore assoluto (non considero il simbolo $\alpha_{0}$) \`e:
\begin{displaymath}
	r_{1} = \alpha_{1}.\alpha_{2} \ldots \alpha_{m} \beta_{1} \ldots \beta_{s} = 
		1.\underbrace{0 \ldots 0}_{m - 1} \underbrace{0 \ldots 0}_{s}
\end{displaymath}
A cui corrisponde la valutazione $val(r_{1}) = b^{-\nu}$.

\item La configurazione che rappresenta il massimo numero in valore assoluto
(non considero il simbolo $\alpha_{0}$) \`e:
\begin{displaymath}
	r_{2} = \alpha_{1}.\alpha_{2} \ldots \alpha_{m} \beta_{1} \ldots \beta_{s} =
		(b - 1).\underbrace{(b - 1)}_{m - 1}\underbrace{(b - 1)}_{s}
\end{displaymath}
Calcolo adesso la valutazione di $r_{2} = \rho b^{e - \nu}$. Ricavo $e$:
\begin{displaymath}
		e = (b-1)\sum_{i = 1}^{s}{b^{s-i}} = (b-1)\sum_{i = 0}^{s-1}{b^{i}} =
			 (b-1)\frac{b^{s}-1}{b-1} = b^{s}-1
\end{displaymath}
Ricavo $r_{2}$:
\begin{displaymath}
	\begin{split}
		r_{2} 	&= \left ( (b-1)\sum_{i = 1}^{m}{b^{1 - i}} \right)b^{b^{s}-1-\nu} = 
 			\left ( (b-1)\sum_{i = 1}^{m}{b^{-(i - 1)}} \right)b^{b^{s}-1-\nu} = \\
	 			&= \left ( (b-1)\sum_{i = 1}^{m}{\left ( \frac{1}{b} \right ) ^{i - 1}} 
					\right)b^{b^{s}-1-\nu} = 
					\left ( (b-1)\frac{\left ( \frac{1}{b} \right ) ^{m} - 1}{
						\left ( \frac{1}{b} \right ) - 1}
					\right)b^{b^{s}-1-\nu} = \\
				&= \left ( (b-1) \frac{b \left( b^{-m} - 1 \right )}{
						b \left ( \left ( \frac{1}{b} \right ) - 1 \right ) }
					\right)b^{b^{s}-1-\nu} = 
					\left ( (b-1) \frac{b \left( b^{-m} - 1 \right )}{1 - b} 
						\right)b^{b^{s}-1-\nu} = \\
				&= \left ( (b-1) \frac{b \left( 1 - b^{-m} \right )}{b - 1} 
					\right)b^{b^{s}-1-\nu} = 
					\left ( b \left( 1 - b^{-m} \right ) \right)b^{b^{s}-1-\nu} = 
					\left( 1 - b^{-m} \right ) b^{b^{s} - \nu}
	\end{split}
\end{displaymath}
\end{itemize}

\begin{exercise}
Dimostrare che vale $|\mathcal{M}| = b^{m+1}(exp_{max} - exp_{min} + 1) + 1$.
\end{exercise}
\begin{proof}
Il fattore $b^{m+1}$ contiene $+1$ nell'esponente in quanto ho $m$ simboli per
rappresentare una mantissa in valore assoluto, per\`o posso avere sia simboli positivi
che simboli negativi, quindi devo aggiungere il simbolo del segno $\alpha_{0}$.
Usando i risultati dell'esercizio precedente posso riscrivere:
\begin{displaymath}
\begin{split}
	exp_{max} = b^{s} - 1 - \nu, \quad exp_{min} = -\nu \\
	|\mathcal{M}| = b^{m+1}(b^{s} - 1 - \nu + \nu + 1) + 1 = b^{m+s+1} + 1
\end{split}
\end{displaymath}
Cerco adesso di dare una spiegazione meno formale, fissando come parametri per
la rappresentazione che uso in questo piccolo esempio $b = 2, m = 3, s = 2, \nu = 2$,
con mantisse normalizzate (implica $\alpha_{1} = 1$).
Segue quindi $exp_{max} = 1, exp_{min} = -2$. 

Per quanto riguarda le mantisse che posso rappresentare sono 
$$mantisse = \lbrace 1.00, 1.01, 1.10, 1.11 \rbrace$$

Per quanto riguarda i possibili esponenti, ottenuti partendo da $exp_{min}$ sommando
uno fino a $exp_{max}$ (sommo uno per la definizione dell'esponente per la rappresentazione
di numeri reali $-\nu + \sum_{i = 1}^{s}{\beta_{i} b^{s - i}}$):
$$esponenti = \left \lbrace 2^{-2} = \frac{1}{4}, 2^{-1} = \frac{1}{2},
		2^{0} = 1, 2^{1} = 2 \right \rbrace$$

Per cui posso rappresentare $|mantisse \times esponenti| = 16$ numeri positivi, 
a cui vanno aggiunti i $16$ opposti e lo zero, per un totale di $33 = 2^{5} + 1$.
\end{proof}

\begin{exercise}[1.8]
Per il testo dell'esercizio consultare il libro di testo.
\end{exercise}
Dato che rappresento mediante arrotondamento:
\begin{displaymath}
\begin{split}
u &= \frac{1}{2}b^{1-m} \quad , b = 10 \\
\log_{10}{u} &= \log_{10}{\left ( \frac{1}{2}10^{1-m} \right )} = -\log_{10}{2} + (1 - m) \\
m &= 1 - \log_{10}{2} - \log_{10}{u} \quad , u = 4.66 \times 10^{-10}
\end{split}
\end{displaymath}
\begin{lstlisting}
octave:10> 1 - log10(2) - log10(4.66e-10)
ans =  1.00305840876460e+01
\end{lstlisting}

\begin{oss}
Moltiplicare e dividere per la base $b$ di lavoro \`e ad errore zero se c'\`e 
"posto" sufficiente nell'esponente.
\end{oss}
Cerco di spiegare il significato della precedente osservazione con un esempio.
Sia $n = 1.1111 \times b^{-\nu}$, il quale ha il minimo esponente rappresentabile. Se 
moltiplico e successivamente divido per $10^{-4}$ devo dapprima denormalizzare, 
in quanto non posso rappresentare $b^{-(\nu + 4)}$, ottengo quindi:
\begin{displaymath}
	\frac{10^{-4} \times n}{10^{-4}} = \frac{0.0001 \times b^{-\nu}}{10^{-4}} = 
		10^{4} \times 0.0001 \times b^{-\nu} = 1.0000 \times b^{-\nu}
\end{displaymath}
Perdendo informazione sulle 4 cifre decimali di $n$.

\begin{exercise}[1.9]
Per il testo dell'esercizio consultare il libro di testo.
\end{exercise}
Formalizzo la richiesta: $-\log_{10}{u} = r$, con $r$ uguale al numero di cifre
esattamente rappresentate nella mantissa. Per definizione di $u$ segue che $u > 0$,
questo mi permette di non considerare i valori assoluti.

Distinguo i due casi, relativamente al metodo di rappresentazione usato:
\begin{description}
	\item[per troncamento] 
		\begin{displaymath}
		\begin{split}
			u &= b^{1-m} \\
			-\log_{10}{b^{1-m}} &= r \\
			(m-1)\log_{10}{b} &= r \Rightarrow \lbrace \text{fix } b = 10 \rbrace \\
			\Rightarrow m = 1 - \log_{10}{u}			
		\end{split}
		\end{displaymath}

	\item[per arrotondamento] 
		\begin{displaymath}
		\begin{split}
			u &= \frac{b^{1-m}}{2} \\
			-\log_{10}{\frac{b^{1-m}}{2}} &= r \\
			-\left( -\log_{10}{2} + (1-m)\log_{10}{b} \right) &= r \\
			\log_{10}{2} + (m-1)\log_{10}{b} &= r \Rightarrow 
				\lbrace \text{fix } b = 10 \rbrace \\
			\Rightarrow m = 1 - (\log_{10}{u} + \log_{10}{2}) = 1 - (\log_{10}{2u})
		\end{split}
		\end{displaymath}
\end{description}

\begin{exercise}[1.10]
Per il testo dell'esercizio consultare il libro di testo.
\end{exercise}
\begin{enumerate}
	\item per il \emph{pi\`u grande numero di macchina} $x$ considero la mantissa normalizzata,
	quindi per la definizione dello standard IEEE754 valgono: il massimo esponente che posso
	usare $e = 2046$, $\nu = 1023$.
	La mantissa massima che posso fissare \`e $\alpha_{1}.\alpha_{2} \ldots \alpha_{m}$:
	sempre per la definizione dello standard segue che $m = 53$ e preso $x$ normalizzato,
	segue $\alpha_{1} = 1$. Ottengo la rappresentazione 
		$mantissa(x) = 1.\underbrace{1 \ldots 1}_{52}$, con:
	\begin{displaymath}
	\begin{split}
		\rho &= \sum_{i = 1}^{m}{b^{1-i}} = \sum_{i = 0}^{m-1}{b^{-i}} = 
			%\sum_{i = 0}^{52}{b^{-i}} = % non ha molto senso fissare qui l'indice di arrivo
			\frac{b\left( b^{-m} - 1 \right)}{1 - b} = \frac{b^{1-m} - b}{1 - b} = \\
			&= \lbrace b = 2 \text{ by IEEE754 definition} \rbrace = -2(2^{-m}-1) = 
			2(1 - 2^{-m}) \\ \\
		val(x) &= \rho \times b^{2046-1023} = 2(1-2^{-53})2^{1023}
	\end{split}
	\end{displaymath}
	\begin{lstlisting}
		octave:18> 2*(1-2^-53)*2^1023
		ans =  1.79769313486232e+308
		octave:19> realmax
		ans =  1.79769313486232e+308
	\end{lstlisting}

	\item per il \emph{pi\`u piccolo numero di macchina} $x$ considero la mantissa normalizzata,
	quindi per la definizione dello standard IEEE754 valgono: il minimo esponente che posso
	usare $e = 1$, $\nu = 1023$.
	La mantissa minima che posso fissare \`e $\alpha_{1}.\alpha_{2} \ldots \alpha_{m}$:
	sempre per la definizione dello standard segue che $m = 53$ e preso $x$ normalizzato,
	segue $\alpha_{1} = 1$. Ottengo la rappresentazione 
		$mantissa(x) = 1.\underbrace{0 \ldots 0}_{52}$, con:
	\begin{displaymath}
		val(x) = b^{0} \times b^{1-1023} = 2^{-1022}
	\end{displaymath}
	\begin{lstlisting}
		octave:4> 2^-1022
		ans =  2.22507385850720e-308
		octave:5> realmin
		ans =  2.22507385850720e-308
	\end{lstlisting}

	\item per il \emph{pi\`u piccolo numero di macchina denormalizzato} $x$ considero la 
	mantissa denormalizzata,
	quindi per la definizione dello standard IEEE754 valgono: il minimo esponente che posso
	usare $e = 0$, $\nu = 1022$.
	La mantissa minima che posso fissare \`e $\alpha_{1}.\alpha_{2} \ldots \alpha_{m}$:
	sempre per la definizione dello standard segue che $m = 53$ e preso $x$ denormalizzato,
	segue $\alpha_{1} = 0$. Ottengo la rappresentazione 
		$mantissa(x) = 0.\underbrace{0 \ldots 0}_{51} 1$, con:
	\begin{displaymath}
		val(x) = b^{-52} \times b^{-1022} = 2^{-1074}
	\end{displaymath}
	\begin{lstlisting}
		octave:10> 2^-1074
		ans =  4.94065645841247e-324
	\end{lstlisting}
	
	\item per la \emph{precisione di macchina} considero la definizione dello 
		standard IEEE754; segue quindi  $m = 53$ e $b = 2$.
	\begin{displaymath}
		u = \frac{b^{1-m}}{2} = \frac{2^{1-53}}{2} = 2^{-53}
	\end{displaymath}
	\begin{lstlisting}
		octave:27> 2^-53
		ans =  1.11022302462516e-16
		octave:28> eps
		ans =  2.22044604925031e-16
	\end{lstlisting}

\end{enumerate}

\begin{exercise}[1.11]
 Spiegare il non funzionamento delle seguenti istruzioni:
\end{exercise}
\begin{lstlisting}
	x = 0; 
	delta = 0.1;
	while x ~= 1
		x = x + delta 
	end
\end{lstlisting}
Questo programma non termina, produce un ciclo infinito. Si ha questo comportamento
perch\`e non \`e possibile rappresentare correttamente il numero $0.1$ in macchina. 
Scrivo la sua rappresentazione in base 2:
\begin{displaymath}
\begin{split}
	0.1 \times 2 &= 0.2 \\
	0.2 \times 2 &= 0.4 \\
	0.4 \times 2 &= 0.8 \\
	0.8 \times 2 &= 1.6 \\
	0.6 \times 2 &= 1.2 \\
	0.2 \times 2 &= 0.4 \\
\end{split}
\end{displaymath}
Quindi $delta = (0.1)_{10} = (0.0\overline{0011})_{2}$, posso normalizzare e ottengo
$delta = 1.1\underbrace{0011 0011\ldots}_{51} \times 2^{-4}$. 

Sommando $delta$ ripetutamente ad $x$ non sar\`a possibile raggiungere l'uguaglianza
$x = 1$, quindi la guardia del while sar\`a sempre vera.

Una possibile soluzione \`e di irrobustire la guardia con $x \leq 1$, produce una 
computazione finita, ma non esegue lo stesso numero di passi della versione originale
del problema.

\begin{exercise}[1.12]
Per il testo dell'esercizio consultare il libro di testo.
\end{exercise}
Calcolare $\sqrt{x^{2} + y^{2}}$ usando l'uguaglianza dell'aritmetica classica 
$x^{2} = x \times x$ non \`e una buona strategia in quanto potrebbe non produrre
un output valido ($\forall x \geq \frac{realmax}
	{.9 \times 10^{155}} = \frac{1.8 \times 10^{308}}
	{.9 \times 10^{155}}, x \times x = \infty$) 
in aritmetica finita.
Questo lo dimostra il seguente codice:
\begin{lstlisting}
octave:23> x = .9e155
x =  9.00000000000000e+154
octave:24> x*x
ans = Inf
\end{lstlisting}
Posso quindi riformulare il problema introducendo una variabile $m$ tale che:
\begin{displaymath}
\begin{split}
	m &= max \lbrace |x|, |y| \rbrace \\
	\sqrt{x^{2} + y^{2}} &= m \sqrt{{\left |\frac{x}{m} \right |}^{2} + 
		{\left |\frac{y}{m} \right |}^{2}}
\end{split}
\end{displaymath}
Posso osservare che:
\begin{displaymath}
 \sqrt{x^{2} + y^{2}} = \left \lbrace
		\begin{array}{lc}
			|x| \sqrt{1 + {\left |\frac{y}{m} \right |}^{2}} & \text{se } m = |x| \\ 
			|y| \sqrt{1 + {\left |\frac{x}{m} \right |}^{2}} & \text{se } m = |y|
		\end{array} \right .
\end{displaymath}
Lo schema che posso costruire da questo esempio \`e di utilizzare una buona approssimazione
($\sqrt{1 + \ldots}$) sommando poi un errore "piccolo" (il termine quadratico). Questo aspetto 
verr\`a approfondito nelle osservazioni successive.

\begin{exercise}[1.13]
Implementare i seguenti programmi:
\begin{displaymath}
\begin{split}
	\left ( \left ( \frac{eps}{2} + 1 \right ) - 1 \right ) \left ( \frac{2}{frac} \right ) \\
	\left ( \frac{eps}{2}  + \left ( 1 - 1 \right ) \right ) \left ( \frac{2}{frac} \right )
\end{split}
\end{displaymath}
\end{exercise}
Questo l'help della funzione $eps$:
\begin{lstlisting}
octave:31> help eps
	`eps' is a built-in function

     Return a scalar, matrix or N-dimensional array whose elements are
     all eps, the machine precision.  More precisely, `eps' is the
     relative spacing between any two adjacent numbers in the machine's
     floating point system.  This number is obviously system dependent.
     On machines that support IEEE floating point arithmetic, `eps' is
     approximately 2.2204e-16 for double precision and 1.1921e-07 for
     single precision.
\end{lstlisting}
Questo il codice che implementa:
\begin{lstlisting}
octave:30> ((eps/2 + 1) - 1) * (2/eps)
ans =  0.00000000000000e+00
octave:31> (eps/2 + (1 - 1)) * (2/eps)
ans =  1.00000000000000e+00
\end{lstlisting}
Perch\`e:
\begin{lstlisting}
octave:46> 2/eps
ans =  9.00719925474099e+15
octave:47> eps/2
ans =  1.11022302462516e-16
octave:48> eps/2 + 1
ans =  1.00000000000000e+00
\end{lstlisting}

\begin{exercise}[1.14]
Per il testo dell'esercizio consultare il libro di testo.
\end{exercise}
Questo il codice che implementa:
\begin{lstlisting}
octave:50> (1e300-1e300)*1e300
ans =  0.00000000000000e+00
octave:51> (1e300*1e300)-(1e300*1e300)
ans = NaN
\end{lstlisting}
Perch\`e:
\begin{lstlisting}
octave:52> (1e300*1e300)
ans = Inf
\end{lstlisting}
Osservo che \emph{octave:50} riesce ad eseguire la computazione in quanto $1e300 \in \mathcal{M}$,
mentre \emph{octave:51} restituisce \emph{NaN} perch\`e $1e300\times1e300 \not \in \mathcal{M}$
(\emph{overflow}) e operazioni che coinvolgono \emph{Inf} restituiscono \emph{NaN} 
per l'implementazione di Octave. Non vale la propriet\`a distributiva, riporto l'help di Octave:
\begin{lstlisting}
octave:53> help NaN
`NaN' is a built-in function

     Return a scalar, matrix, or N-dimensional array whose elements are
     all equal to the IEEE symbol NaN (Not a Number).  NaN is the
     result of operations which do not produce a well defined numerical
     result.  Common operations which produce a NaN are arithmetic with
     infinity (Inf - Inf), zero divided by zero (0/0), and any
     operation involving another NaN value (5 + NaN).

     Note that NaN always compares not equal to NaN (NaN != NaN).  This
     behavior is specified by the IEEE standard for floating point
     arithmetic.  To find NaN values, use the `isnan' function.

\end{lstlisting}

\begin{exercise}[1.15]
Per il testo dell'esercizio consultare il libro di testo.
\end{exercise}
In aritmetica finita quello che si chiede di calcolare \`e, usando l'equazione 1.18:
\begin{displaymath}
\begin{split}
	fl(d) = fl(fl(fl(x) + fl(y)) + fl(z)) \\
	fl(e) = fl(fl(x) + fl(fl(y) + fl(z)))
\end{split}
\end{displaymath}
Usando il \emph{Teorema 1.4} posso riscrivere la prima equazione:
\begin{displaymath}
\begin{split}
	fl(d) &= fl(fl(x + x\varepsilon_{x} + y + y\varepsilon_{y}) + fl(z)) = \\
	  &=  fl(x + x\varepsilon_{x} + y + y\varepsilon_{y} + 
			x\varepsilon_{w} + x\varepsilon_{x}\varepsilon_{w} + 
			y\varepsilon_{w} + y\varepsilon_{y}\varepsilon_{w} + fl(z))
\end{split}
\end{displaymath}
Posso non considerare i termini in cui compare un prodotto di due errori relativi
$\varepsilon_{1} \varepsilon_{2}$, ottenendo:
\begin{displaymath}
	fl(d) = fl(x + x\varepsilon_{x} + y + y\varepsilon_{y} + 
			x\varepsilon_{w} +  y\varepsilon_{w} + fl(z))
\end{displaymath}
Posso sviluppare il termine $fl(z)$ e successivamente la funzione $fl$ pi\`u esterna:
\begin{displaymath}
\begin{split}
	fl(d) &= fl(x + x\varepsilon_{x} + y + y\varepsilon_{y} + 
			x\varepsilon_{w} +  y\varepsilon_{w} + z + z\varepsilon_{z}) = \\
	  &= x + x\varepsilon_{x} + y + y\varepsilon_{y} + 
			x\varepsilon_{w} +  y\varepsilon_{w} + z + z\varepsilon_{z} + 
		x\varepsilon_{q} + x\varepsilon_{x}\varepsilon_{q} + y\varepsilon_{q} + 
			y\varepsilon_{y}\varepsilon_{q} + 
			x\varepsilon_{w}\varepsilon_{q} + \\
	  &+  y\varepsilon_{w}\varepsilon_{q} +
			 z\varepsilon_{q} + z\varepsilon_{z}\varepsilon_{q}
\end{split}
\end{displaymath}
Non considero i termini con $\varepsilon_{1} \varepsilon_{2}$ e sviluppo $fl(d)$:
\begin{displaymath}
	d + d\varepsilon_{d} = x + x\varepsilon_{x} + y + y\varepsilon_{y} + 
			x\varepsilon_{w} +  y\varepsilon_{w} + z + z\varepsilon_{z} + 
		x\varepsilon_{q} + y\varepsilon_{q} + 
			 z\varepsilon_{q}
\end{displaymath}
Per la definizione del problema $d = x + y + z$ (in aritmetica esatta questa uguaglianza
vale, senza dover specificare le parentesi, perch\`e vale la propriet\`a associativa):
\begin{displaymath}
	\varepsilon_{d} = \frac{
			x(\varepsilon_{x} + \varepsilon_{w} + \varepsilon_{q}) + 
			y(\varepsilon_{y} + \varepsilon_{w} + \varepsilon_{q}) + 
			z(\varepsilon_{z} + \varepsilon_{q})}{x + y + z}
\end{displaymath}
Con argomento simmetrico si dimostra la seconda equazione $fl(e) = fl(fl(x) + fl(fl(y) + fl(z)))$,
ottenendo:
\begin{displaymath}
	\varepsilon_{d} = \frac{
			x(\varepsilon_{x} + \varepsilon_{q}) + 
			y(\varepsilon_{y} + \varepsilon_{w} + \varepsilon_{q}) + 
			z(\varepsilon_{z} + \varepsilon_{w} + \varepsilon_{q})}{x + y + z}
\end{displaymath}
Fisso $\varepsilon_{max} = max \lbrace |\varepsilon_{x}|, |\varepsilon_{y}|, |\varepsilon_{w}|,
|\varepsilon_{z}|, |\varepsilon_{q}|  \rbrace$
posso quindi maggiorare $\varepsilon_{d}$:
\begin{displaymath}
	|\varepsilon_{d}| \leq \frac{|x| + |y| + |z|}{|x + y + z|} \varepsilon_{max}
\end{displaymath}
Osservo che se $xy > 0 \wedge yz > 0$ allora $\varepsilon_{d} \leq \varepsilon_{max}$ ed il
problema \`e ben condizionato, altrimenti se 
$xy < 0 \vee yz < 0 \vee xz < 0 \vee (x + y + z) \rightarrow 0$ allora il problema \`e mal
condizionato.

\begin{exercise}
Riprendere l'esercizio precedente, siano $a, b, c \in \mathcal{M}, b = 10, m = 4$, con
rappresentazione della mantissa mediante arrotondamento. Verificare le disuguaglianza:
\begin{displaymath}
	(a + b) + c \not = a + (b + c)
\end{displaymath}
con $a = 2.000 \times 10^{-4}, b = 4.000 \times 10^{-4}, c = 7.000 \times 10^{0}$.
\end{exercise}
\begin{displaymath}
\begin{split}
	fl(fl(a + b) + c) &= fl(6.000 \times 10^{-4} + 7.000 \times 10^{0}) =  \\
	&= fl(0.0006 \times 10^{0} + 7.000 \times 10^{0}) = \\
	&= fl(7.0006 \times 10^{0}) = 7.001 \times 10^{0}
\end{split}
\end{displaymath}
\begin{displaymath}
\begin{split}
	fl(a + fl(b + c)) &= fl(2.000 \times 10^{-4} + 7.000 \times 10^{0}) = \\
	&= fl(0.0002 \times 10^{0} + 7.000 \times 10^{0}) = \\
  	&= fl(7.0002 \times 10^{0}) = 7.000 \times 10^{0}
\end{split}
\end{displaymath}
Nei precedenti passaggi \`e sbagliato calcolare $fl(a), fl(b), fl(c)$ in quanto, per ipotesi,
$a, b, c$ sono gi\`a un numeri di macchina (e quindi gi\`a affetti da errore).

\begin{exercise}[1.16]
Per il testo dell'esercizio consultare il libro di testo.
\end{exercise}
\begin{proof}
Formalmente devo studiare:
\begin{displaymath}
	\tilde{y} = f(\tilde{x}), \quad \quad y = f(x) = \sqrt{x}
\end{displaymath}
Posso riscrivere $\tilde{y}, \tilde{x}$ introducendo gli errori relativi e, per 
la definizione di errore relativo, vale:
\begin{displaymath}
	y(1 + \varepsilon_{y}) = f(x(1 + \varepsilon_{x}))
\end{displaymath}
La precedente relazione utilizza l'implementazione esatta di $f$, quindi considero
lo sviluppo di $f$ al primo ordine centrato in $x$:
\begin{displaymath}
\begin{split}
	y(1 + \varepsilon_{y}) & = f(x) + f'(x)(x + x\varepsilon_{x} - x) + O(\varepsilon_{x}^{2}) \\
	y\varepsilon_{y} & = f'(x)x\varepsilon_{x} + O(\varepsilon_{x}^{2})
\end{split}
\end{displaymath}
Per le ipotesi del problema, usando per la sostituzione quanto detto ad inizio prova, vale:
\begin{displaymath}
f(x) = x^{\frac{1}{2}} \quad \quad f'(x) = \frac{1}{2}x^{-\frac{1}{2}} 
= \frac{1}{2\sqrt{x}}
\end{displaymath}
Posso approssimare:
\begin{displaymath}
|\varepsilon_{y}| \approx \left | \frac{1}{2\sqrt{x}} \frac{x}{y} \right||\varepsilon_{x}|= 
\left | \frac{1}{2\sqrt{x}} \frac{x}{\sqrt{x}} \right||\varepsilon_{x}| = 
\frac{|\varepsilon_{x}|}{2}
\end{displaymath}
Ottengo $k = \frac{1}{2}$ come richiesto, la funzione $f(x) = \sqrt{x}$
\`e sempre ben condizionata, la prova \`e conclusa.
\end{proof}

\begin{exercise}[1.17]
Per il testo dell'esercizio consultare il libro di testo.
\end{exercise}
Per la definizione del problema vale:
\begin{displaymath}
\begin{split}
	x_{1} = 0.12345678 & \quad x_{2} = 0.12341234 \\
	fl(x_{1}) = 1.235 \times 10^{-1} & \quad fl(x_{2}) = 1.234 \times 10^{-1} \\
	y = x_{1} - x_{2} &= 0.00004444 = 4.444 \times 10^{-5} \\
	fl(x_{1}) - fl(x_{2}) &= 0.001 \times 10^{-1} = 1.000 \times 10^{-4}
\end{split}
\end{displaymath}
Osservo che le due differenze, quella esatta con quella in aritmetica finita,
non hanno nessuna cifra in comune, vengono perse tutte le cifre significative.

L'errore relativo che si commette sull'intera operazione \`e:
\begin{displaymath}
	\varepsilon_{y} = \frac{4.444 \times 10^{-5} - 1.000 \times 10^{-4}}{4.444 \times 10^{-5}} = 
		 \frac{4.444 \times 10^{-5} - 10.000 \times 10^{-5}}{4.444 \times 10^{-5}} 
\end{displaymath}
\begin{lstlisting}
octave:5> (4.444e-5 - 10e-5) / 4.444e-5
ans = -1.2502
\end{lstlisting}
Se considero l'analisi del condizionamento ottengo:
\begin{displaymath}
\begin{split}
	\varepsilon_{x_{1}} & = \frac{1.2345678 \times 10^{-1} - 1.235 \times 10^{-1}}
			{1.2345678 \times 10^{-1}} \\
	\varepsilon_{x_{2}} & = \frac{1.2341234 \times 10^{-1} - 1.234 \times 10^{-1}}
			{1.2341234 \times 10^{-1}}
\end{split}
\end{displaymath}
\begin{lstlisting}
octave:10> x1 = (1.2345678e-1 - 1.235e-1) / 1.2345678e-1
x1 = -3.50082028706699e-04
octave:11> x2 = (1.2341234e-1 - 1.234e-1) / 1.2341234e-1
x2 =  9.99900009998966e-05
octave:17> epsilon_Max = max(abs(x1),abs(x2))
epsilon_Max =  3.50082028706699e-04
\end{lstlisting}
\begin{displaymath}
\begin{split}
	\varepsilon_{max} & = max \lbrace |\varepsilon_{x_{1}}|, |\varepsilon_{x_{2}}| \rbrace 
		= |\varepsilon_{x_{1}}| = 3.50082028706699 \times 10^{-4}\\
	|\varepsilon_{y}| & \leq \frac{|1.2345678 \times 10^{-1}| + |1.2341234 \times 10^{-1}|}
		{|1.2345678 \times 10^{-1} - 1.2341234 \times 10^{-1}|}\varepsilon_{max}
\end{split}
\end{displaymath}
\begin{lstlisting}
octave:15> ((abs(1.2345678e-1)+abs(1.2341234e-1))/abs(1.2345678e-1 - 1.2341234e-1))*max(abs(x1),abs(x2))
ans =  1.94474442742179e+00
\end{lstlisting}
Torna infatti la stima dell'errore fatta considerando l'errore sull'intera 
operazione $|\varepsilon_{y}| = 1.2502$ con quella fatta considerando il 
condizionamento sui dati in ingresso$|\varepsilon_{y}| \leq 1.94474442742179$.

\begin{exercise}[1.18]
\label{exercise:numericalEraseExercise}
Per il testo dell'esercizio consultare il libro di testo.
\end{exercise}
Implemento:
\begin{lstlisting}
octave:1> format long e
octave:2> a = 0.1
a =  1.00000000000000e-01
octave:3> b = 0.099999999999
b =  9.99999999990000e-02
octave:4> a-b
ans =  1.00000563385549e-12
\end{lstlisting}
Eseguo lo studio sul condizionamento. Per il \emph{Teorema 1.4} il massimo errore
che posso commettere usando il metodo di arrotondamento \`e $u = \frac{1}{2}b^{1-m}$.
Lo standard IEEE754 fissa $b = 2, m = 53$, quindi $u = 2^{-53}$. Questo errore
\`e comune per entrambi in quanto non devo applicare la funzione $fl$ ai due numeri $a, b$,
quindi $\varepsilon_{max} = \varepsilon_{a} = \varepsilon_{b} = 2^{-53} = 
1.11022302462516 \times 10^{-16}$.

Per l'equazione 1.23 vale:
\begin{displaymath}
\begin{split}
	\varepsilon_{y} & \leq \frac{|1.00000000000000 \times 10^{-1}| + 
			|9.99999999990000 \times 10^{-2}|}
		{|1.00000000000000 \times 10^{-1} - 
			9.99999999990000 \times 10^{-2}|}\varepsilon_{max} \\
	\varepsilon_{y} & \leq \frac{1.99999999999000\times 10^{-1}}
		{1.00000563385549\times 10^{-12}}\varepsilon_{max} = 
		(1.99998873234249\times 10^{11})\varepsilon_{max} = \\
	& = 2.22043353963751\times 10^{-5}
\end{split}
\end{displaymath}
Si osserva che $k \in O(10^{11})$, il problema \`e mal condizionato.

Queste le giustificazioni dei valori precedenti:
\begin{lstlisting}
octave:15> epsilon_Max = 2^-53
epsilon_Max =  1.11022302462516e-16
octave:16> a+b
ans =  1.99999999999000e-01
octave:17> abs(a-b)
ans =  1.00000563385549e-12
octave:18> (a+b)/(abs(a-b))
ans =  1.99998873234249e+11
octave:19> (a+b)/(abs(a-b))*epsilon_Max
ans =  2.22043353963751e-05
\end{lstlisting}

Calcolo adesso l'errore relativo, considerando come valore esatto la differenza
$a - b = 1 \times 10^{-12}$:
\begin{displaymath}
	\varepsilon_{x} = \frac{(1.00000563385549 \times 10^{-12} - 1 \times
	10^{-12})}{1 \times 10^{-12}}
\end{displaymath}
Ottengo:
\begin{lstlisting}
octave:4> (1.00000563385549e-12 - 1e-12)/1e-12
ans =  5.63385549010602e-06
\end{lstlisting}
















