\section{Round-off}

\begin{exercise}[1.5]
Per il testo dell'esercizio consultare il libro di testo.
\end{exercise}
Scrivo le rappresentazioni in formato stringa:
\begin{displaymath}
	\begin{split}
		\alpha_{0}\alpha_{1}\ldots\alpha_{15} = 0\underbrace{1\ldots1}_{13}01 & 
			\rightarrow  32765 \\
		\alpha_{0}\alpha_{1}\ldots\alpha_{15} = 0\underbrace{1\ldots1}_{13}10 & 
			\rightarrow  32766 \\
		\alpha_{0}\alpha_{1}\ldots\alpha_{15} = 0\underbrace{1\ldots1}_{13}11 & 
			\rightarrow  32767 \\
		\alpha_{0}\alpha_{1}\ldots\alpha_{15} = 1\underbrace{0\ldots0}_{15} & 
			\rightarrow  -32768 \\
	\end{split}
\end{displaymath}
Ad ogni passo incremento di uno, questo comporta che incrementando la rappresentazione
di $32767$ si "invade" il bit del segno $\alpha_{0}$. Per la definizione della funzione di 
valutazione ottengo $val(1\underbrace{0\ldots0}_{15}) = -2^{16} = -32768$.

\begin{exercise}[1.6]
$\mathbb{M}$ ha un numero finito di elementi.
\end{exercise}
\begin{proof}
Per far vedere che $\mathbb{M}$ ha un numero finito di elementi posso far \\
vedere che vale
$equinumerous(\mathbb{M}, \lbrace 1, \ldots, n\rbrace)$, con $n \in \mathbb{N}$.
Per la definizione della relazione $equinumerous$ posso costruire una funzione $f$
\emph{one-to-one} tale che $f: \mathbb{M} \rightarrow \mathbb{N}$.

Considerare la rappresentazione in formato stringa 
$x = \alpha_{0}\alpha_{1}\ldots\alpha_{m}\beta_{1}\ldots\beta_{s}$, $\forall x \in \mathbb{M}$.

Per ogni rappresentazione costruisco una nuova rappresentazione che astrae da $\alpha$ e $\beta$, 
ovvero $x = \alpha_{0}\alpha_{1}\ldots\alpha_{m}\beta_{1}\ldots\beta_{s} = \delta_{0} \ldots 
\delta_{m + s} = x'$
con $\alpha_{0} = \delta_{0}, \ldots, \alpha_{m} = \delta_{m}, \beta_{1} = \delta_{m + 1}, \ldots, 
\beta_{s} = \delta_{m + s}$.

Adesso posso costruire la funzione:
\begin{displaymath}
f(\delta_{0} \ldots \delta_{m + s}) = \sum_{i = 0}^{m + s}{\delta{i} * b^{m + s - i}}
\end{displaymath}
In questo modo ho costruito una biezione tra le rappresentazioni ed un sottoinsieme 
dei numeri naturali. \\\\
Adesso devo far vedere che $|\mathbb{M}| = n, n \in \mathbb{N}$.

Ragiono per assurdo. Suppongo che $\not \exists n \in \mathbb{N}: |\mathbb{M}| = n$.
Considero la rappresentazione con massimo valore della funzione $f$ costruita in precedenza,
ovvero sia $x = \delta_{0} \ldots \delta_{m + s} = \underbrace{1 \ldots 1}_{m+s+1}$.
Per le ipotesi di assurdo, allora posso trovare una rappresentazione $x':f(x') \geq f(x)$.
Quindi la rappresentazione di $x'$ dovr\`a essere della forma $x' = \delta_{0}' \delta_{0} 
\ldots \delta_{m + s}$,
ovvero devo aggiungere un bit $\alpha_{0}'$ affinch\`e possa rappresentare 
$x' = 1 \underbrace{0 \ldots 0}_{m+s+1}$. Ma \`e impossibile costruire una rappresentazione
con $m+s+2$ simboli in quanto la dimensione delle rappresentazioni \`e fissata, uguale a $m+s+1$ 
e questo termina la prova.

\end{proof}

\begin{exercise}[1.6]
$r_{1} = b^{-\nu} \wedge r_{2} = (1 - b^{-m})b^{\varphi}$, con $\varphi = b^{s} - \nu$.
\end{exercise}
Suppongo che $x$ sia normalizzato.
\begin{itemize}
\item La configurazione che rappresenta il minimo
numero in valore assoluto (non considero il simbolo $\alpha_{0}$) \`e:
\begin{displaymath}
	r_{1} = \alpha_{1}.\alpha_{2} \ldots \alpha_{m} \beta_{1} \ldots \beta_{s} = 
		1.\underbrace{0 \ldots 0}_{m - 1} \underbrace{0 \ldots 0}_{s}
\end{displaymath}
A cui corrisponde la valutazione $val(r_{1}) = b^{-\nu}$.

\item La configurazione che rappresenta il massimo numero in valore assoluto
(non considero il simbolo $\alpha_{0}$) \`e:
\begin{displaymath}
	r_{2} = \alpha_{1}.\alpha_{2} \ldots \alpha_{m} \beta_{1} \ldots \beta_{s} =
		(b - 1).\underbrace{(b - 1)}_{m - 1}\underbrace{(b - 1)}_{s}
\end{displaymath}
Calcolo adesso la valutazione di $r_{2} = \rho b^{e - \nu}$. Ricavo $e$:
\begin{displaymath}
		e = (b-1)\sum_{i = 1}^{s}{b^{s-i}} = (b-1)\sum_{i = 0}^{s-1}{b^{i}} =
			 (b-1)\frac{b^{s}-1}{b-1} = b^{s}-1
\end{displaymath}
Ricavo $r_{2}$:
\begin{displaymath}
	\begin{split}
		r_{2} 	&= \left ( (b-1)\sum_{i = 1}^{m}{b^{1 - i}} \right)b^{b^{s}-1-\nu} = 
 			\left ( (b-1)\sum_{i = 1}^{m}{b^{-(i - 1)}} \right)b^{b^{s}-1-\nu} = \\
	 			&= \left ( (b-1)\sum_{i = 1}^{m}{\left ( \frac{1}{b} \right ) ^{i - 1}} 
					\right)b^{b^{s}-1-\nu} = 
					\left ( (b-1)\frac{\left ( \frac{1}{b} \right ) ^{m} - 1}{
						\left ( \frac{1}{b} \right ) - 1}
					\right)b^{b^{s}-1-\nu} = \\
				&= \left ( (b-1) \frac{b \left( b^{-m} - 1 \right )}{
						b \left ( \left ( \frac{1}{b} \right ) - 1 \right ) }
					\right)b^{b^{s}-1-\nu} = 
					\left ( (b-1) \frac{b \left( b^{-m} - 1 \right )}{1 - b} 
						\right)b^{b^{s}-1-\nu} = \\
				&= \left ( (b-1) \frac{b \left( 1 - b^{-m} \right )}{b - 1} 
					\right)b^{b^{s}-1-\nu} = 
					\left ( b \left( 1 - b^{-m} \right ) \right)b^{b^{s}-1-\nu} = 
					\left( 1 - b^{-m} \right ) b^{b^{s} - \nu}
	\end{split}
\end{displaymath}

\begin{exercise}[1.6]
Per il testo dell'esercizio consultare il libro di testo.
\end{exercise}
Dato che rappresento mediante arrotondamento:
\begin{displaymath}
\begin{split}
u &= \frac{1}{2}b^{1-m} \quad , b = 10 \\
\log_{10}{u} &= \log_{10}{\left ( \frac{1}{2}10^{1-m} \right )} = -\log_{10}{2} + (1 - m) \\
m &= 1 - \log_{10}{2} - \log_{10}{u} \quad , u = 4.66 * 10^{-10}
\end{split}
\end{displaymath}
\begin{lstlisting}
octave:10> 1 - log10(2) - log10(4.66e-10)
ans =  1.00305840876460e+01
\end{lstlisting}


\end{itemize}
