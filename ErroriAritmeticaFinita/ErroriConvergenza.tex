\section{Convergenza}

\begin{exercise}[1.3] 
\label{exercise:exerciseIterativeMethodFixedPoint}
Dimostrare che il metodo iterativo $$x_{n+1}=\phi(x_{n})$$ convergente a x*,
deve verificare la condizione di \textbf{consistenza} $$x^{*}=\phi(x^{*})$$
ovvero la soluzione cercata deve essere un punto fisso per la funzione di
iterazione che definisce il metodo.
\end{exercise}
\begin{proof}
Suppongo che il metodo $\phi$ sia monotono e convergente a $x^{*}$. Definisco
il preordine $\rightarrow$ per catturare la convergenza:
\begin{displaymath}
\rightarrow = \lbrace (x_{n}, x_{n+1}) : x_{n+1} = \phi(x_{n}) \wedge \lim_{n
\rightarrow \infty}{x_{n}} = x^{*}
\rbrace
\end{displaymath}
Suppongo \emph{per assurdo} che $$\phi(x^{*}) = x^{\triangle} \not = x^{*}$$
Per l' ipotesi $x_{n} \rightarrow x^{*}$ (per la transitivit\`a di
$\rightarrow$), posso applicare la monotonia di $\phi$ rispetto a $\rightarrow$:
$$\phi(x_{n}) \rightarrow \phi(x^{*}) = x^{\triangle}$$ ma questo contraddice 
l'ipotesi che $\phi$ converge a $x^{*}$.
\end{proof}

\begin{oss}
La prova precedente \`e stata scritta da me per\`o rivedendola ad un ricevimento
con la professoressa Sestini, abbiamo visto che richiede che $\phi$ sia
monotona. Questa richiesta \`e pi\`u restrittiva rispetto agli argomenti
trattati nel testo. Una prova pi\`u semplice \`e la seguente:
\end{oss}
\begin{proof}
Sia $\phi$ continua, per l'equazione $(1.2)$ del testo, vale $x_{n} \rightarrow
x^{*}$ e $x_{n + 1} \rightarrow x^{*}$. Per la definizione del metodo iterativo
si ha:
\begin{displaymath}
x_{n + 1} = \phi(x_{n})
\end{displaymath}
Dato che $\phi$ si assume continua e per l'equazione $(1.2)$ allora vale
\begin{displaymath}
\begin{split}
x_{n + 1} &= \phi(x_{n}) \\
\downarrow & \quad \downarrow \\
x^{*} &= \phi(x^{*})
\end{split}
\end{displaymath}
E questo dimostra che se il metodo \`e convergente allora soddisfa la condizione
necessaria di consistenza.
\end{proof}

\begin{oss}
L'ultima frase della precedente prova \`e importante, ovvero se un metodo
converge alla soluzione esatta $x^{*}$ allora soddisfa la condizione necessaria,
non \`e vero il contrario. Infatti un metodo potrebbe soddisfare la condizione
necessaria alla convergenza ma non convergere alla soluzione esatta.
\end{oss}

\begin{exercise}
Considerare il metodo iterativo:
\begin{displaymath}
	x_{n+1} = \phi(x_{n}) = \frac{1}{2} \left ( x_{n} + \frac{2}{x_{n}} \right ), 
		\quad x_{0} = 2
\end{displaymath}
dimostrare che il metodo genera una successione di approssimazioni tale che 
$x_{n} \rightarrow \sqrt{2}$.
\end{exercise}
\begin{proof}
Questo metodo \`e a passo \emph{singolo} in quanto usa un solo innesto per calcolare
$x_{n + 1}$.
Affinch\`e il metodo iterativo sia convergente, per l'esercizio 
\ref{exercise:exerciseIterativeMethodFixedPoint} posso costruire
un'equazione per trovare il punto fisso di $\phi$:
\begin{displaymath}
	\phi(x) =  \frac{1}{2} \left ( x + \frac{2}{x} \right ) = x 
\end{displaymath}
Manipolando: $x + \frac{2}{x} = 2x \Rightarrow x^{2} + 2 = 2x^{2} \Rightarrow 
	2 = x^{2}$.
\end{proof}

\begin{exercise}[1.4] Per il testo dell'esercizio consultare il libro di testo.
\end{exercise}
Eseguendo il codice riportato nella sezione \nameref{sec:iterativeMethodToSQRT2}, 
questo \`e l'output di Octave sulla mia macchina:
\begin{lstlisting}
octave:8> iterative(2, 1.5)
next =  1.42857142857143e+00
Difference with last step:0.0714285714285714
next =  1.41463414634146e+00
Difference with last step:0.0139372822299655
next =  1.41421568627451e+00
Difference with last step:0.000418460066953008
next =  1.41421356268887e+00
Difference with last step:2.12358564088966e-06
next =  1.41421356237310e+00
Difference with last step:3.15774073555986e-10
Difference with sqrt(2):0
\end{lstlisting}
In 5 passi si raggiunge la precisione richiesta, uno in pi\`u rispetto ai risultati
mostrati in \emph{Tabella 1.1} del libro.



