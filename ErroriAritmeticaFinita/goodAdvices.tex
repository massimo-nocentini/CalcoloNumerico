\section{Consigli per operazioni di macchina}
La seguente lista porta dei cosigli su come eseguire le operazioni di macchina a
quando si deve implementare un problema formulato in aritmetica esatta.

\begin{enumerate}
\item \emph{good approximation plus a small correction term} ecco alcuni
esempi
\begin{displaymath}
\begin{split}
a + \frac{b - a}{2} \quad & \text{\`e migliore di} \quad \frac{a + b}{2} \quad
\text{hint: } a - \frac{a}{2} = \frac{a}{2} \\ 
x - \frac{x^{2} - a}{2x} \quad & \text{\`e migliore di} \quad \frac{1}{2}\left
(x + \frac{a}{x} \right ) \quad \text{hint: } x - \frac{x}{2} = \frac{x}{2} \\
x_{2} - y_{2} \frac{x_{2} - x_{1}}{y_{2} - y_{1}} \quad & \text{\`e migliore di}
\quad \frac{y_{2}x_{1} - x_{2}y_{1}}{y_{2} - y_{1}} \quad
\text{hint: add } x_{2} - x_{2} \text{ and factor } -x_{2}
\end{split}
\end{displaymath}
\item \emph{add the small term first} quando devo sommare una collezione di
valori alla quale appartengono valori relativamente piccoli \`e una buona idea ordinare
i valori in modo decrescente ed iniziare a sommare dai valori pi\`u piccoli ai
valori pi\`u grandi.
\item \emph{be careful when substracting almost equals numbers} la differenza
\`e una operazione di macchina mal condizionata, quindi deve essere utilizzata con
cautela. \`E utile nei termini di correzione (vedi due punti sopra) nella
riformulazione di un problema, mentre \`e causa di cancellazione numerica se
utilizzata nel calcolare informazioni essenziali come una prima approssimazione
(vedi esercizio \ref{exercise:numericalEraseExercise}).
\item \emph{avoid large partial result on the road to a small final answer}
prendiamo come esempio la funzione $f(x) = e^{x}$ ed il suo sviluppo di
MacLaurin  $Ml(x)$. Se applico ad $x = -10$, ottengo $f(-10) = 4.54e-5, Ml(-10) 
\approx 2700$. Una soluzione potrebbe essere riformulare il problema, calcolando
$Ml(10)$ al posto di $Ml(-10)$ e prendendo il reciproco.
\item \emph{use mathematical reformulation to avoid 3. and 4.}
\item \emph{series expansion can supplement 5.}
\item \emph{use integer calculation when possible} come fatto nel metodo
\nameref{sec:metodoDiBisezione}, \`e migliore impostare un ciclo di dimensione
finita e conosciuta al posto di utilizzare una guardia che deve essere ogni
volta valutata.
\end{enumerate}

\subsection{Spacing between machine numbers'}
La spaziatura fra due coppie di numeri di macchina non \`e sempre la stessa, ma
varia per ogni $b^{e_{min} + i}$ con $i = 0, \ldots, e_{max} - e_{min}$.
La spaziatura fra numeri piccoli in valore assoluto (con $i \rightarrow 0$)
\`e pi\`u raffinata rispetto alla spaziatura fra numeri grandi in valore
assoluto (con $i \rightarrow e_{max} - e_{min}$).
