\section{Formule composite} 

\begin{exercise}[5.1] 
Per il testo dell'esercizio consultare il libro di testo.
\end{exercise}
Per la \emph{(5.2)} del libro di testo vale $\kappa = b-a = e^{21} \gg 1$,
quindi il problema risulta molto mal condizionato.

\begin{exercise}[5.2] 
Per il testo dell'esercizio consultare il libro di testo.
\end{exercise}
\begin{proof}[Proof of Coefficienti formula trapezi]
Dalla definizione della formula generica di \emph{Newton-Cotes} istanzio per il
caso dei trapezi:
\begin{displaymath}
\begin{split}
I_{1}(f) &= (b-a) \sum_{k = 0}^{1}{c_{k1}f_{k}} \\
c_{01} &= \int_{0}^{1}{\prod_{j=0,j\not = 0}^{1}{\frac{t-j}{0-j} dt}} = 
\int_{0}^{1}{\frac{t-1}{0-1}dt} = -\int_{0}^{1}{t dt} +
\int_{0}^{1}{dt} = -\left .\frac{t^{2}}{2}\right |_{0}^{1} + \left .t\right |_{0}^{1} =
-\frac{1}{2} + 1 =
 \frac{1}{2} \\
 c_{11} &= \int_{0}^{1}{\prod_{j=0,j\not = 1}^{1}{\frac{t-j}{1-j} dt}} = 
\int_{0}^{1}{\frac{t-0}{1-0}dt} = \int_{0}^{1}{t dt} = \left
.\frac{t^{2}}{2}\right |_{0}^{1} = \frac{1}{2} 
\end{split}
\end{displaymath}
\end{proof}
\begin{proof}[Proof of Coefficienti formula Simpson]
Dalla definizione della formula generica di \emph{Newton-Cotes} istanzio per il
caso di Simpson:
\begin{displaymath}
\begin{split}
I_{2}(f) &= \frac{b-a}{2} \sum_{k = 0}^{2}{c_{k2}f_{k}} \\
c_{02} &= \int_{0}^{2}{\prod_{j=0,j\not = 0}^{2}{\frac{t-j}{0-j} dt}} = 
\int_{0}^{2}{\frac{(t-1)(t-2)}{(0-1)(0-2)}dt} = \frac{1}{2}\int_{0}^{2}{t^{2}
-3t +2 dt} = \\
&= \frac{1}{2}\left(\int_{0}^{2}{t^{2} dt} -3 \int_{0}^{2}{t dt} 
+ 2\int_{0}^{2}{dt}\right) = 
\frac{1}{2}\left(\left.\frac{t^{3}}{3}\right |_{0}^{2} -3 \left.\frac{t^{2}}{2}
\right |_{0}^{2} + 2\left.t\right |_{0}^{2}\right) = \\
&=  \frac{1}{2}\left(\frac{2^{3}}{3} - \frac{3}{2}2^{2} + 2^{2}\right) = 
\frac{1}{2} 2^{2} \left(\frac{2}{3} - \frac{3}{2} + 1\right) =
2\frac{4-9+6}{6} = \frac{1}{3}
\end{split}
\end{displaymath}
\begin{displaymath}
\begin{split}
c_{12} &= \int_{0}^{2}{\prod_{j=0,j\not = 1}^{2}{\frac{t-j}{1-j} dt}} = 
\int_{0}^{2}{\frac{(t-0)(t-2)}{(1-0)(1-2)}dt} = -\int_{0}^{2}{t^{2} -2t dt} = \\
&= -\int_{0}^{2}{t^{2} dt} +2 \int_{0}^{2}{t dt} = 
-\left.\frac{t^{3}}{3}\right |_{0}^{2} +2 \left.\frac{t^{2}}{2}
\right |_{0}^{2} =  -\frac{2^{3}}{3} + 2^{2} = 2^{2}\left(1 - \frac{2}{3}
\right) = \frac{4}{3}
\end{split}
\end{displaymath}
\begin{displaymath}
\begin{split}
c_{22} &= \int_{0}^{2}{\prod_{j=0,j\not = 2}^{2}{\frac{t-j}{2-j} dt}} = 
\int_{0}^{2}{\frac{(t-0)(t-1)}{(2-0)(2-1)}dt} = \frac{1}{2}\int_{0}^{2}{t^{2}
-t dt} = \\ 
&= \frac{1}{2}\left(\int_{0}^{2}{t^{2} dt} - \int_{0}^{2}{t dt}\right) = 
\frac{1}{2}\left(\left.\frac{t^{3}}{3}\right |_{0}^{2} - \left.\frac{t^{2}}{2}
\right |_{0}^{2}\right) = \frac{1}{2}\left(\frac{2^{3}}{3} - \frac{2^{2}}{2}
\right) = \left(\frac{2^{2}}{3} - 1 \right) = \frac{1}{3}
\end{split}
\end{displaymath}
\end{proof}

\begin{exercise}[5.3] 
Per il testo dell'esercizio consultare il libro di testo.
\end{exercise}
\begin{proof}
Parto dalla definizione dell'\emph{errore di quadratura}:
\begin{displaymath}
E_{n}(f) =
\nu_{n}\frac{f^{(n+k)}(\xi)}{(n+k)!}\left(\frac{b-a}{n}\right)^{n+k+1}
\end{displaymath}
Per il \emph{metodo dei trapezi} si fissa $n = 1$, che implica la scelta di $k
= 1$:
\begin{displaymath}
E_{1}(f) =
\nu_{1}\frac{f^{(2)}(\xi)}{2}\left(b-a\right)^{3}
\end{displaymath}
Ricavo $\nu_{1}$:
\begin{displaymath}
\begin{split}
\nu_{1} &= \int_{0}^{1}{\prod_{j=0}^{1}{(t-j)dt}} = 
\int_{0}^{1}{(t-0)(t-1)dt} = \int_{0}^{1}{t^{2}
-t dt} = \\ 
&= \left(\int_{0}^{1}{t^{2} dt} - \int_{0}^{1}{t dt}\right) = 
\left(\left.\frac{t^{3}}{3}\right |_{0}^{1} - \left.\frac{t^{2}}{2}
\right |_{0}^{1}\right) = \frac{1}{3} - \frac{1}{2} = -\frac{1}{6} 
\end{split}
\end{displaymath}
Andando a sostituire nel metodo si verifica l'uguaglianza esposta nel testo
\emph{(5.10)}:
\begin{displaymath}
E_{1}(f) =
-\frac{1}{6}\frac{f^{(2)}(\xi)}{2}\left(b-a\right)^{3} =
-\frac{f^{(2)}(\xi)}{12}\left(b-a\right)^{3}
\end{displaymath}
Per il \emph{metodo dei Simpson} si fissa $n = 2$, che implica la scelta di $k
= 2$:
\begin{displaymath}
E_{2}(f) =
\nu_{2}\frac{f^{(4)}(\xi)}{4!}\left(\frac{b-a}{2}\right)^{5}
\end{displaymath} 
Ricavo $\nu_{2}$:
\begin{displaymath}
\begin{split}
\nu_{2} &= \int_{0}^{2}{t \prod_{j=0}^{2}{(t-j)dt}} = 
\int_{0}^{2}{t(t-0)(t-1)(t-2)dt} = \int_{0}^{2}{t^{2}(t-1)(t-2)dt} = \\
&= \int_{0}^{2}{t^{2}(t^{2}-3t+2)dt} = 
\int_{0}^{2}{t^{4}dt} -3\int_{0}^{2}{t^{3}dt} +2\int_{0}^{2}{t^{2}dt} = 
\left.\frac{t^{5}}{5}\right |_{0}^{2} - 3\left.\frac{t^{4}}{4}\right |_{0}^{2} +
2 \left.\frac{t^{3}}{3}\right |_{0}^{2} = \\
&= \frac{2}{5}2^{4} -\frac{3}{4}2^{4} + \frac{1}{3}2^{4} =
\left(\frac{24-45+20}{60}\right)2^{4} = -\frac{1}{2^{2}\cdot 3 \cdot 5}2^{4} =
-\frac{2^{2}}{3\cdot 5}
\end{split}
\end{displaymath}
Andando a sostituire nel metodo si verifica l'uguaglianza esposta nel testo
\emph{(5.11)}:
\begin{displaymath}
E_{2}(f) =
-\frac{2^{2}}{3\cdot 5}\frac{f^{(4)}(\xi)}{2^{2} \cdot
3 \cdot 2}\left(\frac{b-a}{2}\right)^{5} =
-\frac{f^{(4)}(\xi)}{90}\left(\frac{b-a}{2}\right)^{5}
\end{displaymath}
\end{proof}


