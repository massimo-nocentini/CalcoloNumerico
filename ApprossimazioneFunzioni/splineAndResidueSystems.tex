\section{Spline e Minimi quadrati}

\begin{exercise}[4.16] 
Per il testo dell'esercizio consultare il libro di testo.
\end{exercise} 
Stiamo costruendo un polinomietto della spline cubica relativo all'$i$-esimo
intervallo $[x_{i-1}, x_{i}]$. Il sistema che otteniamo utilizzando lo schema 
\emph{naturale} usa queste condizioni:
\begin{displaymath}
	\forall i \in \{1,\ldots, n-1\}[m_{i} = s_{3}^{(3)}(x_{i})] \wedge m_{0} = 0
	\wedge m_{n} = 0
\end{displaymath}
Rappresentiamo in forma matriciale il modello che ci permette di ricavare le
incognite $m_{k}$:
\begin{displaymath}
\begin{split}
	\begin{bmatrix}
		2 & \xi_{1} \\
		\varphi_{2} & 2 & \xi_{2} \\
			&	\ddots &	\ddots & \ddots \\
			&	&	\varphi_{n-2} & 2 & \xi_{n-2} \\
			&	&	& \varphi_{n-1} & 2
	\end{bmatrix} %times
	\begin{bmatrix}
		m_{1} \\
		m_{2} \\
		\vdots \\
		m_{n-2} \\
		m_{n-1}
	\end{bmatrix} = 6 %times
	\begin{bmatrix}
		f[x_{0}, x_{1}, x_{2}] \\
		f[x_{1}, x_{2}, x_{3}] \\
		\vdots \\
		f[x_{n-3}, x_{n-2}, x_{n-1}] \\
		f[x_{n-2}, x_{n-1}, x_{n}]
	\end{bmatrix} 
\end{split}
\end{displaymath}
Posso fattorizzare la matrice dei coefficienti in questo modo:
\begin{displaymath}
\begin{split}
	L = \begin{bmatrix}
		1 \\
		l_{2} & 1  \\
			&	\ddots &	\ddots \\
			&	&	l_{n-2} & 1 \\
			&	&	& l_{n-1} & 1
	\end{bmatrix} \quad \quad
	U = \begin{bmatrix}
		u_{1} & \xi_{1}\\
		 & u_{2} & \xi_{2} \\
			&	&	\ddots	& \ddots \\
			&	&	& u_{n-2}	& \xi_{n-2} \\
			&	&	&  & u_{n-1}
	\end{bmatrix} 
\end{split}
\end{displaymath}
Adesso considero il prodotto dell'$i$-esima riga di $L$ con la $(i-1)$-esima
colonna di $U$:
\begin{displaymath}
\begin{split}
	\begin{matrix}
		& & i-2 & i-1 & i  \\
		& & \downarrow & \downarrow & \downarrow \\
		\vect{e}_{i}^{T}L = [& & 0 & l_{i} & 1 & & &]
	\end{matrix}
	\quad & \quad
	U \vect{e}_{i-1} = \begin{bmatrix}
		\\ 
		\\ 
		\xi_{i-2} \\ 
		u_{i-1} \\
		0 \\
	 	\\
	 	\\
	\end{bmatrix} 
	\begin{matrix}
		\\ 
		\\ 
		\leftarrow i-2 \\ 
		\leftarrow i-1 \\
		\leftarrow i \\
	 	\\
	 	\\
	\end{matrix} \\
	(\vect{e}_{i}^{T}L )(U \vect{e}_{i-1}) &= l_{i} u_{i-1}
\end{split}
\end{displaymath}
Ma dato che la matrice dei coefficienti \`e diagonale dominante, allora \`e
fattorizzabile $LU$, quindi deve valere:
\begin{displaymath}
	(\vect{e}_{i}^{T} A \vect{e}_{i-1}) = \varphi_{i} = l_{i} u_{i-1}
\end{displaymath}
Adesso considero il prodotto dell'$i$-esima riga di $L$ con la $i$-esima
colonna di $U$:
\begin{displaymath}
\begin{split}
	\begin{matrix}
		& & i-1 & i  \\
		& & \downarrow & \downarrow \\
		\vect{e}_{i}^{T}L = [& & l_{i} & 1 & & &]
	\end{matrix}
	\quad & \quad
	U \vect{e}_{i} = \begin{bmatrix}
		\\ 
		\\ 
		\xi_{i-1} \\ 
		u_{i} \\
	 	\\
	 	\\
	\end{bmatrix} 
	\begin{matrix}
		\\ 
		\\  
		\leftarrow i-1 \\
		\leftarrow i \\
	 	\\
	 	\\
	\end{matrix} \\
	(\vect{e}_{i}^{T}L )(U \vect{e}_{i}) &= l_{i} \xi_{i-1} + u_{i} 
\end{split}
\end{displaymath}
Usando ancora la propriet\`a di diagonale dominanza deve valere:
\begin{displaymath}
	(\vect{e}_{i}^{T} A \vect{e}_{i}) = 2 = l_{i} \xi_{i-1} + u_{i}
\end{displaymath}
Rimane da far vedere che la matrice dei coefficienti (la chiamo $A$) \`e
diagonale dominante. \begin{proof} Dimostro che sia per righe che per colonne
vale l'asserto.
	\begin{itemize}
	  \item \emph{per righe} Per definizione vale $\varphi_{i} + \xi_{i} = 1$.\\ 
	  Osservando la struttura della matrice vediamo che le righe $r_{k} =
	  \vect{e}_{k}^{T}A, \forall k \in \{2, \ldots, n-2\}$ contengono esattamente
	  tre elementi, mentre la riga $\vect{e}_{1}^{T}A$ e la riga
	  $\vect{e}_{n-1}^{T}A$ ne contengono esattamente due. \\
	  Ma questo insieme di elementi che
	  consideriamo contiene l'elemento diagonale, e i vari $\varphi_{k},
	  \xi_{k}$. Posso impostare:
	  \begin{displaymath}
	  	\forall i \in \{ 1, \ldots, n-1\}:
	  	\begin{bmatrix} 
	  		A_{ii} = 2 > \sum_{j = 1, j \not = i}^{n-1}{|A_{ij}|} = \varphi_{i} + 
	  		\xi_{i} = 1
	  	\end{bmatrix}
	  \end{displaymath}
	  e la condizione di diagonale dominanza per righe viene soddisfatta.
	  \item \emph{per colonne} Per definizione vale $\varphi_{i} + \xi_{i} = 1$.\\ 
	  Osservando la struttura della matrice vediamo che le colonne $c_{k} =
	  A\vect{e}_{k}, \forall k \in \{2, \ldots, n-2\}$ contengono esattamente
	  tre elementi, mentre la colonna $A\vect{e}_{1}$ e la colonna
	  $A\vect{e}_{n-1}$ ne contengono esattamente due. \\
	  Ma questo insieme di elementi che
	  consideriamo contiene l'elemento diagonale, e i vari $\varphi_{k},
	  \xi_{k}$. Posso impostare:
	  \begin{displaymath}
	  	\forall j \in \{ 1, \ldots, n-1\}:
	  	\begin{bmatrix} 
	  		A_{jj} = 2 > \sum_{i = 1, i \not = j}^{n-1}{|A_{ij}|} = 
	  		\xi_{i} + \varphi_{i+2}= 1
	  	\end{bmatrix}
	  \end{displaymath}
	  e la condizione di diagonale dominanza per colonne viene soddisfatta.
	\end{itemize}
\end{proof}

\begin{exercise} 
Provare ad interpolare la seguente funzione, in una partizione composta da 12
ascisse di interpolazione:
\begin{displaymath}
f(x) = \left \lbrace 
\begin{matrix}
	(x+5)(x+1)e^{x} & \text{se } x \leq -1 \\
	\sqrt{1 - x^{2}} & \text{se } -1 < x \leq 1 \\
	\frac{x-1}{x^{2}} & \text{se } 1 < x
\end{matrix} 
\right. 
\end{displaymath}
\end{exercise}
Per l'implementazione del codice di questo esercizio vedere il codice
\nameref{subsec:splineStress}.
Nel seguente plot rappresento due interpolazioni con spline cubiche, in verde
utilizzando lo schema \emph{not-a-knot}, mentre in rosso lo schema
\emph{naturale}.
\begin{center}   
\input{ApprossimazioneFunzioni/splineStress-interpolationPlotOutput.tex}
\end{center}
Nel seguente plot rappresento il modello degli errori delle precedenti
interpolazioni con spline cubiche, in rosso riferito allo
schema \emph{not-a-knot}, mentre in blu riferito allo schema \emph{naturale}.
\begin{center}   
\input{ApprossimazioneFunzioni/splineStress-errorsPlotOutput.tex}
\end{center}

\begin{exercise}[4.19] 
Per il testo dell'esercizio consultare il libro di testo. 
\end{exercise}
Riporto il plot delle interpolazioni della funzione di \emph{Runge}, in verde
l'interpolazione eseguita con lo schema \emph{not-a-knot}, in rosso
l'interpolazione eseguita con lo schema \emph{normale}, in blu la curva
originale. Questo plot \`e stato generato con il codice
\nameref{subsec:exercise419RungeInterpolationPlotter} usando il motore\\
\nameref{subsec:exercise419RungeInterpolation}.
\begin{center}   
\input{ApprossimazioneFunzioni/exercise419-rungeInterpolationPlotOutput.tex}
\end{center}
Riporto il plot delle interpolazioni della funzione di \emph{Bernstein}, in
verde l'interpolazione eseguita con lo schema \emph{not-a-knot}, in rosso
l'interpolazione eseguita con lo schema \emph{normale}, in blu la curva
originale. Questo plot \`e stato generato con il codice
\nameref{subsec:exercise419BernsteinInterpolationPlotter} usando il motore\\
\nameref{subsec:exercise419BernsteinInterpolation}.
\begin{center}   
\input{ApprossimazioneFunzioni/exercise419-bernsteinInterpolationPlotOutput.tex}
\end{center}



