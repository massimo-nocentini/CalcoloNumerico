\section{Interpolazione Polinomiale} 

\begin{exercise}[4.1] 
Trovare un polinomio $p(x)$ che interpola la funzione $f(x) = 4x^{2} -12x +1$
nei punti di ascissa $x_{i} = i$, con $ i \in \{ 0,\ldots, 4 \}$.
\end{exercise}
Per l'implementazione vedere il codice \nameref{subsec:exercise41} e per la 
sua esecuzione lanciare lo script \nameref{subsec:scriptForExercise41}.
\begin{center} 
\input{ApprossimazioneFunzioni/exercise41PlotOutput.tex}
\end{center}
In cyan \`e reppresentata la curva della funzione $f$, mentre con i simboli
$+$ sono rappresentati i punti interpolati dal polinomio $p$.

\begin{exercise}[4.2] 
Per il testo dell'esercizio consulare il libro di testo.
\end{exercise}
\begin{proof}
L'algoritmo riportato implementa questo schema (utilizzo gli indici nella
notazione usata nella formulazione matematica, quindi sono zero-based):
\begin{displaymath}
\begin{split}
	p^{(0)}(x) &= a_{n} \\
	p^{(i+1)}(x) &= p^{(i)}x + a_{n-i} 
\end{split}
\end{displaymath}
con $p^{(i)}$ indico il valore di $p$ all'$i$-esimo passo dei esecuzione. 
Se considero il valore di $p$ ad un generico passo $i$ di esecuzione si
osserva che ha questa struttura:
\begin{displaymath}
\begin{split}
	p^{(i)}(x) &= (a_{n}x^{i-1} + a_{n-1}x^{i-2} + \ldots + a_{n-i+1})x + a_{n-i}\\
	&= a_{n}x^{i} + a_{n-1}x^{i-1} + \ldots + a_{n-i+1}x + a_{n-i}
\end{split}
\end{displaymath}
Il polinomio $p^{(i)}$ \`e di grado $i$, quindi saturando l'indice $i$ arrivando
a calcolare $p^{n}$, si ottiene il polinomio:
\begin{displaymath}
	p(x) = p^{(n)}(x) = a_{n}x^{n} + a_{n-1}x^{n-1} + \ldots + a_{1}x + a_{0}
\end{displaymath}
Ovvero quello che si chiede nel problema.
\end{proof}